\documentclass[a4paper,12pt,landscape]{memoir}
\usepackage{fontspec}

% Simplified font configuration
\setmainfont{Noto Serif Devanagari}

% Simple Latin font for scientific names
\newfontfamily{\latintext}{DejaVu Serif}[
    Scale=MatchUppercase
]

% Remove all font options to avoid stack overflow
\newfontfamily{\sectionfont}{Noto Serif Devanagari}[
    Scale=1.2
]

\usepackage{geometry}
\usepackage{graphicx}
\graphicspath{{../images/}}
\usepackage{xcolor}
\usepackage{hyperref}
\usepackage{mdframed}
\usepackage{tikz}
\usetikzlibrary{decorations.pathmorphing}
\usepackage{makeidx}
\makeindex

\geometry{
  a4paper,
  landscape,
  left=1cm,
  right=1cm,
  top=2cm,
  bottom=2cm,
  includehead
}

\chapterstyle{bianchi}
\setsecheadstyle{\sffamily\bfseries\color{headingcolor}\Large}
\setsubsecheadstyle{\sffamily\bfseries\color{headingcolor}}
\definecolor{headingcolor}{RGB}{34, 139, 34}
\definecolor{highlightcolor}{RGB}{139, 69, 19}

\newcommand{\missingimage}[1]{%
  \framebox[0.4\textwidth]{%
    \parbox{0.35\textwidth}{%
      % You can insert any placeholder or text showing a missing image here
    }%
  }%
}

\newcommand{\getcreditname}[1]{%
  \def\stripext##1.jpg{##1}%
  \edef\creditfile{\stripext#1_credit.txt}%
}

\newcounter{birdnumber}
\setcounter{birdnumber}{0}

\makeatletter
\newcommand{\indexnames}[2]{%
  \index{\textit{#2}}%
  \index{#1}%
  \expandafter\makesurnameindex\expandafter{#1}%
}

\newcommand{\makesurnameindex}[1]{%
  \def\two##1 ##2\@nil{%
    \ifx\relax##2\relax
      % Surname/Firstname logic if desired
    \fi
  }%
  \expandafter\two#1 \@nil
}
\makeatother

% Update bird entry template to match bird_guide.tex
\newcommand{\birdentry}[9]{%
  \stepcounter{birdnumber}%
  \indexnames{#1}{#2}%
  \begin{minipage}[t]{0.48\textwidth}
    \begin{mdframed}[
      linecolor=highlightcolor,
      linewidth=1pt,
      roundcorner=5pt,
      leftmargin=0pt,
      rightmargin=0pt
    ]
      \begin{center}
      \IfFileExists{../images/#9}{%
        \includegraphics[width=0.95\textwidth,height=0.8\textheight,keepaspectratio]
        {../images/#9}%
        \par\vspace{-1em}%
        \getcreditname{#9}%
        \hfill{\small\em\latintext Credit: \input{../images/\creditfile}}%
      }{%
        \missingimage{#1}%
      }%
      \end{center}
    \end{mdframed}
  \end{minipage}\hfill
  \begin{minipage}[t]{0.48\textwidth}
    {\sectionfont\bfseries\thebirdnumber. #1} {\latintext (#2)}%
    \begin{mdframed}[
      linecolor=headingcolor,
      linewidth=1pt,
      roundcorner=5pt,
      leftmargin=0pt,
      rightmargin=0pt,
      backgroundcolor=headingcolor!5
    ]
      {\sffamily\bfseries Size:} #3 \\[0.5em]
      {\sffamily\bfseries Status:} #4 \\[0.5em]
      {\sffamily\bfseries Field characters:} #5 \\[0.5em]
      {\sffamily\bfseries Distribution:} #6 \\[0.5em]
      {\sffamily\bfseries Habits:} #7 \\[0.5em]
      {\sffamily\bfseries Nesting:} #8
    \end{mdframed}
  \end{minipage}
  \newpage
}

\newcommand{\introsection}[2]{%
  \begin{minipage}[t]{0.48\textwidth}
    \begin{mdframed}[
      linecolor=headingcolor,
      linewidth=1pt,
      roundcorner=5pt,
      leftmargin=0pt,
      rightmargin=0pt,
      backgroundcolor=headingcolor!5
    ]
      #1
    \end{mdframed}
  \end{minipage}\hfill
  \begin{minipage}[t]{0.48\textwidth}
    \begin{mdframed}[
      linecolor=headingcolor,
      linewidth=1pt,
      roundcorner=5pt,
      leftmargin=0pt,
      rightmargin=0pt,
      backgroundcolor=headingcolor!5
    ]
      #2
    \end{mdframed}
  \end{minipage}
  \newpage
}

\makepagestyle{birdguide}

\title{पुणे विद्यापीठातील पक्षी}
\author{}
\date{}

\begin{document}
\maketitle

\begin{center}
\vspace{1cm}
{\large{\latintext \textcopyright} 2024 योगेश वाडदेकर {\latintext (wadadekar@gmail.com)}}\\
सर्व हक्क राखीव.
\end{center}

\tableofcontents

\chapter*{परिचय}
\introsection{%
  \section*{परिसराचा आढावा}
  सावित्रीबाई फुले पुणे विद्यापीठ कॅम्पस हे पुण्याच्या मध्यभागी पसरलेले एक समृद्ध जैवविविधता क्षेत्र आहे. अंदाजे ४११ एकरांवर (१६६.३३ हेक्टर) व्यापलेल्या या परिसरात वृक्षराजी, गवताळ भाग, तळी आणि इमारती असा संमिश्र निसर्ग पाहायला मिळतो. 
}{%
  \section*{पाणस्रोत आणि माणसाचा प्रभाव}
  कॅम्पसमधील पाण्याचे स्रोत पक्ष्यांसाठी महत्त्वाचे अधिवास तयार करतात:
  \begin{itemize}
  \item कृत्रिम तळे
  \item लहान तळी आणि परिसरातले पाणी साठणारे भाग
  \item पावसाळ्यात तयार होणारे नैसर्गिक पाणथळ भाग
  \end{itemize}
}

\introsection{%
  \section*{संवर्धन प्रयत्न}
  विद्यापीठ प्रशासन कॅम्पसमधील जैवविविधता राखण्यासाठी खालील उपाययोजना राबवते:
  \begin{itemize}
  \item शांत क्षेत्रांची निर्मिती
  \item वृक्षाच्छादन आणि प्रौढ वृक्षांचे संवर्धन
  \item संवेदनशील अधिवासांसाठी निर्बंधित प्रवेश
  \item पक्षिजनगणना व सर्वेक्षण
  \item विद्यार्थ्यांसाठी पर्यावरण जागरूकता कार्यक्रम
  \end{itemize}
}{%
  \section*{या मार्गदर्शिकेबद्दल}
  या मार्गदर्शिकेत कॅम्पसमधील विविध पक्ष्यांची माहिती, त्यांच्या ओळखीची वैशिष्ट्ये, अधिवास, अन्नस्रोत आणि प्रजनन याविषयी तपशील आहे. नावे विद्यमान वर्गीकरणाप्रमाणे दिली असून, Book of Indian Birds आणि Indian Birds जर्नल येथील अद्यतनांचा आधार घेतला आहे.
}

\vspace{1cm}
\chapter{पक्षी}

\birdentry{काळा घार}{\textit{Milvus migrans}}
  {घारीएवढा}
  {अतिसामान्य रहिवासी}
  {मोठा शिकारी पक्षी, विशिष्ट काटेरी शेपटी आणि लांब पंख. पंख काळसर-तपकिरी, खालचा भाग फिकट.}
  {पुणे विद्यापीठ कॅम्पससह संपूर्ण भारतीय उपखंडात आढळतो.}
  {नेहमी आकाशात उंच गरगरताना दिसतो, शिकार शोधत असतो. लहान सस्तन, पक्षी आणि मृत प्राणी खातो.}
  {डिसेंबर ते एप्रिल दरम्यान प्रजनन. उंच झाडांवर काड्या आणि वनस्पतींपासून घरटे बांधतो. २-३ पांढरी अंडी घालतो, त्यावर तपकिरी ठिपके असतात. दोन्ही पालक ३० दिवस अंडी उबवतात आणि पिल्लांची काळजी घेतात.}
  {black-kite.jpg}

\birdentry{शिक्रा}{\textit{Accipiter badius}}
  {मैना+}
  {सामान्य रहिवासी}
  {लहान ससाणा, लाल डोळे आणि पट्टेदार छाती विशिष्ट. वरचा भाग करडा, खालचा भाग पांढरा, त्यावर तांबूस पट्टे.}
  {पुणे विद्यापीठ कॅम्पससह संपूर्ण भारतीय उपखंडात व्यापक वितरण.}
  {वृक्षराजी आणि बागा पसंत करतो. लहान पक्षी, सस्तन आणि कीटक खातो.}
  {मार्च ते जुलै दरम्यान प्रजनन. झाडांवर फांद्या आणि पाने वापरून घरटे बांधतो. ३-४ फिकट निळसर-पांढरी अंडी. मादी बहुतांश अंडी उबवते (२८-३० दिवस), नर अन्न पुरवतो.}
  {shikra.jpg}

\birdentry{पाणकोंबडी}{\textit{Amaurornis phoenicurus}}
  {बुलबुल+}
  {सामान्य रहिवासी}
  {मध्यम आकाराचा पक्षी, पांढरा चेहरा, गळा आणि छाती, गडद तपकिरी शरीर. चोचीच्या मुळाशी लाल पट्टा विशिष्ट.}
  {पुणे विद्यापीठ कॅम्पससह संपूर्ण भारतीय उपखंडात आढळतो.}
  {दलदलीचे भाग आणि पाणथळ जागा पसंत करतो. कीटक, लहान मासे आणि वनस्पती खातो.}
  {पाण्याजवळील दाट वनस्पतीत घरटे बनवतो. ४-६ क्रीमी पांढरी अंडी घालतो, त्यावर तांबूस-तपकिरी ठिपके असतात. दोन्ही पालक १९-२० दिवस अंडी उबवतात आणि पिल्लांची काळजी घेतात.}
  {white-breasted-waterhen.jpg}

\birdentry{राणी बगळा}{\textit{Ardeola grayii}}
  {मैना+}
  {अतिसामान्य रहिवासी}
  {मध्यम आकाराचा बगळा, तपकिरी शरीर आणि पांढरे पंख. पिवळी चोच आणि पाय विशिष्ट.}
  {पुणे विद्यापीठ कॅम्पससह संपूर्ण भारतीय उपखंडात आढळतो.}
  {पाणथळ भाग, तळी आणि दलदल पसंत करतो. मासे, बेडूक आणि कीटक खातो.}
  {जून ते सप्टेंबर दरम्यान प्रजनन. पाण्याजवळील झाडे किंवा झुडपांमध्ये काड्या आणि वनस्पतींपासून घरटे बांधतो. ३-५ फिकट निळसर-हिरवी अंडी घालतो. दोन्ही पालक १८-२४ दिवस अंडी उबवतात आणि पिल्लांना खाऊ घालतात.}
  {indian-pond-heron.jpg}

\birdentry{रात्रीचा बगळा}{\textit{Nycticorax nycticorax}}
  {कावळा+}
  {दुर्मिळ रहिवासी}
  {मध्यम आकाराचा बगळा, काळी टोपी आणि पाठ, करडे पंख आणि पांढरा खालचा भाग. लाल डोळे आणि जाड चोच विशिष्ट.}
  {पुणे विद्यापीठ कॅम्पससह संपूर्ण भारतीय उपखंडात आढळतो.}
  {पाणथळ भाग, तळी आणि दलदल पसंत करतो. मासे, बेडूक आणि कीटक खातो.}
  {जून ते सप्टेंबर दरम्यान प्रजनन. पाण्याजवळील झाडे किंवा झुडपांमध्ये काड्या आणि वनस्पतींपासून घरटे बांधतो. ३-५ फिकट निळसर-हिरवी अंडी घालतो. दोन्ही पालक २४-२६ दिवस अंडी उबवतात आणि पिल्लांची काळजी घेतात.}
  {black-crowned-night-heron.jpg}

\birdentry{गवती बगळा}{\textit{Bubulcus ibis}}
  {मैना++}
  {अतिसामान्य रहिवासी}
  {मध्यम आकाराचा बगळा, पांढरा रंग आणि पिवळी चोच. प्रजननकाळात डोक्यावर, छातीवर आणि पाठीवर केशरी-पिवळे पिसारे येतात.}
  {पुणे विद्यापीठ कॅम्पससह संपूर्ण भारतीय उपखंडात आढळतो.}
  {नेहमी गुरांच्या जवळपास दिसतो, चरणाऱ्या जनावरांमुळे उडणारे कीटक आणि लहान प्राणी खातो.}
  {जून ते सप्टेंबर दरम्यान पावसाळ्यात प्रजनन. वसाहतींमध्ये काड्या आणि वनस्पतींपासून घरटे बांधतो. ३-४ फिकट निळी अंडी घालतो. दोन्ही पालक २३-२६ दिवस अंडी उबवतात आणि पिल्लांना खाऊ घालतात.}
  {cattle-egret.jpg}

\birdentry{लहान बगळा}{\textit{Egretta garzetta}}
  {मैना++}
  {सामान्य रहिवासी}
  {मध्यम आकाराचा बगळा, पांढरा रंग, काळी चोच आणि काळे पाय, पिवळे पंजे. प्रजननकाळात डोक्यावर, छातीवर आणि पाठीवर नाजूक पिसारे येतात.}
  {पुणे विद्यापीठ कॅम्पससह संपूर्ण भारतीय उपखंडात आढळतो.}
  {पाणथळ भाग, तळी आणि दलदल पसंत करतो. मासे, बेडूक आणि कीटक खातो.}
  {जून ते सप्टेंबर दरम्यान प्रजनन. वसाहतींमध्ये काड्या आणि वनस्पतींपासून घरटे बांधतो. ३-५ फिकट निळसर-हिरवी अंडी घालतो. दोन्ही पालक २१-२५ दिवस अंडी उबवतात आणि पिल्लांची काळजी घेतात.}
  {little-egret.jpg}

\birdentry{पारवा}{\textit{Columba livia}}
  {मैना+}
  {अतिसामान्य रहिवासी}
  {जाडजूड करडा पक्षी, मानेवर इंद्रधनुष्यी पिसारे. दोन काळे पट्टे पंखांवर आणि शेपटीच्या टोकावर काळा पट्टा हे विशेष लक्षण.}
  {पुणे विद्यापीठ कॅम्पससह संपूर्ण भारतीय उपखंडात, विशेषतः शहरी भागात विपुल.}
  {शहरी जीवनाशी अनुकूलित. कळपांमध्ये जमिनीवर धान्य आणि बिया खाताना दिसतो.}
  {वर्षभर प्रजनन करतो. इमारतींच्या कपाऱ्यात, खिडक्यांच्या कठड्यांवर आणि इतर कृत्रिम जागी काड्या आणि छोट्या फांद्यांपासून घरटे बांधतो. २ पांढरी अंडी घालतो. दोन्ही पालक १७-१९ दिवस अंडी उबवतात आणि पिल्लांना खाऊ घालतात.}
  {blue-rock-pigeon.jpg}

\birdentry{ठिपक्यांचा पारवा}{\textit{Streptopelia chinensis}}
  {मैना}
  {सामान्य रहिवासी}
  {बारीक, लांब शेपटीचा पारवा, गुलाबी-करडा रंग, मानेवर विशिष्ट काळे-पांढरे ठिपके.}
  {कॅम्पस आणि आजूबाजूच्या भागात सर्वत्र आढळतो.}
  {नेहमी जोडीने किंवा छोट्या कळपात जमिनीवर धान्य आणि बिया खाताना दिसतो.}
  {झाडे आणि झुडपांमध्ये सैल घरटे बांधतो. २ पांढरी अंडी घालतो. दोन्ही पालक १४-१६ दिवस अंडी उबवतात आणि पिल्लांची काळजी घेतात.}
  {spotted-dove.jpg}

\birdentry{छोटा पारवा}{\textit{Streptopelia senegalensis}}
  {मैना-}
  {सामान्य रहिवासी}
  {लहान आणि नाजूक पारवा, फिकट तपकिरी रंग, मानेवर जांभळी छटा. मानेच्या बाजूंवर काळे-लाल चौकटी नमुना.}
  {कॅम्पसमध्ये सर्वत्र, विशेषतः मोकळ्या भागात आढळतो.}
  {नेहमी जोडीने जमिनीवर बिया खाताना दिसतो. गोड संगीतमय आवाज करतो.}
  {झुडपे आणि लहान झाडांवर साधे काड्यांचे घरटे बांधतो. २ पांढरी अंडी घालतो. दोन्ही पालक १३-१५ दिवस अंडी उबवतात आणि पिल्लांना खाऊ घालतात.}
  {little-brown-dove.jpg}

\birdentry{लाल टिकली टिटवी}{\textit{Vanellus indicus}}
  {मैना++}
  {सामान्य रहिवासी}
  {मोठी टिटवी, डोळ्यांसमोर लाल मांसल वाढ विशिष्ट. तपकिरी पंख, काळे डोके आणि छाती, पांढरा चेहरा आणि खालचा भाग.}
  {कॅम्पसमधील मोकळ्या भागात, विशेषतः पाणवठ्यांजवळ आढळते.}
  {त्यांच्या "डिड-ही-डू-इट" या धोक्याच्या आवाजासाठी प्रसिद्ध. दिवसा आणि रात्री सक्रिय, कीटक आणि लहान प्राणी खाते.}
  {जमिनीवर उथळ खड्ड्यात, बहुधा खडकाळ किंवा दगडगोट्यांच्या भागात घरटे बांधते. ३-४ जैतूनी-तपकिरी अंडी घालते, त्यावर काळे ठिपके असतात. दोन्ही पालक २८-३० दिवस अंडी उबवतात आणि पिल्लांची काळजी घेतात.}
  {red-wattled-lapwing.jpg}

\birdentry{कोकिळा}{\textit{Eudynamys scolopacea}}
  {कावळा-}
  {सामान्य रहिवासी}
  {नर चमकदार काळा, लाल डोळे; मादी तपकिरी, पांढरे ठिपके, खालून पट्टेदार. लिंगभेद स्पष्ट.}
  {कॅम्पसमध्ये सर्वत्र, विशेषतः वृक्षराजीत आढळतो.}
  {विशिष्ट "कू-ऊ" आवाजासाठी प्रसिद्ध. कावळ्यांच्या घरट्यात परजीवी. फळे आणि बेरी खातो.}
  {कावळे आणि इतर पक्ष्यांच्या घरट्यात परजीवी म्हणून अंडी घालतो.}
  {asian-koel.jpg}

\birdentry{भारद्वाज}{\textit{Centropus sinensis}}
  {कावळा}
  {सामान्य रहिवासी}
  {मोठा काळा पक्षी, केशरी पंख आणि लांब शेपटी. लाल डोळे आणि वाकडी काळी चोच विशिष्ट.}
  {कॅम्पसमधील दाट वनस्पतीत आढळतो.}
  {लपून-छपून वावरतो, जमिनीवर चालताना दिसतो. खोल गुरगुरण्याचा आवाज. कीटक, लहान प्राणी खातो.}
  {दाट वनस्पतीत जमिनीजवळ छतासारखे घरटे बांधतो. ३-५ पांढरी अंडी घालतो. दोन्ही पालक १५-१६ दिवस अंडी उबवतात आणि पिल्लांची काळजी घेतात.}
  {greater-coucal.jpg}

\birdentry{राजपोपट}{\textit{Psittacula krameri}}
  {मैना++}
  {अतिसामान्य रहिवासी}
  {हिरवा पोपट, लांब शेपटी. नराच्या मानेभोवती गुलाबी आणि काळा पट्टा, मादीला हा पट्टा नसतो.}
  {कॅम्पसमध्ये सर्वत्र, विशेषतः वृक्षराजीत विपुल.}
  {आवाजी, कळपवासी पक्षी. मोठ्या कळपात दिसतो. फळे, बिया आणि धान्य खातो.}
  {डिसेंबर ते मे दरम्यान प्रजनन. झाडांच्या पोकळीत घरटे करतो, इतर पक्ष्यांशी स्पर्धा करतो. ३-४ पांढरी अंडी घालतो. दोन्ही पालक २२-२४ दिवस अंडी उबवतात आणि पिल्लांना खाऊ घालतात.}
  {rose-ringed-parakeet.jpg}

\birdentry{पेडगी पोपट}{\textit{Psittacula eupatria}}
  {कावळा}
  {दुर्मिळ रहिवासी}
  {राजपोपटापेक्षा मोठा, जाड लाल चोच. नराला गुलाबी आणि काळा गळपट्टा.}
  {कॅम्पसच्या वृक्षराजीत आढळतो, इतर पोपटांपेक्षा कमी.}
  {राजपोपटासारखीच सवये पण जास्त सावध. फळे आणि बिया खातो.}
  {मोठ्या जुन्या झाडांच्या पोकळीत घरटे करतो. २-४ पांढरी अंडी घालतो. दोन्ही पालक २३-२६ दिवस अंडी उबवतात आणि पिल्लांची काळजी घेतात.}
  {alexandrine-parakeet.jpg}

\birdentry{जांभळा डोका पोपट}{\textit{Psittacula cyanocephala}}
  {मैना}
  {सामान्य रहिवासी}
  {नराचे डोके जांभळे-लाल, मादीचे निळसर-करडे. दोघांचेही शरीर हिरवे आणि पिवळट शेपटीचे टोक.}
  {कॅम्पसच्या वृक्षराजीत नियमित भेटतो.}
  {नेहमी जोडीने किंवा लहान कळपात. इतर पोपटांपेक्षा जास्त झाडावर राहतो. मुख्यतः फळे आणि बिया खातो.}
  {झाडांच्या पोकळीत घरटे करतो, प्रजनन काळ स्थानपरत्वे बदलतो. ३-४ पांढरी अंडी घालतो. दोन्ही पालक २२-२४ दिवस अंडी उबवतात आणि पिल्लांना खाऊ घालतात.}
  {plum-headed-parakeet.jpg}

\chapter*{अनुक्रमणिका}
\addcontentsline{toc}{chapter}{अनुक्रमणिका}
\printindex
\end{document}