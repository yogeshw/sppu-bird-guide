% Document class and core packages
\documentclass[a4paper,12pt,landscape]{memoir}
\usepackage{geometry}
\usepackage{fontspec}
\usepackage{unicode-math}
\usepackage{polyglossia}
\usepackage{microtype}
\usepackage[noautomatic]{imakeidx}
\usepackage{xparse}
\usepackage{xcolor}
\usepackage{mdframed}
\usepackage{graphicx}
\usepackage{hyperref}

\geometry{
  a4paper,
  landscape,
  left=1cm,
  right=1cm,
  top=1.5cm,
  bottom=1.5cm,
  includehead
}

% Configure index
\makeindex[name=birdindex,title=सूची,columns=1,options=-s bird_guide_marathi.ist]

% Define indexing commands
\makeatletter
\newcommand{\indexnames}[2]{%
  \index[birdindex]{#1@{\devanagarifont #1}}% Devanagari name
  \index[birdindex]{#2@{\latintext #2}}% Scientific name
}

\def\@idxitem{\par\hangindent 1em}
\def\subitem{\par\hangindent 2em}
\def\subsubitem{\par\hangindent 3em}
\makeatother

% Set index formatting
\renewcommand{\indexname}{सूची}
\renewcommand{\indexspace}{\vspace{10pt plus 2pt minus 3pt}}


% Set default languages
\setdefaultlanguage{marathi}
\setotherlanguage{english}

\newfontfamily\latinfont{Latin Modern Roman}         % a font with a proper bullet glyph


% Set up main Devanagari font with simpler configuration
\setmainfont{Noto Serif Devanagari}[
    Script=Devanagari,
    Language=Marathi,
    UprightFont=*-Regular,
    BoldFont=*-Bold,
    SlantedFont=*-Regular,  
    BoldSlantedFont=*-Bold, 
    Scale=1.2,
    Mapping=devanagarinumerals
]

% Set up Latin font for English text
\newfontfamily{\latintext}[
    Script=Latin,
    UprightFont=NotoSerif-Regular,
    BoldFont=NotoSerif-Bold,
    ItalicFont=NotoSerif-Italic,
    BoldItalicFont=NotoSerif-BoldItalic,
    Scale=1.0
]{Noto Serif}

% Add separate monospace font family
\newfontfamily{\latinmono}{Noto Sans Mono}

% Explicitly set up Devanagari font family for headers and special text
\newfontfamily{\devanagarifont}[
    Script=Devanagari,
    Language=Marathi,
    UprightFont=*-Regular,
    BoldFont=*-Bold,
    Scale=1.2
]{Noto Serif Devanagari}

% Add Devanagari sans-serif font definition
\newfontfamily{\devanagarifontsf}[
    Script=Devanagari,
    Language=Marathi,
    UprightFont=*-Regular,
    BoldFont=*-Bold,
    Scale=1.2
]{Noto Sans Devanagari}

% Ensure proper font is loaded at start
\AtBeginDocument{%
    \devanagarifont
    \raggedright  % Better for Devanagari text
}

% Set up bullet point symbol
\renewcommand{\labelitemi}{{\latinfont\textbullet}}

% Define sectionfont command before using it
\newcommand{\sectionfont}{\devanagarifont}

% Section font configuration
\setsecheadstyle{{\devanagarifont\bfseries\color{headingcolor}\Large}}
\setsubsecheadstyle{{\devanagarifont\bfseries\color{headingcolor}}}

% Header font configuration
\makepagestyle{birdguide}
\makepsmarks{birdguide}{%
  \let\ps@plain\ps@empty
  \def\chaptermark##1{\markboth{##1}{}}%
  \def\sectionmark##1{\markright{##1}}%
}

% Colors and other configurations remain unchanged
\definecolor{headingcolor}{RGB}{34, 139, 34}
\definecolor{highlightcolor}{RGB}{139, 69, 19}

\newcommand{\missingimage}[1]{%
  \framebox[0.4\textwidth]{%
    \parbox{0.35\textwidth}{%
      % You can insert any placeholder or text showing a missing image here
    }%
  }%
}

\newcommand{\getcreditname}[1]{%
  \def\stripext##1.jpg{##1}%
  \edef\creditfile{\stripext#1_credit.txt}%
}

\newcounter{birdnumber}
\setcounter{birdnumber}{0}

\newcommand{\birdentry}[9]{%
  \stepcounter{birdnumber}%
  \indexnames{#1}{#2}%
  \begin{minipage}[t]{0.48\textwidth}
    \begin{mdframed}[
      linecolor=highlightcolor,
      linewidth=1pt,
      roundcorner=5pt,
      leftmargin=0pt,
      rightmargin=0pt
    ]
      \begin{center}
      \IfFileExists{../images/#9}{%
        \includegraphics[width=0.95\textwidth,height=0.8\textheight,keepaspectratio]
        {../images/#9}
        \par\vspace{1em}
        \getcreditname{#9}%
        \hfill{\small\em\latintext Credit: \input{../images/\creditfile}}%
      }{%
        \missingimage{#1}%
      }%
      \end{center}
    \end{mdframed}
  \end{minipage}\hfill
  \begin{minipage}[t]{0.48\textwidth}
    {\devanagarifont\bfseries\thebirdnumber. #1} {\latintext (#2)}%
    \begin{mdframed}[
      linecolor=headingcolor,
      linewidth=1pt,
      roundcorner=5pt,
      leftmargin=0pt,
      rightmargin=0pt,
      backgroundcolor=headingcolor!5
    ]
      {\latintext\bfseries Size:} #3 \\[0.5em]
      {\latintext\bfseries Status:} #4 \\[0.5em]
      {\latintext\bfseries Field characters:} #5 \\[0.5em]
      {\latintext\bfseries Distribution:} #6 \\[0.5em]
      {\latintext\bfseries Habits:} #7 \\[0.5em]
      {\latintext\bfseries Nesting:} #8
    \end{mdframed}
  \end{minipage}
  \newpage
}

\newcommand{\introsection}[2]{%
  \begin{minipage}[t]{0.48\textwidth}
    \begin{mdframed}[
      linecolor=headingcolor,
      linewidth=1pt,
      roundcorner=5pt,
      leftmargin=0pt,
      rightmargin=0pt,
      backgroundcolor=headingcolor!5
    ]
      #1
    \end{mdframed}
  \end{minipage}\hfill
  \begin{minipage}[t]{0.48\textwidth}
    \begin{mdframed}[
      linecolor=headingcolor,
      linewidth=1pt,
      roundcorner=5pt,
      leftmargin=0pt,
      rightmargin=0pt,
      backgroundcolor=headingcolor!5
    ]
      #2
    \end{mdframed}
  \end{minipage}
  \newpage
}

\makepagestyle{birdguide}

\title{सावित्रीबाई फुले पुणे विद्यापीठ परिसरातले पक्षी}
\author{}
\date{}

\begin{document}
\maketitle

% Add license text after title
\begin{center}
\vspace{1cm}
{\large हे पुस्तक  क्रिएटिव्ह कॉमन्स अट्रिब्यूशन-नॉनकमर्शियल-शेअरअलाइक ४.० आंतरराष्ट्रीय परवान्यांतर्गत ({\latintext CC BY-NC-SA 4.0}) उपलब्ध आहे.}

\vspace{0.5cm}
{\normalsize आपल्याला खालील गोष्टींची मुभा आहे:
\begin{itemize}
\item शेअर करणे -- कोणत्याही माध्यमातून किंवा स्वरूपात सामग्री कॉपी व वितरित करणे
\item रुपांतर -- सामग्रीचे मिश्रण, रूपांतर व त्यावर आधारित नवीन सामग्री तयार करणे
\end{itemize}

खालील अटींच्या अधीन:
\begin{itemize}
\item श्रेयनिर्देश -- आपण योग्य श्रेय द्यावे, परवान्याची लिंक द्यावी, आणि बदल केले असल्यास ते नमूद करावे.
\item अव्यावसायिक -- आपण या सामग्रीचा व्यावसायिक वापर करू शकत नाही.
\item समान शेअरिंग -- आपण सामग्रीत बदल करून नवीन सामग्री तयार केल्यास, ती मूळ सामग्रीच्याच परवान्यांतर्गत वितरित करावी.
\end{itemize}
}

\vspace{0.5cm}
{\small अधिक माहितीसाठी भेट द्या: 
{\latintext \href{https://creativecommons.org/licenses/by-nc-sa/4.0/}{https://creativecommons.org/licenses/by-nc-sa/4.0/}}}

\vspace{1cm}

{\large{\latintext \textcopyright} २०२५ योगेश वाडदेकर}
\end{center}

%\tableofcontents

\chapter*{परिचय}

\introsection{%
  \section*{\textbf{परिसराचा आढावा}}
  सावित्रीबाई फुले पुणे विद्यापीठ ({\latintext SPPU}) परिसर, शहराच्या मध्यभागी असलेले ४११ एकर (१६६.३३ हेक्टर) क्षेत्रफळ असलेले 
  जैवविविधता हॉटस्पॉट आहे. {\latintext 18.5529°N 73.8352°E} अक्षांश-रेखांशावर, हा परिसर समुद्रसपाटीपासून अंदाजे ५६० मीटर उंचीवर आहे.
}{%
  \section*{कॅम्पसचे वातावरण}
  कॅम्पसमध्ये विविध अधिवास आहेत, जे विविध प्रकारच्या पक्ष्यांसाठी उपयुक्त आहेत. या भूभागात मूळ आणि विदेशी झाडे, गवताळ प्रदेश, लॉन, कृत्रिम जलाशय,
   बागा आणि इमारती आहेत. अधिवासांचे हे मिश्रण स्थानिक आणि स्थलांतरित पक्ष्यांसाठी उत्कृष्ट संधी प्रदान करते.
}

\introsection{%
  \section*{\textbf{वनस्पती आणि झाडी}}
  विद्यापीठाच्या परिसरात १५० हून अधिक प्रकारची झाडे आहेत, ज्यामुळे हे एक शहरी वन बनले आहे. 
  वड ({\latintext \textit{Ficus benghalensis}}), पिंपळ ({\latintext \textit{Ficus religiosa}}), 
  कडुलिंब ({\latintext\textit{Azadirachta indica}}) आणि काटेसावर  ({\latintext \textit{Bombax ceiba}}) 
  यांसारख्या मूळ प्रजाती भूभागावर वर्चस्व गाजवतात, तर शोभेची आणि आयात केलेली झाडे जसे आफ्रिकन ब्लॅकवुड 
  ({\latintext \textit{Dalbergia melanoxylon}}), रेन ट्री ({\latintext{}\textit{Samanea saman}}), गुलमोहर ({\latintext\textit{Delonix regia}}) 
  आणि कॉपर पॉड ({\latintext \textit{Peltophorum pterocarpum}}) विविधतेत भर घालतात. ही समृद्ध वनस्पती विविध पक्षी प्रजातींसाठी आवश्यक घरटी बांधण्याची जागा, 
  निवारा आणि अन्नाचे स्रोत पुरवते.
}{%
  \section*{\textbf{मानवी हस्तक्षेप}}
  कॅम्पस परिसंस्थेवर विविध मानवी क्रियाकलापांचा दबाव आहे, ज्यामुळे पक्ष्यांच्या लोकसंख्येवर परिणाम होतो:
  \begin{itemize}
  \item कॅम्पस रस्त्यांवर नियमित वाहनांची वर्दळ
  \item चालू असलेली बांधकाम आणि विकासकामे
  \item मोठ्या संख्येने विद्यार्थ्यांची उपस्थिती
  \item नियमित देखभाल क्रिया जसे बागकाम
  \item शहरी भागातून येणारे आवाज आणि धूळ प्रदूषण
  \item कॅम्पस इमारती आणि सुविधांमधून होणारे प्रकाश प्रदूषण
  \end{itemize}
}

\introsection{%
  \section*{\textbf{कधी पाहावे}}
  \textbf{दिवसाची वेळ:}
  \begin{itemize}
  \item पहाटे (सकाळी ६:००-९:००): पक्षी नाश्त्यासाठी फिरत असल्याने सर्वाधिक क्रियाशील
  \item दुपारनंतर (दुपारी ४:००-६:३०): दुसरी {\latintext Feeding} ची {\latintext activity}
  \item मध्यान: {\latintext Thermal currents} वर उडणाऱ्या शिकारी पक्ष्यांना पाहण्यासाठी उत्तम
  \item संध्याकाळ: घुबडांसारख्या निशाचर प्रजाती सक्रिय होण्याची उत्तम वेळ
  \end{itemize}
}{%
  \section*{\textbf{हंगामानुसार:}}
  \begin{itemize}
  \item हिवाळा (नोव्हेंबर-फेब्रुवारी): अनेक हिवाळी पाहुण्यांमुळे सर्वोत्तम
  \item पावसाळा (जून-सप्टेंबर): {\latintext Breeding activities} पाहण्यासाठी सर्वोत्तम
  \item उन्हाळा (मार्च-मे): स्थानिक {\latintext Breeding birds} पाहण्यासाठी चांगला
  \item पावसाळ्यानंतर (ऑक्टोबर): स्थलांतरित पक्षी शोधण्याची संधी
  \end{itemize}
}

\introsection{%
  \section*{\textbf{आभार}}
  {\latintext SPPU} कॅम्पसमधील पक्ष्यांची नोंद {\latintext eBird} द्वारे ({\latintext
  \href{https://ebird.org/hotspot/L1838309}{SPPU eBird Hotspot}})
  करण्यात ज्या पक्षी निरीक्षकांनी योगदान दिले त्यांचे विशेष आभार.
}{%
  त्यांच्या पद्धतशीर नोंदीमुळे कॅम्पसमधील पक्ष्यांची विविधता आणि हंगामी बदल समजण्यास मदत झाली आहे.
}

\introsection{%
  \section*{\textbf{कसे पाहावे}}
  \textbf{आवश्यक उपकरणे:}
  \begin{itemize}
  \item दुर्बीण: नवशिक्यांसाठी {\latintext 8 X 42} किंवा {\latintext 10 X 42} उत्तम
  \item {\latintext Field guide: Grimmett, Inskipp \& Inskipp} यांचे {\latintext \textit{Birds of the Indian Subcontinent}}
  \item नोंदी घेण्यासाठी वही
  \end{itemize}
}{%
  \section*{\textbf{उपयुक्त ॲप्स:}}
  \begin{itemize}
  \item {\latintext eBird}: नोंदी करण्यासाठी आणि हॉटस्पॉट एक्सप्लोर करण्यासाठी
  \item {\latintext Merlin Bird ID}: फोटो किंवा आवाजावरून पक्षी ओळखण्यास मदत करते
  \item {\latintext BirdNet}: {\latintext AI}-आधारित पक्ष्यांच्या आवाजाची ओळख
  \end{itemize}
}

\introsection{%
  \section*{\textbf{अधिक टिप्स}}
  \begin{itemize}
  \item आजूबाजूच्या वातावरणाशी जुळण्यासाठी माती रंगाचे कपडे वापरा
  \item पक्ष्यांना त्रास होऊ नये म्हणून हळू आणि शांतपणे चाला
  \item फ्लॅश फोटोग्राफी टाळा
  \item घरटी आणि {\latintext Breeding areas} पासून सुरक्षित अंतर ठेवा
  \item प्रत्येक {\latintext Session} नंतर {\latintext eBird} मध्ये आपल्या नोंदी रेकॉर्ड करा
  \end{itemize}
}{%
  \section*{\textbf{या मार्गदर्शिकेबद्दल}}
  या पुस्तिकेत विद्यापीठ परिसरात आढळणाऱ्या विविध पक्षी प्रजातींची माहिती आहे. कॅम्पसमध्ये नोंदवलेल्या पक्ष्यांच्या प्रजातींपैकी सर्वात सामान्य प्रजाती, ज्या सुमारे ७० आहेत, त्यांचा समावेश येथे केला आहे.
}

\birdentry{घार}{\textit{Milvus migrans}}
  {घारीएवढा}
  {अतिनेहमी आढळणारा स्थानिक}
  {मोठा शिकारी पक्षी. काटेरी शेपटी आणि लांब पंख हे वैशिष्ट्य. पंख काळसर तपकिरी, तळाशी किंचित फिकट.}
  {परिसरातील सर्वत्र}
  {उंच आकाशात गरगर फिरताना दिसते. लहान सस्तन प्राणी, पक्षी आणि मृत प्राणी खाते.}
  {डिसेंबर ते एप्रिल दरम्यान प्रजनन. उंच झाडांवर काड्या-कचऱ्यापासून घरटे. २-३ पांढरी, तपकिरी ठिपक्यांची अंडी. दोन्ही पालक अंडी उबवतात.}
  {black-kite.jpg}

\birdentry{काळा शराटी}{\textit{Pseudibis papillosa}}
  {घारीएवढा}
  {नेहमी आढळणारा स्थानिक}
  {मोठा काळा पक्षी, मानेवर लाल पट्टा. लांब वाकडी चोच आणि डोक्यावरील उघडा लाल भाग ठळक.}
  {परिसरातील मोकळ्या जागा, गवताळ प्रदेश आणि पाणवठे}
  {हळूहळू चालत शिकार करतो. लहान गटात आढळतो. कीटक आणि लहान प्राणी खातो.}
  {पावसाळ्यात प्रजनन. मोठ्या झाडांवर मंचासारखे घरटे. २-४ फिकट निळसर-पांढरी अंडी.}
  {red-naped-ibis.jpg}

\birdentry{बहिरी ससाणा}{\textit{Falco peregrinus}}
  {घारीएवढा}
  {दुर्मिळ हिवाळी पाहुणा}
  {मोठा, शक्तिशाली ससाणा. वरचा भाग गडद करडा, खालून पट्टेदार. काळा 'मिशांसारखा' पट्टा विशिष्ट.}
  {कधीकधी उंच इमारती आणि मोकळ्या जागांमध्ये}
  {वेगवान आणि चपळ शिकारी. हवेत भरारी मारून पक्षी पकडतो.}
  {परिसरात प्रजनन करत नाही; फक्त हिवाळी पाहुणा.}
  {peregrine-falcon.jpg}

\birdentry{शिक्रा}{\textit{Accipiter badius}}
  {मैनेपेक्षा मोठा}
  {नेहमी आढळणारा स्थानिक}
  {छोटा ससाणा, लाल डोळे आणि पट्टेदार छाती विशिष्ट. वरून करडा, खालून पांढरा, तांबूस पट्ट्यांसह.}
  {परिसरातील वृक्षराजी आणि बागा}
  {लहान पक्षी, सस्तन प्राणी आणि कीटक खातो. खूप चपळ शिकारी.}
  {मार्च ते जुलै प्रजनन काळ. झाडांवर काड्या-पानांचे घरटे. ३-४ निळसर पांढरी अंडी.}
  {shikra.jpg}

\birdentry{पांढऱ्या छातीची पाणकोंबडी}{\textit{Amaurornis phoenicurus}}
  {बुलबुलापेक्षा मोठी}
  {नेहमी आढळणारा स्थानिक}
  {मध्यम आकाराचा पक्षी. पांढरा चेहरा, गळा, छाती; गडद तपकिरी शरीर. चोचीजवळ लाल पट्टा.}
  {परिसरातील पाणवठे आणि दलदलीचे भाग}
  {पाण्याजवळ राहते. कीटक, छोटे मासे आणि वनस्पती खाते.}
  {दाट वनस्पतीत घरटे. ४-६ क्रीमी पांढरी, तपकिरी ठिपक्यांची अंडी. दोन्ही पालक काळजी घेतात.}
  {white-breasted-waterhen.jpg}

\birdentry{ढोकरी}{\textit{Ardeola grayii}}
  {मैनेपेक्षा मोठा}
  {अतिनेहमी आढळणारा स्थानिक}
  {मध्यम आकाराचा बगळा. तपकिरी शरीर, पांढरे पंख. पिवळी चोच आणि पाय.}
  {परिसरातील पाणवठे आणि दलदलीचे भाग}
  {मासे, बेडूक आणि कीटक खातो. पाण्याजवळ बसून शिकार करतो.}
  {जून ते सप्टेंबर प्रजनन काळ. पाण्याजवळील झाडांवर घरटे. ३-५ फिकट निळी अंडी.}
  {indian-pond-heron.jpg}

\birdentry{रात्र ढोकरी}{\textit{Nycticorax nycticorax}}
  {कावळ्यापेक्षा मोठा}
  {दुर्मिळ स्थानिक}
  {मध्यम आकाराचा बगळा. काळी टोपी, पाठ; करडे पंख; पांढरा तळभाग. लाल डोळे, जाड चोच.}
  {परिसरातील पाणवठे आणि दलदलीचे भाग}
  {रात्री सक्रिय. मासे, बेडूक आणि कीटक खातो.}
  {जून ते सप्टेंबर प्रजनन काळ. पाण्याजवळील झाडांवर घरटे. ३-५ फिकट निळसर अंडी.}
  {black-crowned-night-heron.jpg}

\birdentry{गवती बगळा}{\textit{Bubulcus ibis}}
  {मैनेपेक्षा मोठा}
  {अतिनेहमी आढळणारा स्थानिक}
  {मध्यम आकाराचा पांढरा बगळा. प्रजनन काळात डोके, छाती, पाठीवर केशरी पिसारे.}
  {परिसरातील गवताळ भाग आणि मोकळ्या जागा}
  {गुरांच्या जवळपास राहतो. कीटक आणि लहान प्राणी खातो.}
  {जून ते सप्टेंबर प्रजनन काळ. वसाहतीत राहतो. ३-४ फिकट निळी अंडी.}
  {cattle-egret.jpg}

\birdentry{छोटा बगळा}{\textit{Egretta garzetta}}
  {मैनेपेक्षा मोठा}
  {नेहमी आढळणारा स्थानिक}
  {मध्यम आकाराचा शुभ्र पांढरा बगळा. काळी चोच, काळे पाय, पिवळे पंजे.}
  {परिसरातील पाणवठे आणि दलदलीचे भाग}
  {मासे, बेडूक आणि कीटक खातो. पाण्यात उभा राहून शिकार करतो.}
  {जून ते सप्टेंबर प्रजनन काळ. वसाहतीत राहतो. ३-५ फिकट निळसर अंडी.}
  {little-egret.jpg}

\birdentry{पारवा}{\textit{Columba livia}}
  {मैनेपेक्षा मोठा}
  {अतिनेहमी आढळणारा स्थानिक}
  {जाडजूड करडा पक्षी. मानेवर इंद्रधनुषी पिसारे. पंखांवर दोन काळे पट्टे, शेपटीवर काळा पट्टा.}
  {परिसरातील इमारती आणि मोकळ्या जागा}
  {कळपाने राहतो. जमिनीवरून धान्य वेचतो. शहरी जीवनाशी जुळवून घेतो.}
  {वर्षभर प्रजनन. इमारतींवर काड्यांचे घरटे. २ पांढरी अंडी. दोन्ही पालक काळजी घेतात.}
  {blue-rock-pigeon.jpg}

\birdentry{ठिपकेवाला होला}{\textit{Streptopelia chinensis}}
  {मैनेएवढा}
  {नेहमी आढळणारा स्थानिक}
  {बारीक, लांब शेपटीचा पारवा. गुलाबी-करडा रंग. मानेवर विशिष्ट काळे-पांढरे ठिपके.}
  {परिसरातील वृक्षराजी आणि बागांमध्ये}
  {नेहमी जोडीने किंवा छोट्या कळपात. जमिनीवर धान्य आणि बिया खातो.}
  {झाडे आणि झुडपांमध्ये सैल घरटे बांधतो. २ पांढरी अंडी. दोन्ही पालक १४-१६ दिवस अंडी उबवतात.}
  {spotted-dove.jpg}

\birdentry{छोटा तपकिरी होला}{\textit{Streptopelia senegalensis}}
  {मैनेपेक्षा लहान}
  {नेहमी आढळणारा स्थानिक}
  {लहान आणि नाजूक पारवा. फिकट तपकिरी रंग, मानेवर जांभळी छटा. मानेच्या बाजूंवर काळे-लाल चौकटी नमुना.}
  {परिसरातील सर्वत्र, विशेषतः मोकळ्या भागात}
  {नेहमी जोडीने दिसतो. जमिनीवर बिया वेचतो. गोड संगीतमय आवाज करतो.}
  {झुडपे आणि लहान झाडांवर साधे घरटे. २ पांढरी अंडी. दोन्ही पालक १३-१५ दिवस अंडी उबवतात.}
  {little-brown-dove.jpg}

\birdentry{टिटवी}{\textit{Vanellus indicus}}
  {मैनेपेक्षा बरीच मोठी}
  {नेहमी आढळणारा स्थानिक}
  {मोठी टिटवी. डोळ्यांसमोर लाल मांसल वाढ विशिष्ट. तपकिरी पंख, काळे डोके व छाती, पांढरा चेहरा व तळभाग.}
  {परिसरातील मोकळ्या भागात, विशेषतः पाणवठ्यांजवळ}
  {"डिड-ही-डू-इट" या धोक्याच्या आवाजासाठी प्रसिद्ध. दिवसा-रात्री सक्रिय. कीटक व लहान प्राणी खाते.}
  {जमिनीवर उथळ खड्ड्यात घरटे. खडकाळ भागात. ३-४ जैतूनी-तपकिरी अंडी. दोन्ही पालक २८-३० दिवस उबवतात.}
  {red-wattled-lapwing.jpg}

\birdentry{कोकीळ}{\textit{Eudynamys scolopacea}}
  {कावळ्यापेक्षा लहान}
  {नेहमी आढळणारा स्थानिक}
  {नर चमकदार काळा, लाल डोळे. मादी तपकिरी, पांढरे ठिपके, खालून पट्टेदार. लिंगभेद स्पष्ट.}
  {परिसरातील सर्वत्र, विशेषतः वृक्षराजीत}
  {विशिष्ट "कू-ऊ" आवाजासाठी प्रसिद्ध. कावळ्यांच्या घरट्यात परजीवी. फळे व बेरी खातो.}
  {कावळे व इतर पक्ष्यांच्या घरट्यात परजीवी म्हणून अंडी घालतो.}
  {asian-koel.jpg}

\birdentry{कारुण्य कोकीळ}{\textit{Cacomantis passerinus}}
  {मैनेपेक्षा लहान}
  {उन्हाळी पाहुणा}
  {करडा पक्षी, छातीवर गडद करडा रंग, तळाशी फिकट. पिवळा डोळ्याभोवतीचा कडा विशिष्ट.}
  {परिसरातील वृक्षराजीत}
  {लपून-छपून राहतो. हळुवार शिट्टीसारखा आवाज करतो. मुख्यतः कीटक खातो.}
  {फुलपाखरे व शिंपी पक्ष्यांच्या घरट्यात परजीवी म्हणून अंडी घालतो.}
  {grey-bellied-cuckoo.jpg}

\birdentry{पावश्या}{\textit{Hierococcyx varius}}
  {मैनेपेक्षा बराच मोठा}
  {नेहमी आढळणारा स्थानिक}
  {शिक्र्यासारखा दिसणारा. खालून पट्टेदार. पिवळा डोळ्याभोवतीचा कडा. उडताना शिक्र्यासारखा दिसतो.}
  {परिसरातील वृक्षराजीत सर्वत्र}
  {विशिष्ट "ब्रेन-फीवर" आवाजासाठी प्रसिद्ध. कीटक व अळ्या खातो.}
  {सातभाई व इतर पक्ष्यांच्या घरट्यात परजीवी. मार्च ते जुलै दरम्यान सक्रिय.}
  {common-hawk-cuckoo.jpg}

\birdentry{भारद्वाज}{\textit{Centropus sinensis}}
  {कावळ्याएवढा}
  {नेहमी आढळणारा स्थानिक}
  {मोठा काळा पक्षी. केशरी पंख व लांब शेपटी. लाल डोळे व वाकडी काळी चोच विशिष्ट.}
  {परिसरातील दाट वनस्पतीत}
  {लपून-छपून वावरतो. जमिनीवर चालताना दिसतो. खोल गुरगुरण्याचा आवाज. कीटक व लहान प्राणी खातो.}
  {दाट वनस्पतीत जमिनीजवळ छतासारखे घरटे. ३-५ पांढरी अंडी. दोन्ही पालक १५-१६ दिवस उबवतात.}
  {greater-coucal.jpg}

\birdentry{राजपोपट}{\textit{Psittacula krameri}}
  {मैनेपेक्षा बराच मोठा}
  {अतिनेहमी आढळणारा स्थानिक}
  {हिरवा पोपट, लांब शेपटी. नराच्या मानेभोवती गुलाबी व काळा पट्टा, मादीला नसतो.}
  {परिसरातील सर्वत्र, विशेषतः वृक्षराजीत}
  {आवाजी, कळपवासी. मोठ्या कळपात दिसतो. फळे, बिया व धान्य खातो.}
  {डिसेंबर ते मे प्रजनन. झाडांच्या पोकळीत घरटे. ३-४ पांढरी अंडी. दोन्ही पालक २२-२४ दिवस उबवतात.}
  {rose-ringed-parakeet.jpg}

\birdentry{करण पोपट}{\textit{Psittacula eupatria}}
  {कावळ्याएवढा}
  {दुर्मिळ स्थानिक}
  {राजपोपटापेक्षा मोठा. जाड लाल चोच. नराला गुलाबी व काळा गळपट्टा.}
  {परिसरातील वृक्षराजीत, इतर पोपटांपेक्षा कमी}
  {राजपोपटासारखीच सवय पण जास्त सावध. फळे व बिया खातो.}
  {मोठ्या जुन्या झाडांच्या पोकळीत घरटे. २-४ पांढरी अंडी. दोन्ही पालक २३-२६ दिवस उबवतात.}
  {alexandrine-parakeet.jpg}

\birdentry{टोई पोपट}{\textit{Psittacula cyanocephala}}
  {मैनेएवढा}
  {नेहमी आढळणारा स्थानिक}
  {नराचे डोके जांभळे-लाल, मादीचे निळसर-करडे. दोघांचेही शरीर हिरवे व पिवळट शेपटीचे टोक.}
  {परिसरातील वृक्षराजीत नियमित}
  {नेहमी जोडीने किंवा लहान कळपात. इतर पोपटांपेक्षा जास्त झाडावर राहतो. फळे व बिया खातो.}
  {झाडांच्या पोकळीत घरटे. काळ स्थानपरत्वे बदलतो. ३-४ पांढरी अंडी. दोन्ही पालक २२-२४ दिवस उबवतात.}
  {plum-headed-parakeet.jpg}

\birdentry{घर पाकोळी}{\textit{Apus affinis}}
  {मैनेपेक्षा लहान}
  {अतिनेहमी आढळणारा स्थानिक}
  {लहान घारांडी, पांढरी कंबर व काटेरी शेपटी. गडद तपकिरी-काळा रंग.}
  {परिसरातील इमारतींच्या परिसरात}
  {हवेत तरंगत राहणारा. मोठ्या कळपात इमारतींभोवती घिरट्या घालताना दिसतो.}
  {वसाहतीत राहणारा. इमारतींच्या छपराखाली मातीचे घरटे. २-३ पांढरी अंडी. दोन्ही पालक १९-२१ दिवस उबवतात.}
  {house-swift.jpg}

\birdentry{ताड पाकोळी}{\textit{Cypsiurus balasiensis}}
  {चिमणीपेक्षा लहान}
  {नेहमी आढळणारा स्थानिक}
  {घरच्या घारांडीपेक्षा लहान, बारीक शरीर व खोलगट काटेरी शेपटी. एकूण धुरकट तपकिरी.}
  {परिसरातील ताडाच्या झाडांभोवती}
  {अतिशय वेगवान उड्डाणक्षम, क्वचितच बसतो. ताडाभोवती दिसतो. हवेतील कीटक खातो.}
  {वर्षभर प्रजनन. ताडाच्या पानाखाली लाळेने घरटे चिकटवतो. २-३ पांढरी अंडी घालतो.}
  {asian-palm-swift.jpg}

\birdentry{पांढऱ्या छातीचा धीवर}{\textit{Halcyon smyrnensis}}
  {मैनेपेक्षा मोठा}
  {नेहमी आढळणारा स्थानिक}
  {मोठा खंड्या, चमकदार निळी पाठ, चॉकलेटी डोके व पांढरी छाती. मोठी लाल चोच ठळक.}
  {परिसरातील पाणवठे व बागा}
  {ठळकपणे बसून राहतो, शिकारीसाठी झडप घालतो. मासे, बेडूक, सरडे व मोठे कीटक खातो.}
  {मार्च ते जून प्रजनन. जमिनीच्या कडेत भुयार खणून घरटे. ४-७ पांढरी अंडी. दोन्ही पालक २०-२२ दिवस उबवतात.}
  {white-breasted-kingfisher.jpg}

\birdentry{सामान्य धीवर}{\textit{Alcedo atthis}}
  {बुलबुलापेक्षा लहान}
  {नेहमी आढळणारा स्थानिक}
  {लहान खंड्या, चमकदार निळी वरची बाजू व केशरी खालची बाजू. लांब चोच व छोटी शेपटी.}
  {परिसरातील पाणवठे व तलावांजवळ}
  {पाण्यावर खाली बसून मासे पकडतो. शिकारीपूर्वी मान हलवतो.}
  {फेब्रुवारी ते एप्रिल प्रजनन. जमिनीच्या कडेत भुयार खणून घरटे. ५-७ पांढरी अंडी घालतो.}
  {common-kingfisher.jpg}

\birdentry{वेडा राघू}{\textit{Merops orientalis}}
  {चिमणीपेक्षा मोठा}
  {नेहमी आढळणारा स्थानिक}
  {बारीक हिरवा पक्षी, निळा गळा, काळी डोळ्यांची पट्टी. लांब मधली शेपटीची पिसे ठळक.}
  {परिसरातील विरळ झाडांच्या मोकळ्या भागात}
  {सुंदर उड्डाणपटू, हवेत कीटक पकडतो. तारा व उघड्या फांद्यांवर बसतो.}
  {मार्च ते जून प्रजनन. जमिनीत भुयार खणून घरटे. ४-७ पांढरी अंडी घालतो.}
  {asian-green-bee-eater.jpg}

\birdentry{तांबट}{\textit{Megalaima haemacephala}}
  {बुलबुलापेक्षा मोठा}
  {नेहमी आढळणारा स्थानिक}
  {लहान, हिरवा पक्षी, किरमिजी कपाळ व गळा. पिवळा डोळ्याभोवतीचा भाग ठळक.}
  {परिसरातील वृक्षराजीत सर्वत्र}
  {धातूसारख्या 'टुक-टुक-टुक' आवाजासाठी प्रसिद्ध. मुख्यतः फळे, विशेषतः उंबर खातो.}
  {मेलेल्या झाडांच्या खोडात किंवा फांदीत घरटे खणतो. २-४ पांढरी अंडी. दोन्ही पालक १३-१५ दिवस उबवतात.}
  {coppersmith-barbet.jpg}

\birdentry{धूसर कडा पंकोळी}{\textit{Hirundo concolor}}
  {चिमणीपेक्षा लहान}
  {नेहमी आढळणारा स्थानिक}
  {लहान, गडद तपकिरी घारूळी, किंचित काटेरी शेपटी. एकसमान धुरकट रंग.}
  {परिसरातील इमारती व खडकाळ भागात}
  {हवेत कीटक पकडणारी. इमारतींजवळ उडताना दिसते.}
  {इमारतींवर व खडकांवर मातीचे वाटीसारखे घरटे. २-३ पांढरी अंडी. दोन्ही पालक १४-१६ दिवस उबवतात.}
  {dusky-crag-martin.jpg}

\birdentry{माळ भिंगरी}{\textit{Hirundo rustica}}
  {चिमणीएवढी}
  {सामान्य हिवाळी पाहुणा}
  {लांब, खोल काटेरी शेपटी. वर निळसर-काळी, कपाळ व गळा तांबूस, खालून फिकी.}
  {हिवाळ्यात परिसरातील मोकळ्या भागात}
  {सुंदर उड्डाणपटू, हवेत कीटक पकडते. तारांवर बसते.}
  {परिसरात प्रजनन करत नाही; फक्त हिवाळी पाहुणी.}
  {barn-swallow.jpg}

\birdentry{लाल पुठ्ठ्याची भिंगरी}{\textit{Hirundo daurica}}
  {चिमणीएवढी}
  {नेहमी आढळणारा स्थानिक}
  {होलोखेल घारूळीसारखी पण तांबूस कंबर व छोट्या शेपटीची पिसे.}
  {परिसरातील सर्वत्र, विशेषतः इमारतींजवळ}
  {हवेत कीटक पकडणारी, इतर घारूळींच्या कळपात मिसळते.}
  {इमारतींखाली बाटलीसारखे मातीचे घरटे. ३-४ पांढरी अंडी. दोन्ही पालक १४-१६ दिवस उबवतात.}
  {red-rumped-swallow.jpg}

\birdentry{तारबाली भिंगरी}{\textit{Hirundo smithii}}
  {चिमणीएवढी}
  {नेहमी आढळणारा स्थानिक}
  {ठळक तारेसारखी लांब शेपटीची पिसे. चमकदार निळसर-काळी वर, पांढरी खाली.}
  {परिसरातील पाणवठे व मोकळ्या जागा}
  {सुंदर उड्डाणपटू, तारांवर बसते. हवेतील कीटक खाते.}
  {पुलांखाली व इमारतींवर मातीचे वाटीसारखे घरटे. २-३ पांढरी अंडी. दोन्ही पालक १४-१६ दिवस उबवतात.}
  {wire-tailed-swallow.jpg}

\birdentry{लांब शेपटीचा खाटीक}{\textit{Lanius schach}}
  {बुलबुलापेक्षा मोठा}
  {नेहमी आढळणारा स्थानिक}
  {करडा-तपकिरी वरचा भाग, खालून पांढरा. डोळ्यातून काळी पट्टी. लांब टोकदार शेपटी.}
  {परिसरातील विरळ झाडी असलेले मोकळे भाग}
  {ठळक जागी बसून शिकार करतो. शिकार काट्यांवर टोचून ठेवतो. कीटक, सरडे व लहान पक्षी खातो.}
  {काटेरी झुडपात नीटनेटके घरटे बांधतो. ३-६ पांढरी, तपकिरी ठिपक्यांची अंडी. दोन्ही पालक १५-१६ दिवस उबवतात.}
  {long-tailed-shrike.jpg}

\birdentry{हळद्या}{\textit{Oriolus oriolus}}
  {मैनेपेक्षा मोठा}
  {नेहमी आढळणारा स्थानिक}
  {नर सोनेरी पिवळा, काळे पंख व शेपटी. मादी फिकट हिरवी.}
  {परिसरातील वृक्षराजीत}
  {लाजाळू स्वभाव, वृक्षमुकुटात राहतो. गोड बासरीसारखा आवाज. फळे व कीटक खातो.}
  {झाडाच्या फांदीवर झोळीसारखे घरटे. २-४ पांढरी, तांबूस ठिपक्यांची अंडी. दोन्ही पालक १४-१६ दिवस उबवतात.}
  {golden-oriole.jpg}

\birdentry{कोतवाल}{\textit{Dicrurus macrocercus}}
  {मैनेपेक्षा मोठा}
  {नेहमी आढळणारा स्थानिक}
  {चमकदार काळा पक्षी, काटेरी शेपटी. जाड वाकडी चोच.}
  {परिसरातील मोकळ्या झाडांच्या भागात}
  {धाडसी व आक्रमक. इतर पक्ष्यांचा पाठलाग करतो. हवेत कीटक पकडण्यात तरबेज.}
  {मार्च ते जुलै प्रजनन. बाह्य फांदीवर वाटीसारखे घरटे. ३-५ गुलाबी-पांढरी, तांबूस ठिपक्यांची अंडी.}
  {black-drongo.jpg}

\birdentry{राखी कोतवाल}{\textit{Dicrurus leucophaeus}}
  {मैनेपेक्षा मोठा}
  {हिवाळी पाहुणा}
  {कृष्ण भारद्वाजासारखा पण एकूण करडा रंग. कमी काटेरी शेपटी.}
  {परिसरातील वृक्षराजीत हिवाळ्यात}
  {कृष्ण भारद्वाजापेक्षा कमी आक्रमक. हवेत व पानांमधून कीटक पकडतो.}
  {परिसरात प्रजनन करत नाही; फक्त हिवाळी पाहुणा.}
  {ashy-drongo.jpg}

\birdentry{साळुंकी}{\textit{Acridotheres tristis}}
  {मैना}
  {नेहमी आढळणारा स्थानिक}
  {तपकिरी शरीर, काळे डोके, पिवळी चोच व पाय. उडताना पंखांवर पांढरे पट्टे दिसतात.}
  {परिसरातील सर्वत्र विपुल}
  {धीट, जुळवून घेणारा पक्षी. जोडीने किंवा कळपात दिसतो. सर्व प्रकारचे अन्न खातो.}
  {झाडांच्या व इमारतींच्या पोकळीत घरटे. ४-६ निळसर-हिरवी अंडी. दोन्ही पालक १३-१५ दिवस उबवतात.}
  {common-myna.jpg}

\birdentry{जंगली मैना}{\textit{Acridotheres fuscus}}
  {मैना}
  {नेहमी आढळणारा स्थानिक}
  {सामान्य मैनेसारखी पण अधिक करडी. चोचीजवळ पिसांची झुपके. नारिंगी-पिवळी चोच व पाय.}
  {परिसरातील वृक्षराजी व बागा, शहरी भागात कमी}
  {नेहमी जोडीने किंवा छोट्या कळपात. सामान्य मैनेपेक्षा जास्त झाडावर राहते. कीटक व फळे खाते.}
  {मार्च ते जुलै प्रजनन. झाडांच्या पोकळीत घरटे. ४-५ फिकट निळी अंडी.}
  {jungle-myna.jpg}

\birdentry{भांगपाडी मैना}{\textit{Sturnia pagodarum}}
  {मैना}
  {नेहमी आढळणारा स्थानिक}
  {विशिष्ट बदामी रंगाची मैना, काळी टोपी, करडे पंख व पांढरा तळभाग.}
  {परिसरातील वृक्षराजी व बागांमध्ये}
  {जोडीने किंवा छोट्या कळपात दिसते. फळे, बेरी व कीटक खाते.}
  {एप्रिल ते जुलै प्रजनन. झाडांच्या व इमारतींच्या पोकळीत घरटे. ३-५ फिकट निळी अंडी.}
  {brahminy-starling.jpg}

\birdentry{दयाळ}{\textit{Copsychus saularis}}
  {बुलबुल}
  {नेहमी आढळणारा स्थानिक}
  {नर काळा-पांढरा ठळक रंगसंगती, मादी अधिक करडी. लांब शेपटी वर धरतो.}
  {परिसरातील वृक्षराजी व बागांमध्ये सर्वत्र}
  {धीट व गोड आवाजी गायक. कीटक व बेरी खातो.}
  {मार्च ते जुलै प्रजनन. झाडांच्या व इमारतींच्या पोकळीत घरटे. ४-५ हिरवट-पांढरी, तपकिरी ठिपक्यांची अंडी.}
  {oriental-magpie-robin.jpg}

\birdentry{पांढऱ्या कंठाची नाचण}{\textit{Rhipidura albogularis}}
  {चिमणीपेक्षा लहान}
  {नेहमी आढळणारा स्थानिक}
  {लहान काळा-पांढरा पक्षी, ठिपकेदार छाती, पांढरी भुवई व पंख्यासारखी शेपटी पसरून ठेवतो.}
  {परिसरातील दाट झाडी असलेले भाग}
  {चपळ व कसरती, कीटक पकडताना शेपटी पसरतो. हवेत छोटी उड्डाणे करून कीटक पकडतो.}
  {मुख्यतः मार्च ते सप्टेंबर प्रजनन. झाडाच्या फांदीवर नीटनेटके वाटीसारखे घरटे. २-३ क्रीम रंगाची, तांबूस ठिपक्यांची अंडी.}
  {spot-breasted-fantail.jpg}

\birdentry{कवड्या गप्पीदास}{\textit{Saxicola caprata}}
  {बुलबुलापेक्षा लहान}
  {नेहमी आढळणारा स्थानिक}
  {नर काळा, पांढरी कंबर व पंखांवर पट्टे; मादी तपकिरी, खालून फिकी.}
  {परिसरातील विरळ झुडपे असलेले मोकळे भाग}
  {झुडपे व कुंपणांवर ठळकपणे बसतो. छोटी उड्डाणे करून कीटक पकडतो.}
  {मार्च ते ऑगस्ट प्रजनन. गवतात व झुडपात वाटीसारखे घरटे. ३-४ फिकट निळसर-हिरवी, तांबूस ठिपक्यांची अंडी.}
  {pied-bushchat.jpg}

\birdentry{लाल बुड्या बुलबुल}{\textit{Pycnonotus cafer}}
  {बुलबुल}
  {नेहमी आढळणारा स्थानिक}
  {गडद तपकिरी-काळा पक्षी, खवल्यांसारखी रचना, लाल मूळशेपटी व काळी शिखा विशिष्ट.}
  {परिसरातील सर्वत्र}
  {सक्रिय व आवाजी. जोडीने किंवा लहान कळपात. फळे व कीटक खातो.}
  {वर्षभर प्रजनन, फेब्रुवारी ते मे शिखर. झुडपात वाटीसारखे घरटे. २-३ गुलाबी-पांढरी, तपकिरी ठिपक्यांची अंडी. दोन्ही पालक १४ दिवस उबवतात.}
  {red-vented-bulbul.jpg}

\birdentry{शिपाई बुलबुल}{\textit{Pycnonotus jocosus}}
  {बुलबुल}
  {नेहमी आढळणारा स्थानिक}
  {तपकिरी वरचा भाग, पांढरा खालचा भाग. लाल कानपट्टी व टोकदार काळी शिखा विशिष्ट.}
  {परिसरातील वृक्षराजी व बागांमध्ये}
  {जीवंत पक्षी, गोड आवाजी. नेहमी जोडीने. फळे व कीटक खातो.}
  {झुडपात व लहान झाडांवर नीटनेटके वाटीसारखे घरटे. २-३ गुलाबी-पांढरी, तपकिरी ठिपक्यांची अंडी. दोन्ही पालक १२-१४ दिवस उबवतात.}
  {red-whiskered-bulbul.jpg}

\birdentry{राखी सातभाई}{\textit{Turdoides malcolmi}}
  {मैनेपेक्षा मोठा}
  {नेहमी आढळणारा स्थानिक}
  {करडा-तपकिरी पक्षी, फिकट रेघा. लांब टप्प्याटप्प्यांची शेपटी व वाकडी चोच.}
  {परिसरातील मोकळी झुडपे व बागा}
  {गटाने राहतो, आवाजी. जमिनीवर पाने उलटून कीटक व बेरी खातो.}
  {मार्च ते सप्टेंबर प्रजनन. काटेरी झुडपात अस्ताव्यस्त घरटे. ३-४ निळसर अंडी. गटातील सदस्य मदत करतात.}
  {large-grey-babbler.jpg}

\birdentry{टिकेलची निळी माशीमार}{\textit{Muscicapa tickelliae}}
  {बुलबुलापेक्षा लहान}
  {नेहमी आढळणारा स्थानिक}
  {नर चमकदार निळा वरून, तांबूस खालून. मादी फिकट.}
  {परिसरातील दाट झाडीच्या भागात}
  {सक्रिय पक्षी, बसल्या जागेवरून छोटी उड्डाणे करून कीटक पकडतो. गोड शिट्टीसारखा आवाज.}
  {एप्रिल ते जुलै प्रजनन. झाडाच्या पोकळीत किंवा फटीत नीटनेटके वाटीसारखे घरटे. ३-४ फिकट अंडी, तपकिरी ठिपके.}
  {tickells-blue-flycatcher.jpg}

\birdentry{राखी वटवट्या}{\textit{Prinia socialis}}
  {चिमणीपेक्षा लहान}
  {नेहमी आढळणारा स्थानिक}
  {राखाडी वरचा भाग, पांढरा खालचा भाग. लांब टप्प्याटप्प्यांची शेपटी वर करून धरतो.}
  {परिसरातील बागा व झुडपांच्या भागात}
  {सक्रिय पक्षी, वनस्पतीतून कर्कश आवाज करत फिरतो.}
  {मुख्यतः जून ते सप्टेंबर प्रजनन. गवतात व झुडपात खोल वाटीसारखे घरटे. ३-५ विटकरी अंडी.}
  {ashy-prinia.jpg}

\birdentry{रान वटवट्या}{\textit{Prinia sylvatica}}
  {चिमणीपेक्षा लहान}
  {नेहमी आढळणारा स्थानिक}
  {करडा-तपकिरी वरचा भाग, फिकट खालचा भाग. राखी वटवट्यापेक्षा गडद.}
  {परिसरातील गवताळ व झुडपाळ भाग}
  {लपून-छपून राहतो, मोठ्या आवाजाने ओळख करून देतो.}
  {जून ते सप्टेंबर प्रजनन. गवताच्या झुंबरात चेंडूसारखे घरटे. ३-५ हिरवट-पांढरी अंडी.}
  {jungle-prinia.jpg}

\birdentry{शिंपी}{\textit{Orthotomus sutorius}}
  {चिमणीपेक्षा बराच लहान}
  {नेहमी आढळणारा स्थानिक}
  {छोटा सुतार पक्षी, लांब उभी शेपटी. जैतुनी-हिरवा वरचा भाग, पांढरा खालचा भाग, तांबूस टाळू.}
  {परिसरातील बागा व दाट झुडपांच्या भागात}
  {सक्रिय पक्षी, सतत वनस्पतीतून फिरत राहतो. मोठा 'टू-विट टू-विट' आवाज.}
  {मुख्यतः मे ते सप्टेंबर प्रजनन. जिवंत पानांना शिवून अद्भुत घरटे बनवतो. ३-५ तांबूस-पांढरी, ठिपक्यांची अंडी.}
  {common-tailorbird.jpg}

\birdentry{छोटा शुभ्रकंठी वटवट्या}{\textit{Sylvia curruca}}
  {चिमणीपेक्षा लहान}
  {सामान्य हिवाळी पाहुणा}
  {छोटा सुतार पक्षी, करडा-तपकिरी वरचा भाग, पांढरा गळा व खालचा भाग. डोळ्यातून काळी पट्टी.}
  {परिसरातील झुडपे व झाडीत}
  {सक्रिय पण लपून राहणारा. झुडपात कीटक शोधतो. खडखडणारा विशिष्ट आवाज.}
  {परिसरात प्रजनन करत नाही; फक्त हिवाळी पाहुणा.}
  {lesser-whitethroat.jpg}

\birdentry{चिमणी}{\textit{Passer domesticus}}
  {चिमणी}
  {नेहमी आढळणारा स्थानिक}
  {नर करडा-तपकिरी, काळा गळपट्टा व तांबूस मान; मादी एकसमान तपकिरी.}
  {परिसरातील इमारती व मानवी वस्तीजवळ}
  {कळपवासी, मुख्यतः धान्य व बिया खाते.}
  {जवळपास वर्षभर प्रजनन. इमारतींच्या पोकळीत घरटे. ४-६ पांढरी, तपकिरी ठिपक्यांची अंडी.}
  {house-sparrow.jpg}

\birdentry{सुगरण}{\textit{Ploceus philippinus}}
  {चिमणी}
  {नेहमी आढळणारा स्थानिक}
  {प्रजननकाळात नर चमकदार पिवळा, तपकिरी पाठ; मादी चिमणीसारखी.}
  {परिसरातील गवताळ भाग व बोरूची झाडी}
  {वसाहतीत राहतो. नर कलात्मक लोंबती घरटी बांधतो. गवताचे बी खातो.}
  {पावसाळ्यात प्रजनन. नर अनेक टोपीसारखी घरटी बांधतो. २-४ पांढरी अंडी.}
  {baya-weaver.jpg}

\birdentry{चरक}{\textit{Copsychus fulicatus}}
  {चिमणी}
  {नेहमी आढळणारा स्थानिक}
  {नर चमकदार काळा, पंखांवर पांढरे पट्टे. मादी तपकिरी. शेपटी नेहमी वर धरून असतो.}
  {परिसरातील मोकळे भाग व झुडपाळ प्रदेश}
  {जमिनीवर सक्रिय, वारंवार शेपटी वर करतो. गोड आवाज व शिट्ट्या वाजवतो.}
  {मार्च ते सप्टेंबर प्रजनन. भिंतीच्या भेगा व जमिनीत घरटे. २-४ पांढरी, तपकिरी ठिपक्यांची अंडी.}
  {indian-robin.jpg}

\birdentry{चष्मेवाला}{\textit{Zosterops palpebrosus}}
  {चिमणीपेक्षा लहान}
  {नेहमी आढळणारा स्थानिक}
  {छोटा जैतुनी-हिरवा पक्षी, डोळ्याभोवती पांढरा कडा विशिष्ट. पिवळा गळा व शेपटीखालचा भाग.}
  {परिसरातील वृक्षराजी व बागांमध्ये}
  {चपळ व कसरती. छोट्या कळपात पानांमध्ये फिरतो. कीटक व मधुरस खातो.}
  {मार्च ते सप्टेंबर प्रजनन. झाडाच्या फांदीवर नीटनेटके वाटीसारखे घरटे. २-४ फिकट निळी अंडी.}
  {oriental-white-eye.jpg}

\birdentry{ठिपकेवाली मनोली}{\textit{Lonchura punctulata}}
  {चिमणीपेक्षा लहान}
  {नेहमी आढळणारा स्थानिक}
  {लहान तपकिरी फिंच, छातीवर खवल्यांसारखी रचना. जाड छोटी चोच.}
  {परिसरातील गवताळ व झुडपाळ भाग}
  {कळपाने राहते, गवताचे बी खाते. गोड चिवचिव आवाज.}
  {जुलै ते ऑक्टोबर प्रजनन. काटेरी झुडपात मोठे चेंडूसारखे घरटे. ४-६ पांढरी अंडी.}
  {scaly-breasted-munia.jpg}

\birdentry{टकाचोर}{\textit{Dendrocitta vagabunda}}
  {मैनेपेक्षा बराच मोठा}
  {नेहमी आढळणारा स्थानिक}
  {मोठा तांबूस व काळा पक्षी, लांब टप्प्याटप्प्यांची शेपटी. करडी मान व पांढऱ्या टोकांची शेपटी विशिष्ट.}
  {परिसरातील वृक्षराजीत}
  {धीट व कुतूहली. लहान गटात दिसतो. फळे, कीटक व लहान प्राणी खातो.}
  {मार्च ते जुलै प्रजनन. झाडाच्या फांदीवर नीटनेटके वाटीसारखे घरटे. ३-५ हिरवट-पांढरी, तपकिरी ठिपक्यांची अंडी.}
  {rufous-treepie.jpg}

\birdentry{कावळा}{\textit{Corvus splendens}}
  {कावळा}
  {नेहमी आढळणारा स्थानिक}
  {करडा व काळा कावळा, विशिष्ट करड्या मानेचा पट्टा.}
  {परिसरातील सर्वत्र, विशेषतः मानवी वस्तीजवळ}
  {धीट व बुद्धिमान. सर्व प्रकारचे अन्न खातो.}
  {मार्च ते जुलै प्रजनन. झाडांवर मंचासारखे घरटे. ४-५ फिकट निळसर-हिरवी, तपकिरी ठिपक्यांची अंडी.}
  {house-crow.jpg}

\birdentry{डोमकावळा}{\textit{Corvus macrorhynchos}}
  {कावळ्यापेक्षा मोठा}
  {नेहमी आढळणारा स्थानिक}
  {शहरी कावळ्यापेक्षा मोठा, संपूर्ण काळा, जाड चोच. एकांडा स्वभाव.}
  {परिसरातील वृक्षराजीत व आजूबाजूच्या भागात}
  {शहरी कावळ्यापेक्षा सावध. एकटा किंवा जोडीने दिसतो. सर्व प्रकारचे अन्न खातो.}
  {मार्च ते जून प्रजनन. उंच झाडांवर मोठे मंचासारखे घरटे. ३-५ फिकट हिरवी, तपकिरी ठिपक्यांची अंडी.}
  {large-billed-crow.jpg}

\birdentry{पिंगळा}{\textit{Athene brama}}
  {मैनेपेक्षा मोठा}
  {नेहमी आढळणारा स्थानिक}
  {छोटा, जाडजूड घुबड, पांढरे ठिपके. मोठे पिवळे डोळे व पांढऱ्या भुवया त्याला गंभीर चेहरा देतात.}
  {परिसरातील वृक्षराजी व बागांमध्ये}
  {पहाटे व संध्याकाळी सक्रिय. जागरूक असताना मान हलवतो. कीटक व लहान प्राणी खातो.}
  {फेब्रुवारी ते एप्रिल प्रजनन. झाडांच्या पोकळीत व जुन्या इमारतींमध्ये घरटे. ३-५ पांढरी अंडी. नर अन्न आणतो, मादी २८-३० दिवस अंडी उबवते.}
  {spotted-owlet.jpg}

\birdentry{ठिपकेवाले वन घुबड}{\textit{Strix ocellata}}
  {कावळ्यापेक्षा मोठा}
  {दुर्मिळ स्थानिक}
  {मोठा घुबड, विशिष्ट ठिपकेदार तपकिरी पिसारा. गोल डोके, कान-शिखा नाहीत. काळे डोळे, पिवळट-हिरवी चोच.}
  {परिसरातील दाट वृक्षराजीच्या भागात}
  {पूर्णपणे रात्रीचर. रात्री खोल आवाज करतो. उंदीर, पक्षी व मोठे कीटक खातो.}
  {डिसेंबर ते मार्च प्रजनन. झाडांच्या नैसर्गिक पोकळीत घरटे. २-३ पांढरी अंडी. नर अन्न आणतो, मादी अंडी उबवते.}
  {mottled-wood-owl.jpg}

\birdentry{भारतीय राखी धनेश}{\textit{Ocyceros birostris}}
  {कावळा}
  {नेहमी आढळणारा स्थानिक}
  {मोठा करडा पक्षी, लांब शेपटी व टोपीसारखी वाढ चोचीवर. मादीची टोपी लहान.}
  {परिसरातील मोठी जुनी झाडे असलेल्या भागात}
  {नेहमी जोडीने झाडावरून झाडावर जातो. मुख्यतः फळे, विशेषतः उंबर खातो.}
  {मार्च ते जून प्रजनन. मादी स्वतःला झाडाच्या पोकळीत कोंडून घेते, एक छोटी फट सोडते. नर तिला व पिलांना खाऊ घालतो. २-४ पांढरी अंडी.}
  {indian-grey-hornbill.jpg}

\birdentry{भारतीय नीलपंख}{\textit{Coracias benghalensis}}
  {मैनेपेक्षा बराच मोठा}
  {नेहमी आढळणारा स्थानिक}
  {कावळ्याएवढा पक्षी, चमकदार निळे पंख व शेपटी. तपकिरी डोके व पाठ.}
  {परिसरातील विरळ झाडी असलेल्या मोकळ्या भागात}
  {ठळक जागी बसून राहतो, हवेत नाट्यमय लोंबकळत झेप घेतो. मोठे कीटक व लहान सरडे खातो.}
  {मार्च ते जुलै प्रजनन. झाडांच्या पोकळीत घरटे. ३-५ पांढरी अंडी. दोन्ही पालक १७-१९ दिवस उबवतात.}
  {indian-roller.jpg}

\birdentry{काळ्या डोक्याचा कोकीळ-खाटिक}{\textit{Coracina melanoptera}}
  {मैनेपेक्षा मोठा}
  {नेहमी आढळणारा स्थानिक}
  {नर करडा, काळे डोके; मादी करडी, फिकट आडव्या रेघांसह.}
  {परिसरातील वृक्षराजी व बागांमध्ये}
  {पद्धतशीरपणे पानांमध्ये कीटक शोधतो. इतर शिकारी पक्ष्यांच्या कळपात मिसळतो.}
  {एप्रिल ते जुलै प्रजनन. फांदीवर छोटे वाटीसारखे घरटे. २-३ करडी-हिरवट, तपकिरी ठिपक्यांची अंडी.}
  {black-headed-cuckoo-shrike.jpg}

\birdentry{छोटा गोमेट}{\textit{Pericrocotus cinnamomeus}}
  {चिमणी}
  {नेहमी आढळणारा स्थानिक}
  {नर काळा व नारिंगी; मादी करडी व पिवळी. दोघांनाही लांब शेपटी.}
  {परिसरातील वृक्षराजीत सर्वत्र}
  {सक्रिय, छोट्या कळपात वृक्षमुकुटात फिरतो. गोड शिट्टीसारखा आवाज.}
  {मार्च ते जून प्रजनन. फांदीवर छोटे वाटीसारखे घरटे. २-४ फिकट हिरवी, तपकिरी ठिपक्यांची अंडी.}
  {small-minivet.jpg}

\birdentry{टिकेलचा फुलटोचा}{\textit{Dicaeum erythrorhynchos}}
  {चिमणीपेक्षा बराच लहान}
  {नेहमी आढळणारा स्थानिक}
  {अतिलहान पक्षी, करडा-तपकिरी वर, मळकट पांढरा खाली. छोटी शेपटी व वाकडी चोच.}
  {परिसरातील फुले व फळे असलेल्या झाडांवर}
  {अत्यंत सक्रिय, फुलांमध्ये झपाझप फिरतो. बेरी व मधुरस खातो.}
  {वर्षभर प्रजनन, मार्च ते मे शिखर. नाशपातीसारखे लटकते घरटे. २-३ पांढरी अंडी. दोन्ही पालक पिलांची काळजी घेतात.}
  {tickells-flowerpecker.jpg}

\birdentry{करडा धोबी}{\textit{Motacilla cinerea}}
  {चिमणी}
  {सामान्य हिवाळी पाहुणा}
  {बारीक पक्षी, लांब शेपटी, करडा वरचा भाग व पिवळा खालचा भाग. सतत शेपटी हलवत राहतो.}
  {परिसरातील पाणवठे व ओलसर भागात}
  {सक्रिय शिकारी, धावत-चालत शेपटी हलवत राहतो. जमिनीवरून व हवेतून कीटक पकडतो.}
  {परिसरात प्रजनन करत नाही; फक्त हिवाळी पाहुणा.}
  {grey-wagtail.jpg}

\birdentry{पांढरा धोबी}{\textit{Motacilla alba}}
  {चिमणी}
  {सामान्य हिवाळी पाहुणा}
  {काळा-पांढरा पक्षी, करडी पाठ, पांढरा चेहरा व खालचा भाग. सतत शेपटी हलवतो.}
  {परिसरातील मोकळ्या जागा, गवताळ भाग व पाणवठ्यांजवळ}
  {सुंदर चालत जमिनीवर फिरतो, शेपटी हलवत राहतो. कीटक व लहान जीव खातो.}
  {परिसरात प्रजनन करत नाही; फक्त हिवाळी पाहुणा.}
  {white-wagtail.jpg}

\birdentry{पांढऱ्या भुबईचा धोबी}{\textit{Motacilla maderaspatensis}}
  {चिमणीपेक्षा मोठा}
  {नेहमी आढळणारा स्थानिक}
  {मोठा काळा-पांढरा हलवा, ठळक पांढरी भुवई. लांब शेपटी सतत वर-खाली हलवतो.}
  {परिसरातील पाणवठे व मोकळ्या जागा}
  {सक्रिय शिकारी, धावत-चालत शेपटी हलवत राहतो. नेहमी जोडीने दिसतो.}
  {मार्च ते सप्टेंबर प्रजनन. भिंतींच्या व कड्यांच्या भेगांमध्ये वाटीसारखे घरटे. ३-४ करडी-पांढरी, तपकिरी ठिपक्यांची अंडी.}
  {white-browed-wagtail.jpg}

\birdentry{सुभग}{\textit{Aegithina tiphia}}
  {चिमणीपेक्षा लहान}
  {नेहमी आढळणारा स्थानिक}
  {प्रजननकाळात नर चमकदार पिवळा व काळा; इतर वेळी व मादी हिरवट-पिवळी. पांढरे पंखपट्टे विशिष्ट.}
  {परिसरातील दाट वृक्षराजी असलेल्या भागात}
  {सक्रिय पक्षी, सतत पानांमध्ये फिरत राहतो. गोड संगीतमय आवाज. मुख्यतः कीटक खातो.}
  {एप्रिल ते सप्टेंबर प्रजनन. फांदीत नीटनेटके वाटीसारखे घरटे. २-३ गुलाबी अंडी, तपकिरी ठिपके.}
  {common-iora.jpg}

\birdentry{कवडी रामगंगा}{\textit{Parus cinereus}}
  {चिमणीपेक्षा लहान}
  {नेहमी आढळणारा स्थानिक}
  {छोटा करडा पक्षी, काळी टोपी व गळा, पांढरे गाल व खालचा भाग.}
  {परिसरातील वृक्षराजी व बागांमध्ये}
  {सक्रिय व कसरती, कीटक पकडताना उलटा लटकतो. कीटक व बिया खातो.}
  {मार्च ते जुलै प्रजनन. झाडांच्या पोकळीत घरटे. ४-६ पांढरी, लाल-तपकिरी ठिपक्यांची अंडी.}
  {asian-tit.jpg}

\birdentry{जांभळा शिंजीर}{\textit{Cinnyris asiaticus}}
  {चिमणीपेक्षा बराच लहान}
  {नेहमी आढळणारा स्थानिक}
  {प्रजननकाळात नर चमकदार जांभळा-काळा; इतर वेळी व मादी जैतुनी-तपकिरी वर, पिवळी खाली.}
  {परिसरातील फुलझाडे असलेल्या भागात}
  {अत्यंत सक्रिय, फुलांसमोर तरंगतो. लांब वाकडी चोच मधुरसासाठी. लहान कीटकही पकडतो.}
  {मुख्यतः फेब्रुवारी ते मे प्रजनन. झोळीसारखे लटकते घरटे. २-३ हिरवट-पांढरी, तपकिरी ठिपक्यांची अंडी.}
  {purple-sunbird.jpg}

\birdentry{जांभळ्या पुठ्ठ्याचा शिंजीर}{\textit{Leptocoma zeylonica}}
  {चिमणीपेक्षा बराच लहान}
  {नेहमी आढळणारा स्थानिक}
  {नराचा वरचा भाग चमकदार जांभळा-निळा, पोट पिवळे. मादी जैतुनी वर, पिवळी खाली.}
  {परिसरातील बागा व फुलझाडे असलेल्या भागात}
  {जांभळ्या सूर्यपक्ष्यासारखाच सक्रिय, फुलांसमोर तरंगतो. मधुरस व लहान कीटक खातो.}
  {वर्षभर प्रजनन. प्रवेशद्वारासह झोळीसारखे घरटे. २ हिरवट-पांढरी, तपकिरी ठिपक्यांची अंडी.}
  {purple-rumped-sunbird.jpg}

\indexprologue{}
\printindex[birdindex]
\end{document}