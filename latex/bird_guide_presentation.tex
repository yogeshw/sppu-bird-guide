\documentclass[aspectratio=169]{beamer}

% Theme and color customization
\usetheme{Madrid}
\usecolortheme{owl}
\setbeamercolor{normal text}{fg=white}
\setbeamercolor{background canvas}{bg=black}
\setbeamercolor{structure}{fg=white}
\setbeamercolor{frametitle}{fg=white}
\setbeamercolor{title}{fg=white}
\setbeamercolor{section in toc}{fg=white}
\setbeamertemplate{navigation symbols}{}

% Required packages
\usepackage{graphicx}
\graphicspath{{../images/}}
\usepackage{times}
\usepackage{hyperref}

% Define image credit command
\newcommand{\imagecredit}[1]{%
  \def\stripext##1.jpg{##1}%
  \edef\creditfile{\stripext#1_credit.txt}%
  {\scriptsize\em Credit: \input{../images/\creditfile}}%
}

\title{Birds of the University of Pune}
\author{Yogesh Wadadekar}
\institute{Savitribai Phule Pune University}
\date{2025}

\begin{document}

\begin{frame}
    \maketitle
\end{frame}

\begin{frame}{Campus Environment}
    \begin{columns}[T]
        \column{0.5\textwidth}
        The University of Pune campus is a biodiversity hotspot spread across 411 acres in the heart of Pune city
        \begin{itemize}
            \item Located at 18.5529°N 73.8352°E
            \item Elevation: 560m above sea level
            \item Diverse habitats support rich birdlife
            \item Includes wooded areas, grasslands, water bodies
        \end{itemize}
        \column{0.5\textwidth}
        The campus hosts over 150 species of trees:
        \begin{itemize}
            \item Banyan (\textit{Ficus benghalensis})
            \item Peepal (\textit{Ficus religiosa})
            \item Neem (\textit{Azadirachta indica})
            \item Red Silk Cotton (\textit{Bombax ceiba})
            \item Many ornamental species
        \end{itemize}
    \end{columns}
\end{frame}

\begin{frame}{When to Watch}
    \begin{columns}[T]
        \column{0.5\textwidth}
        \textbf{Time of Day}
        \begin{itemize}
            \item Early morning (6:00-9:00 AM)
            \item Late afternoon (4:00-6:30 PM)
            \item Midday for raptors
            \item Dusk for nocturnal species
        \end{itemize}
        \column{0.5\textwidth}
        \textbf{Seasonal Highlights}
        \begin{itemize}
            \item Winter (November-February)
            \item Monsoon (June-September)
            \item Summer (March-May)
            \item Post-monsoon (October)
        \end{itemize}
    \end{columns}
\end{frame}

\begin{frame}{How to Watch}
    \begin{columns}[T]
        \column{0.5\textwidth}
        \textbf{Essential Equipment}
        \begin{itemize}
            \item Binoculars (8x42 or 10x42)
            \item Field guide
            \item Notebook
        \end{itemize}
        \column{0.5\textwidth}
        \textbf{Useful Apps}
        \begin{itemize}
            \item eBird
            \item Merlin Bird ID
            \item BirdNet
        \end{itemize}
    \end{columns}
\end{frame}

% Template for bird entries
\newcommand{\birdslide}[9]{%
\begin{frame}{#1 (\textit{#2})}
    \begin{columns}[T]
        \column{0.5\textwidth}
        \includegraphics[width=\textwidth,height=0.7\textheight,keepaspectratio]{#9}
        \imagecredit{#9}
        \column{0.5\textwidth}
        \textbf{Size:} #3 \\
        \textbf{Status:} #4 \\[0.5em]
        \textbf{Field characters:} #5 \\[0.5em]
        \textbf{Best seen at:} #6 \\[0.5em]
        \textbf{Habits:} #7 \\[0.5em]
        \textbf{Nesting:} #8
    \end{columns}
\end{frame}
}

% Bird entries (organized by families)
\section{Raptors}
\birdslide{Black Kite}{\textit{Milvus migrans}}
{Kite size}
{Very common resident}
{A large bird of prey with a distinctive forked tail and long wings. The plumage is mostly blackish-brown with lighter underparts.}
{Found throughout the Indian subcontinent, including the University of Pune campus.}
{Often seen soaring high in the sky, searching for prey. Feeds on small mammals, birds, and carrion.}
{Breeds December to April. Builds nests in tall trees.}
{black-kite.jpg}

\birdslide{Red-naped Ibis}{\textit{Pseudibis papillosa}}
{Kite-}
{Common resident}
{Large black bird with red patch on nape. Long curved bill and bare red patch on crown distinctive.}
{Found in open areas, grasslands and waterbodies across campus.}
{Walks deliberately while foraging. Usually in small groups. Probes soil for insects and small vertebrates.}
{Breeds mainly during monsoon. Builds platform nest in large trees.}
{red-naped-ibis.jpg}

\birdslide{Peregrine Falcon}{\textit{Falco peregrinus}}
{Kite-}
{Uncommon winter visitor}
{Large, powerful falcon with dark grey upperparts, barred underparts, and distinctive black 'moustache' marking.}
{Occasionally seen around tall buildings and open areas of campus.}
{Swift and agile hunter, known for spectacular aerial dives. Feeds mainly on birds caught in flight.}
{Does not breed on campus; winter visitor only.}
{peregrine-falcon.jpg}

\birdslide{Shikra}{\textit{Accipiter badius}}
{Myna+}
{Common resident}
{A small hawk with a distinctive red eye and a barred chest. The upperparts are gray, and the underparts are white with fine rufous barring.}
{Widely distributed across the Indian subcontinent, including the University of Pune campus.}
{Prefers wooded areas and gardens. Feeds on small birds, mammals, and insects.}
{Breeds March to July. Builds nests in trees, using twigs and leaves.}
{shikra.jpg}

\section{Waterbirds}
\birdslide{White-breasted Waterhen}{\textit{Amaurornis phoenicurus}}
{Bulbul+}
{Common resident}
{A medium-sized bird with a white face, throat, and breast, and a dark brown body. Red patch on bill base.}
{Found throughout the Indian subcontinent, in wetland areas.}
{Prefers marshy areas and wetlands. Feeds on insects, small fish, and plant matter.}
{Builds nests in dense vegetation near water.}
{white-breasted-waterhen.jpg}

\birdslide{Indian Pond Heron}{\textit{Ardeola grayii}}
{Myna+}
{Very common resident}
{A medium-sized heron with a brownish body and white wings. It has a distinctive yellow bill and legs.}
{Found throughout the Indian subcontinent, including the University of Pune campus.}
{Prefers wetlands, ponds, and marshes. Feeds on fish, frogs, and insects.}
{Breeds June to September. Builds nests in trees or shrubs near water.}
{indian-pond-heron.jpg}

\birdslide{Black-crowned Night Heron}{\textit{Nycticorax nycticorax}}
{Crow+}
{Uncommon resident}
{A medium-sized heron with a black crown and back, gray wings, and white underparts. Red eyes.}
{Found throughout the Indian subcontinent, including the University of Pune campus.}
{Prefers wetlands, ponds, and marshes. Feeds on fish, frogs, and insects.}
{Breeds June to September. Builds nests in trees or shrubs near water.}
{black-crowned-night-heron.jpg}

\birdslide{Cattle Egret}{\textit{Bubulcus ibis}}
{Myna++}
{Very common resident}
{A medium-sized egret with white plumage and yellow bill. Develops orange-buff plumes in breeding.}
{Found throughout the Indian subcontinent, including the University of Pune campus.}
{Often seen near cattle, feeding on insects and small animals disturbed by grazing.}
{Breeds June to September during monsoon. Builds nests in colonies.}
{cattle-egret.jpg}

\birdslide{Little Egret}{\textit{Egretta garzetta}}
{Myna++}
{Common resident}
{A medium-sized egret with white plumage, black bill, and black legs with yellow feet.}
{Found throughout the Indian subcontinent, including the University of Pune campus.}
{Prefers wetlands, ponds, and marshes. Feeds on fish, frogs, and insects.}
{Breeds June to September. Builds nests in colonies.}
{little-egret.jpg}

\section{Pigeons and Doves}
\birdslide{Blue Rock Pigeon}{\textit{Columba livia}}
{Myna+}
{Very common resident}
{A plump, grey-colored bird with iridescent neck feathers. Two dark wingbars.}
{Found throughout the Indian subcontinent, particularly abundant in urban areas.}
{Highly adapted to urban life. Often seen in flocks, feeding on grains and seeds.}
{Breeds throughout the year. Nests in building ledges and structures.}
{blue-rock-pigeon.jpg}

\birdslide{Spotted Dove}{\textit{Streptopelia chinensis}}
{Myna}
{Common resident}
{A slim, long-tailed dove with pinkish-grey plumage and distinctive black-and-white spotting.}
{Widespread across the campus and surrounding areas.}
{Usually seen in pairs or small groups, foraging on the ground for seeds.}
{Builds a flimsy platform nest in trees and shrubs.}
{spotted-dove.jpg}

\birdslide{Little Brown Dove}{\textit{Streptopelia senegalensis}}
{Myna-}
{Common resident}
{A small, delicate dove with pale brown plumage and lilac tinge to neck.}
{Common throughout the campus, especially in open areas.}
{Often seen in pairs, feeding on seeds on the ground. Soft, musical call.}
{Creates simple twig nests in bushes and low trees.}
{little-brown-dove.jpg}

\section{Waders}
\birdslide{Red-wattled Lapwing}{\textit{Vanellus indicus}}
{Myna++}
{Common resident}
{A large plover with distinctive red wattles in front of eyes.}
{Found in open areas across the campus, particularly near water bodies.}
{Known for its alarm call "did-he-do-it". Active during day and night.}
{Nests on the ground in shallow scrapes.}
{red-wattled-lapwing.jpg}

\section{Songbirds}
\birdslide{Common Myna}{\textit{Acridotheres tristis}}
{Myna}
{Very common resident}
{Brown body with black head, yellow bill and legs. White wing patches visible in flight.}
{Abundant throughout campus in all habitats.}
{Bold, adaptable bird. Often in pairs or groups. Omnivorous diet.}
{Nests in holes in trees and buildings.}
{common-myna.jpg}

\section{Owls}
\birdslide{Spotted Owlet}{\textit{Athene brama}}
{Myna+}
{Common resident}
{Small, chunky owl with white spotting. Large yellow eyes and white eyebrows give it a stern expression.}
{Found in wooded areas and gardens across campus.}
{Active at dawn and dusk. Bobs head when alert. Feeds on insects, small vertebrates.}
{Breeds February to April. Nests in tree hollows.}
{spotted-owlet.jpg}

\birdslide{Mottled Wood Owl}{\textit{Strix ocellata}}
{Crow+}
{Uncommon resident}
{Large owl with distinctive mottled brown plumage. Round head without ear-tufts.}
{Found in wooded areas of campus with dense canopy cover.}
{Strictly nocturnal. Deep resonant calls at night. Feeds on rodents, birds, and large insects.}
{Breeds December to March. Nests in tree hollows.}
{mottled-wood-owl.jpg}

\section{Woodpeckers and Barbets}
\birdslide{Coppersmith Barbet}{\textit{Megalaima haemacephala}}
{Bulbul+}
{Common resident}
{Small, green bird with crimson forehead and throat. Yellow eye-patch distinctive.}
{Found in wooded areas throughout the campus.}
{Known for its metallic 'tuk-tuk-tuk' call. Feeds mainly on fruits, especially figs.}
{Excavates nest holes in dead tree trunks or branches.}
{coppersmith-barbet.jpg}

\section{Cuckoos and Coucals}
\birdslide{Asian Koel}{\textit{Eudynamys scolopacea}}
{Crow-}
{Common resident}
{Male glossy black with red eyes; female brown with white spots, barred underneath.}
{Common throughout the campus, especially in wooded areas.}
{Known for its distinctive "ku-oo" call. Parasitizes crows' nests.}
{Brood parasite, lays eggs in nests of crows and other birds.}
{asian-koel.jpg}

\birdslide{Common Hawk-Cuckoo}{\textit{Hierococcyx varius}}
{Myna++}
{Common resident}
{Hawk-like appearance with barred underparts. Yellow eye-ring.}
{Found throughout wooded areas of campus.}
{Known for distinctive "brain-fever" call. Feeds on insects and caterpillars.}
{Brood parasite, primarily targeting babblers.}
{common-hawk-cuckoo.jpg}

\birdslide{Grey-bellied Cuckoo}{\textit{Cacomantis passerinus}}
{Myna-}
{Common summer visitor}
{Grey bird with dark grey breast and pale belly. Yellow eye-ring distinctive.}
{Found in wooded areas across campus.}
{Rather secretive. Known for plaintive whistling calls. Feeds mainly on insects.}
{Brood parasite, laying eggs in warblers' and tailorbirds' nests.}
{grey-bellied-cuckoo.jpg}

\birdslide{Greater Coucal}{\textit{Centropus sinensis}}
{Crow}
{Common resident}
{Large, black bird with chestnut wings and long tail. Red eyes and curved black bill.}
{Found in dense vegetation across the campus.}
{Skulking habit, often seen walking on ground. Deep booming calls.}
{Builds a domed nest in dense vegetation.}
{greater-coucal.jpg}

\section{Parakeets}
\birdslide{Rose-ringed Parakeet}{\textit{Psittacula krameri}}
{Myna++}
{Very common resident}
{Bright green parakeet with long tail. Males have pink and black neck ring.}
{Abundant throughout the campus, especially in wooded areas.}
{Noisy, gregarious birds. Often seen in large flocks.}
{Breeds December to May. Nests in tree hollows.}
{rose-ringed-parakeet.jpg}

\birdslide{Alexandrine Parakeet}{\textit{Psittacula eupatria}}
{Crow}
{Uncommon resident}
{Larger than Rose-ringed Parakeet, with massive red beak.}
{Found in wooded areas of the campus, less common than other parakeets.}
{Similar habits to Rose-ringed Parakeet but more wary.}
{Nests in large tree cavities, preferring old growth trees.}
{alexandrine-parakeet.jpg}

\birdslide{Plum-headed Parakeet}{\textit{Psittacula cyanocephala}}
{Myna}
{Common resident}
{Male has purple-red head, female bluish-grey. Both have green body.}
{Regular visitor to wooded areas of the campus.}
{Usually in pairs or small flocks. More arboreal than other parakeets.}
{Nests in tree hollows, breeding season varies with location.}
{plum-headed-parakeet.jpg}

\section{Swifts}
\birdslide{House Swift}{\textit{Apus affinis}}
{Myna-}
{Very common resident}
{Small swift with white rump and forked tail. Dark brown-black plumage.}
{Common throughout urban areas of the campus.}
{Aerial feeder, constantly on wing. Often seen in large groups.}
{Colonial nester, builds mud nests under building eaves.}
{house-swift.jpg}

\birdslide{Asian Palm Swift}{\textit{Cypsiurus balasiensis}}
{Sparrow-}
{Common resident}
{Smaller than House Swift, slender with deeply forked tail.}
{Found around palm trees across campus.}
{Very fast flier, rarely perches. Often seen around palm trees.}
{Breeds year-round. Attaches small nest to palm frond underside.}
{asian-palm-swift.jpg}

\section{Kingfishers}
\birdslide{White-breasted Kingfisher}{\textit{Halcyon smyrnensis}}
{Myna+}
{Common resident}
{Large kingfisher with bright blue back, chocolate brown head, white breast.}
{Found near water bodies and in gardens across campus.}
{Perches prominently, diving for prey. Feeds on fish, frogs, lizards.}
{Breeds March to June. Nests in horizontal tunnels.}
{white-breasted-kingfisher.jpg}

\birdslide{Common Kingfisher}{\textit{Alcedo atthis}}
{Bulbul-}
{Common resident}
{Small kingfisher with brilliant blue upperparts and orange underparts.}
{Found near water bodies and ponds on campus.}
{Perches low over water, dives vertically for fish.}
{Breeds February to April. Nests in tunnel in earth bank.}
{common-kingfisher.jpg}

\section{Bee-eaters and Barbets}
\birdslide{Asian Green Bee-eater}{\textit{Merops orientalis}}
{Sparrow+}
{Common resident}
{Slender green bird with blue throat patch and black eye-stripe.}
{Found in open areas with scattered trees across campus.}
{Graceful aerial hunter, catches insects in flight.}
{Breeds March to June. Nests in tunnels dug in earth banks.}
{asian-green-bee-eater.jpg}

\birdslide{Coppersmith Barbet}{\textit{Megalaima haemacephala}}
{Bulbul+}
{Common resident}
{Small, green bird with crimson forehead and throat. Yellow eye-patch distinctive.}
{Found in wooded areas throughout the campus.}
{Known for its metallic 'tuk-tuk-tuk' call. Feeds mainly on fruits, especially figs.}
{Excavates nest holes in dead tree trunks or branches.}
{coppersmith-barbet.jpg}

\section{Swallows and Martins}
\birdslide{Dusky Crag Martin}{\textit{Hirundo concolor}}
{Sparrow-}
{Common resident}
{Small, dark brown martin with slightly forked tail.}
{Found around buildings and rocky areas of campus.}
{Aerial feeder, catching insects in flight.}
{Builds cup-shaped mud nests on buildings.}
{dusky-crag-martin.jpg}

\birdslide{Barn Swallow}{\textit{Hirundo rustica}}
{Sparrow}
{Common winter visitor}
{Long, deeply forked tail. Blue-black above, rufous forehead and throat.}
{Winter visitor to campus, seen in open areas.}
{Graceful flier, catching insects on wing. Often perches on wires.}
{Does not breed on campus; winter visitor only.}
{barn-swallow.jpg}

\birdslide{Red-rumped Swallow}{\textit{Hirundo daurica}}
{Sparrow}
{Common resident}
{Similar to Barn Swallow but with rufous rump and shorter tail streamers.}
{Found throughout campus, especially near buildings.}
{Aerial feeder, often in mixed flocks with other swallows.}
{Builds bottle-shaped mud nests under structures.}
{red-rumped-swallow.jpg}

\birdslide{Wire-tailed Swallow}{\textit{Hirundo smithii}}
{Sparrow}
{Common resident}
{Distinctive long wire-like tail streamers. Glossy blue-black above.}
{Found near water bodies and open areas on campus.}
{Graceful flier, often perches on wires. Feeds on flying insects.}
{Builds cup-shaped mud nests under bridges and buildings.}
{wire-tailed-swallow.jpg}

\section{Shrikes and Orioles}
\birdslide{Long-tailed Shrike}{\textit{Lanius schach}}
{Bulbul+}
{Common resident}
{Grey-brown above, whitish below with black mask through eye.}
{Found in open areas with scattered trees.}
{Hunts from prominent perches. Impales prey on thorns.}
{Builds neat cup-shaped nest in thorny bushes.}
{long-tailed-shrike.jpg}

\birdslide{Golden Oriole}{\textit{Oriolus oriolus}}
{Myna+}
{Common resident}
{Male bright yellow with black wings and tail; female duller greenish.}
{Found in wooded areas across campus.}
{Shy bird, keeps to canopy. Melodious flute-like calls.}
{Builds hammock-like nest suspended in tree fork.}
{golden-oriole.jpg}

\section{Drongos}
\birdslide{Black Drongo}{\textit{Dicrurus macrocercus}}
{Myna+}
{Common resident}
{Glossy black bird with deeply forked tail. Strong hooked bill.}
{Found throughout campus in open areas with trees.}
{Bold and aggressive. Often seen chasing other birds. Excellent aerial hunter.}
{Breeds March to July. Builds cup nest in outer branches.}
{black-drongo.jpg}

\birdslide{Ashy Drongo}{\textit{Dicrurus leucophaeus}}
{Myna+}
{Winter visitor}
{Similar to Black Drongo but grey overall. Less deeply forked tail.}
{Winter visitor to wooded areas of campus.}
{Less aggressive than Black Drongo. Catches insects in air.}
{Does not breed on campus; winter visitor only.}
{ashy-drongo.jpg}

\section{Mynas and Starlings}
\birdslide{Common Myna}{\textit{Acridotheres tristis}}
{Myna}
{Very common resident}
{Brown body with black head, yellow bill and legs. White wing patches visible in flight.}
{Abundant throughout campus in all habitats.}
{Bold, adaptable bird. Often in pairs or groups. Omnivorous diet.}
{Nests in holes in trees and buildings.}
{common-myna.jpg}

\birdslide{Jungle Myna}{\textit{Acridotheres fuscus}}
{Myna}
{Common resident}
{Similar to Common Myna but greyer, with tuft at base of bill.}
{Found in wooded areas and gardens, less urban than Common Myna.}
{Usually in pairs or small groups. More arboreal than Common Myna.}
{Breeds March to July. Nests in tree holes.}
{jungle-myna.jpg}

\birdslide{Brahminy Starling}{\textit{Sturnia pagodarum}}
{Myna}
{Common resident}
{Distinctive buff-colored starling with black crest, grey wings.}
{Found in wooded areas and gardens across campus.}
{Usually seen in pairs or small groups. Feeds on fruits and insects.}
{Breeds April to July. Nests in tree holes.}
{brahminy-starling.jpg}

\section{Robins and Chats}
\birdslide{Oriental Magpie-Robin}{\textit{Copsychus saularis}}
{Bulbul}
{Common resident}
{Striking black and white plumage in male, greyer in female.}
{Found throughout campus in wooded areas and gardens.}
{Bold and melodious songster. Feeds on insects and berries.}
{Breeds March to July. Nests in tree hollows.}
{oriental-magpie-robin.jpg}

\birdslide{Spot-breasted Fantail}{\textit{Rhipidura albogularis}}
{Sparrow-}
{Common resident}
{Small black and white bird with spotted breast, white eyebrow.}
{Found in wooded areas and gardens with good tree cover.}
{Active and acrobatic, frequently fans tail while foraging.}
{Breeds mainly March to September. Builds neat cup nest.}
{spot-breasted-fantail.jpg}

\birdslide{Pied Bushchat}{\textit{Saxicola caprata}}
{Bulbul-}
{Common resident}
{Male black with white rump and wing patches; female brown.}
{Found in open areas with scattered bushes across campus.}
{Perches prominently on bushes and fences.}
{Breeds March to August. Builds cup-shaped nest.}
{pied-bushchat.jpg}

\section{Sunbirds and Flowerpeckers}
\birdslide{Purple Sunbird}{\textit{Cinnyris asiaticus}}
{Sparrow--}
{Common resident}
{Male breeding metallic purple-black; non-breeding and female olive brown.}
{Found throughout campus wherever flowering plants occur.}
{Very active, hovers at flowers. Long curved bill for nectar feeding.}
{Breeds mainly February to May. Builds hanging purse-like nest.}
{purple-sunbird.jpg}

\birdslide{Tickell's Flowerpecker}{\textit{Dicaeum erythrorhynchos}}
{Sparrow--}
{Common resident}
{Tiny bird with grey-brown above, dirty white below. Short tail and curved bill.}
{Found wherever flowering and fruiting trees occur.}
{Very active, flits rapidly between flowers. Feeds on berries and nectar.}
{Breeds year-round, builds pear-shaped hanging nest.}
{tickells-flowerpecker.jpg}

\section{Wagtails}
\birdslide{Grey Wagtail}{\textit{Motacilla cinerea}}
{Sparrow}
{Common winter visitor}
{Slender bird with long tail, grey upper parts and bright yellow underparts.}
{Found near water bodies and damp areas on campus.}
{Active forager, running and walking while wagging tail.}
{Does not breed on campus; winter visitor only.}
{grey-wagtail.jpg}

\section{Drongos and Orioles}
\birdslide{Black Drongo}{\textit{Dicrurus macrocercus}}
{Myna+}
{Common resident}
{Glossy black bird with deeply forked tail. Strong hooked bill.}
{Found throughout campus in open areas with trees.}
{Bold and aggressive. Often seen chasing other birds. Excellent aerial hunter.}
{Breeds March to July. Builds cup nest in outer branches.}
{black-drongo.jpg}

\birdslide{Golden Oriole}{\textit{Oriolus oriolus}}
{Myna+}
{Common resident}
{Male bright yellow with black wings and tail; female duller greenish.}
{Found in wooded areas across campus.}
{Shy bird, keeps to canopy. Melodious flute-like calls.}
{Builds hammock-like nest suspended in tree fork.}
{golden-oriole.jpg}

\section{Colorful Residents}
\birdslide{Indian Roller}{\textit{Coracias benghalensis}}
{Myna++}
{Common resident}
{Crow-sized bird with brilliant blue wings and tail. Brown head and back.}
{Found in open areas with scattered trees.}
{Makes dramatic rolling display flights. Feeds on large insects, small lizards.}
{Breeds March to July. Nests in tree hollows.}
{indian-roller.jpg}

\birdslide{Indian Grey Hornbill}{\textit{Ocyceros birostris}}
{Crow}
{Common resident}
{Large grey bird with long tail and distinctive casque on bill.}
{Found in wooded areas across campus with large old trees.}
{Usually seen in pairs, moving from tree to tree. Feeds mainly on fruits.}
{Female seals herself in tree cavity during breeding.}
{indian-grey-hornbill.jpg}

\section{Winter Visitors}
\birdslide{Peregrine Falcon}{\textit{Falco peregrinus}}
{Kite-}
{Uncommon winter visitor}
{Large, powerful falcon with dark grey upperparts and distinctive black 'moustache'.}
{Occasionally seen around tall buildings and open areas.}
{Swift and agile hunter, known for spectacular aerial dives.}
{Does not breed on campus; winter visitor only.}
{peregrine-falcon.jpg}

\section{Bulbuls}
\birdslide{Red-vented Bulbul}{\textit{Pycnonotus cafer}}
{Bulbul}
{Very common resident}
{Dark brown-black with scaled pattern, distinctive red vent.}
{Common throughout campus in all habitats.}
{Active and noisy. Often in pairs or small groups.}
{Breeds year-round, peak February to May.}
{red-vented-bulbul.jpg}

\birdslide{Red-whiskered Bulbul}{\textit{Pycnonotus jocosus}}
{Bulbul}
{Common resident}
{Brown above, white below with red cheek patch and pointed crest.}
{Found in wooded areas and gardens across campus.}
{Lively bird with pleasant calls. Often in pairs.}
{Builds neat cup-shaped nest in shrubs.}
{red-whiskered-bulbul.jpg}

\section{Babblers}
\birdslide{Large Grey Babbler}{\textit{Turdoides malcolmi}}
{Myna+}
{Common resident}
{Grey-brown bird with pale streaking. Long graduated tail.}
{Found in open scrub and gardens across campus.}
{Lives in noisy groups. Forages on ground, flicking leaves.}
{Breeds March to September. Builds untidy nest in thorny bushes.}
{large-grey-babbler.jpg}

\section{Flycatchers}
\birdslide{Tickell's Blue Flycatcher}{\textit{Muscicapa tickelliae}}
{Bulbul-}
{Common resident}
{Male bright blue above, rufous below. Female duller.}
{Found in wooded areas with good undergrowth.}
{Active bird, makes short flycatching sallies from perch.}
{Breeds April to July. Builds neat cup nest in tree hollow.}
{tickells-blue-flycatcher.jpg}

\section{Prinias and Warblers}
\birdslide{Ashy Prinia}{\textit{Prinia socialis}}
{Sparrow-}
{Common resident}
{Ash-grey above, whitish below. Long graduated tail often cocked.}
{Found in gardens and scrub throughout campus.}
{Active bird, moves through vegetation making harsh calls.}
{Breeds mainly June to September. Builds deep cup nest.}
{ashy-prinia.jpg}

\birdslide{Jungle Prinia}{\textit{Prinia sylvatica}}
{Sparrow-}
{Common resident}
{Grey-brown above, pale below. Darker than Ashy Prinia.}
{Found in scrub jungle and grassland areas.}
{Skulking habit, reveals presence by loud calls.}
{Breeds June to September. Builds ball-shaped nest.}
{jungle-prinia.jpg}

\birdslide{Common Tailorbird}{\textit{Orthotomus sutorius}}
{Sparrow--}
{Common resident}
{Tiny warbler with long upright tail. Olive-green above.}
{Found throughout campus in gardens and dense bushes.}
{Active bird, constantly moving through vegetation.}
{Makes remarkable nest by stitching living leaves together.}
{common-tailorbird.jpg}

\birdslide{Lesser Whitethroat}{\textit{Sylvia curruca}}
{Sparrow-}
{Common winter visitor}
{Small warbler with grey-brown upper parts, white throat.}
{Found in scrub and bushes across campus.}
{Active but skulking. Forages in bushes for insects.}
{Does not breed on campus; winter visitor only.}
{lesser-whitethroat.jpg}

\section{Sparrows and Weavers}
\birdslide{House Sparrow}{\textit{Passer domesticus}}
{Sparrow}
{Very common resident}
{Male grey-brown with black bib; female plain brown.}
{Common around buildings and human habitation.}
{Gregarious, feeds mainly on grains and seeds.}
{Breeds almost year-round. Nests in holes in buildings.}
{house-sparrow.jpg}

\birdslide{Baya Weaver}{\textit{Ploceus philippinus}}
{Sparrow}
{Common resident}
{Male bright yellow in breeding; female sparrow-like.}
{Found in grassland areas and reed beds on campus.}
{Colonial nester. Males build elaborate pendant nests.}
{Breeds during monsoon. Males build multiple helmet-shaped nests.}
{baya-weaver.jpg}

\section{Small Insectivores}
\birdslide{Oriental White-eye}{\textit{Zosterops palpebrosus}}
{Sparrow-}
{Common resident}
{Tiny olive-green bird with distinctive white eye-ring.}
{Found in wooded areas and gardens across campus.}
{Active and acrobatic. Moves in small flocks through foliage.}
{Breeds March to September. Builds neat cup nest.}
{oriental-white-eye.jpg}

\birdslide{Scaly-breasted Munia}{\textit{Lonchura punctulata}}
{Sparrow-}
{Common resident}
{Small brown finch with distinctive scaly pattern on breast.}
{Found in grassy areas and scrub throughout campus.}
{Social, feeds in small flocks on grass seeds.}
{Breeds July to October. Builds large ball nest.}
{scaly-breasted-munia.jpg}

\section{Corvids}
\birdslide{Rufous Treepie}{\textit{Dendrocitta vagabunda}}
{Myna++}
{Common resident}
{Large rufous and black bird with long graduated tail.}
{Found in wooded areas across campus.}
{Bold and inquisitive. Often in small groups.}
{Breeds March to July. Builds neat cup nest in tree fork.}
{rufous-treepie.jpg}

\birdslide{House Crow}{\textit{Corvus splendens}}
{Crow}
{Very common resident}
{Grey and black crow with distinct grey neck collar.}
{Abundant throughout campus, especially near human activity.}
{Bold and opportunistic. Omnivorous diet.}
{Breeds March to July. Builds platform nest in trees.}
{house-crow.jpg}

\birdslide{Large-billed Crow}{\textit{Corvus macrorhynchos}}
{Crow+}
{Common resident}
{Larger than House Crow, all black with massive bill.}
{Found in wooded areas of campus and surrounding regions.}
{More cautious than House Crow. Often seen singly or in pairs.}
{Breeds March to June. Builds large platform nest high in trees.}
{large-billed-crow.jpg}

\section{Conclusion}
\begin{frame}{How to Contribute}
    \begin{itemize}
        \item Submit your bird sightings to eBird
        \item Share your photographs
        \item Report any new species
        \item Help maintain bird-friendly habitats
        \item Follow ethical birdwatching practices
    \end{itemize}
    \vspace{1em}
    \centering
    Visit the SPPU eBird Hotspot: \url{https://ebird.org/hotspot/L1838309}
\end{frame}

\begin{frame}{Best Birdwatching Spots on Campus}
    \begin{columns}[T]
        \column{0.5\textwidth}
        \textbf{Habitat Types}
        \begin{itemize}
            \item Wooded areas near Main Building
            \item Garden Department grounds
            \item Campus water bodies
            \item Open grasslands
            \item Old growth trees
        \end{itemize}
        
        \column{0.5\textwidth}
        \textbf{Tips}
        \begin{itemize}
            \item Visit different habitats
            \item Follow quiet paths
            \item Check flowering trees
            \item Look for fruiting trees
            \item Watch for mixed feeding flocks
        \end{itemize}
    \end{columns}
\end{frame}

\begin{frame}{Conservation Concerns}
    \begin{itemize}
        \item Habitat preservation
        \item Protection of old trees
        \item Maintaining water bodies
        \item Reducing disturbance
        \item Native plant species
        \item Light pollution control
        \item Responsible development
    \end{itemize}
\end{frame}

\begin{frame}[plain]{Thank You}
    \begin{center}
        \vspace{2em}
        {\Large Thank you for your attention}
        \vspace{2em}
        
        For more information and updates:\\
        Visit the SPPU eBird Hotspot\\
        \url{https://ebird.org/hotspot/L1838309}
        
        \vspace{1em}
        {\small Presentation created by Yogesh Wadadekar}
    \end{center}
\end{frame}

\end{document}