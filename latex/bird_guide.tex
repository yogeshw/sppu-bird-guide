\documentclass[a4paper,12pt,landscape]{memoir}
\usepackage{geometry}
\usepackage{graphicx}
\graphicspath{{../images/}}
\usepackage{times}      % For Times New Roman-like font
\usepackage{helvet}     % For Arial-like font
\usepackage{xcolor}
\usepackage{hyperref}
\usepackage{mdframed}
\usepackage{tikz}
\usetikzlibrary{decorations.pathmorphing}
\usepackage{makeidx}
\makeindex

% Update geometry settings for A4 landscape with two A5 pages
\geometry{
  a4paper,
  landscape,
  left=1cm,
  right=1cm,
  top=2cm,
  bottom=2cm,
  includehead
}

% Font configuration
\newcommand{\headfont}{\sffamily\bfseries}  % Define headfont command first
\renewcommand{\familydefault}{\rmdefault}    % Times New Roman-like font as default

% Chapter and section styling
\chapterstyle{bianchi}
\setsecheadstyle{\headfont\color{headingcolor}\Large}
\setsubsecheadstyle{\headfont\color{headingcolor}}

% Define color scheme
\definecolor{headingcolor}{RGB}{34, 139, 34} % Forest Green
\definecolor{highlightcolor}{RGB}{139, 69, 19} % Saddle Brown

% Define command for missing images
\newcommand{\missingimage}[1]{%
  \framebox[0.4\textwidth]{%
    \parbox{0.35\textwidth}{%
      \centering
      Image placeholder\\
      #1%
    }%
  }%
}

% Simple credit file handling - just strip .jpg and add _credit.txt
\newcommand{\getcreditname}[1]{%
  \def\stripext##1.jpg{##1}%
  \edef\creditfile{\stripext#1_credit.txt}%
}

% Add counter for birds
\newcounter{birdnumber}
\setcounter{birdnumber}{0}

% Simple index commands without @ characters
\makeatletter
\newcommand{\indexnames}[2]{%
  \index{\textit{#2}}% Latin name
  \index{#1}% Full English name
  \expandafter\makesurnameindex\expandafter{#1}% Surname, Firstname format
}

\newcommand{\makesurnameindex}[1]{%
  \def\two##1 ##2\@nil{%
    \ifx\relax##2\relax
      \index{#1}%
    \else
      \index{##2, ##1}%
    \fi
  }%
  \expandafter\two#1 \@nil
}
\makeatother

% Update bird entry template to take explicit image filename
\newcommand{\birdentry}[9]{%
  \stepcounter{birdnumber}%
  \indexnames{#1}{#2}%
  \begin{minipage}[t]{0.48\textwidth}
    \begin{mdframed}[
      linecolor=highlightcolor,
      linewidth=1pt,
      roundcorner=5pt,
      leftmargin=0pt,
      rightmargin=0pt
    ]
      \begin{center}
      \IfFileExists{../images/#9}{%
        \includegraphics[width=0.95\textwidth,height=0.8\textheight,keepaspectratio]
        {../images/#9}%
        \par\vspace{0.5em}% Increased from -1em to 0.5em for more space
        \getcreditname{#9}%
        \hfill{\small\em Credit: \input{../images/\creditfile}}%
      }{%
        \missingimage{#1}%
      }%
      \end{center}
    \end{mdframed}
  \end{minipage}\hfill
  \begin{minipage}[t]{0.48\textwidth}
    \section*{\thebirdnumber. #1 (#2)}%
    \begin{mdframed}[
      linecolor=headingcolor,
      linewidth=1pt,
      roundcorner=5pt,
      leftmargin=0pt,
      rightmargin=0pt,
      backgroundcolor=headingcolor!5
    ]
      {\sffamily\bfseries Size:} #3 \\[0.5em]
      {\sffamily\bfseries Status:} #4 \\[0.5em]
      {\sffamily\bfseries Field characters:} #5 \\[0.5em]
      {\sffamily\bfseries Best seen at:} #6 \\[0.5em]
      {\sffamily\bfseries Habits:} #7 \\[0.5em]
      {\sffamily\bfseries Nesting:} #8
    \end{mdframed}
  \end{minipage}
  \newpage
}

% Update introduction section command similarly
\newcommand{\introsection}[2]{%
  \begin{minipage}[t]{0.48\textwidth}
    \begin{mdframed}[
      linecolor=headingcolor,
      linewidth=1pt,
      roundcorner=5pt,
      leftmargin=0pt,
      rightmargin=0pt,
      backgroundcolor=headingcolor!5
    ]
      #1
    \end{mdframed}
  \end{minipage}\hfill
  \begin{minipage}[t]{0.48\textwidth}
    \begin{mdframed}[
      linecolor=headingcolor,
      linewidth=1pt,
      roundcorner=5pt,
      leftmargin=0pt,
      rightmargin=0pt,
      backgroundcolor=headingcolor!5
    ]
      #2
    \end{mdframed}
  \end{minipage}
  \newpage
}

% Page style
\makepagestyle{birdguide}
\makeevenhead{birdguide}{\thepage}{}{}
\makeoddhead{birdguide}{}{}{\thepage}
\pagestyle{birdguide}

% Title page
\title{Birds of the University of Pune}
\author{}
\date{}

\begin{document}

\maketitle

% Copyright notice
\begin{center}
\vspace{1cm}
{\large This work is licensed under a Creative Commons Attribution-NonCommercial-ShareAlike 4.0 International License (CC BY-NC-SA 4.0).}

\vspace{0.5cm}
{\normalsize You are free to:
\begin{itemize}
\item Share -- copy and redistribute the material in any medium or format
\item Adapt -- remix, transform, and build upon the material
\end{itemize}

Under the following terms:
\begin{itemize}
\item Attribution -- You must give appropriate credit, provide a link to the license, and indicate if changes were made.
\item NonCommercial -- You may not use the material for commercial purposes.
\item ShareAlike -- If you remix, transform, or build upon the material, you must distribute your contributions under the same license.
\end{itemize}
}

\vspace{0.5cm}
{\small For more information, visit: \url{https://creativecommons.org/licenses/by-nc-sa/4.0/}}

\vspace{1cm}
{\large\textcopyright{} 2025 Yogesh Wadadekar}
\end{center}

% Descriptive information
\chapter*{Introduction}

\introsection{%
  The University of Pune (Savitribai Phule Pune University) campus is a biodiversity hotspot spread across 411 acres (166.33 hectares) in the heart of Pune city. Located at 18.5529°N 73.8352°E, the campus sits at an elevation of approximately 560 meters above sea level.

  \section*{Campus Environment}
  The campus features diverse habitats that support a rich variety of birdlife. The landscape includes extensive wooded areas with native and exotic tree species, open grasslands, lawns, artificial water bodies, gardens, and building complexes. This mosaic of habitats provides excellent opportunities for both resident and migratory birds.

  \section*{Flora and Vegetation}
  The university grounds host over 150 species of trees, creating a veritable urban forest. Native species like Banyan (\textit{Ficus benghalensis}), Peepal (\textit{Ficus religiosa}), Neem (\textit{Azadirachta indica}), and Red Silk Cotton (\textit{Bombax ceiba}) dominate the landscape, while ornamental and introduced species like African Blackwood (\textit{Dalbergia melanoxylon}), Rain Tree (\textit{Samanea saman}), Gulmohar (\textit{Delonix regia}), and Copper Pod (\textit{Peltophorum pterocarpum}) add to the diversity. This rich vegetation provides vital nesting sites, roosting areas, and food sources for various bird species.
}{%
   
  \section*{Human Impact}
  The campus ecosystem faces various anthropogenic pressures that affect bird populations:
  \begin{itemize}
  \item Regular vehicular movement along campus roads
  \item Ongoing construction and development activities
  \item Presence of a large student population
  \item Regular maintenance activities including gardening and landscaping
  \item Noise and dust pollution from nearby urban areas
  \item Light pollution from campus buildings and facilities
  \end{itemize}
}

\introsection{%
  \section*{When to Watch}
  \textbf{Time of Day:}
  \begin{itemize}
  \item Early morning (6:00-9:00 AM): Peak activity as birds forage for breakfast
  \item Late afternoon (4:00-6:30 PM): Second wave of feeding activity
  \item Midday: Good for watching raptors soaring on thermal currents
  \item Dusk: Best time to spot nocturnal species like owls becoming active
  \end{itemize}

  \textbf{Seasonal Highlights:}
  \begin{itemize}
  \item Winter (November-February): Peak season with numerous winter visitors
  \item Monsoon (June-September): Best for watching breeding activities
  \item Summer (March-May): Good for observing resident breeding birds
  \item Post-monsoon (October): Opportunity to spot passage migrants
  \end{itemize}
}{%
  \section*{Acknowledgments}
  Special thanks to the dedicated birdwatchers 
  who have contributed extensively to documenting 
  the avifauna of SPPU campus through eBird 
(\href{https://ebird.org/hotspot/L1838309}{SPPU eBird Hotspot}).
   
  Their systematic observations have greatly enhanced our 
  understanding of bird diversity and seasonal patterns 
  on campus.
}

\introsection{%
  \section*{How to Watch}
  \textbf{Essential Equipment:}
  \begin{itemize}
  \item Binoculars: 8x42 or 10x42 recommended for beginners
  \item Field guide: I highly recommend 
  \textit{Birds of the Indian Subcontinent} by Grimmett, Inskipp \& Inskipp
  \item Notebook for recording observations
  \end{itemize}

  \textbf{Useful Apps:}
  \begin{itemize}
  \item eBird: For recording sightings and exploring hotspots
  \item Merlin Bird ID: Helps identify birds through photos or sounds
  \item BirdNet: AI-powered bird sound identification
  \end{itemize}
}{%
  \section*{More Tips}
  \begin{itemize}
  \item Wear earth-toned clothing to blend with surroundings
  \item Move slowly and quietly to avoid startling birds
  \item Avoid using flash photography
  \item Keep a safe distance from nests and breeding areas
  \item Record your observations in eBird after each session
  \end{itemize}
}

\introsection{%
  \section*{About This Guide}
  This field guide documents the diverse bird species 
  found within the university campus. The most common 
  species, which represent about half the bird species 
  recorded on campus, are included here.
  
  Each entry includes 
  detailed information about the bird's appearance, 
  behavior, and habitat preferences, 
  accompanied by high-quality photographs. 
  The nomenclature follows current taxonomic standards, 
  with both scientific and common names provided for 
  easy reference.

  This source materials for this guide are available online at 
  \href{http://github.com/yogeshw/sppu-bird-guide}{http://github.com/yogeshw/sppu-bird-guide}. 
  Issues and pull requests may be submitted there. 
}

\vspace{1cm}

% Bird sections
\chapter{Birds}

\birdentry{Black Kite}{\textit{Milvus migrans}}
  {Kite size}
  {Very common resident}
  {Large raptor with forked tail. Dark brown plumage. Wings held in shallow V when soaring. Wings appear pale from below.}
  {Throughout campus, especially near human habitation}
  {Scavenges around garbage dumps. During breeding season hunts insects and small vertebrates. Known to rob food from crows.}
  {Breeds December to April. Builds large stick nest in tall trees. Clutch of 2-3 white eggs. Both parents incubate for 30-35 days.}
  {black-kite.jpg}

\birdentry{Red-naped Ibis}{\textit{Pseudibis papillosa}}
  {Kite-sized}
  {Common resident}
  {Large black bird with red patch on nape and bare red crown patch. Long curved bill. Metallic sheen on neck and wings.}
  {Open areas, grasslands and moist ground}
  {Usually in small flocks. Probes marshy areas for insects, snails and small frogs.}
  {Breeds March to August. Platform nest in large trees. Clutch of 2-4 pale blue eggs. Both parents incubate for 28-30 days.}
  {red-naped-ibis.jpg}

\birdentry{Peregrine Falcon}{\textit{Falco peregrinus peregrinator}}
  {Kite-sized}
  {Uncommon resident}
  {Powerful falcon with dark slaty upperparts, white-to-buffish barred underparts, and broad black moustachial stripe.}
  {Around tall buildings and open areas}
  {Makes spectacular stoops on pigeons and waterfowl from great heights. Extremely fast and agile hunter.}
  {Breeds February to April. Nests on cliff ledges or tall buildings. Clutch of 3-4 brownish-spotted eggs. Both parents participate in nesting duties.}
  {peregrine-falcon.jpg}

\birdentry{Shikra}{\textit{Accipiter badius}}
  {Myna+}
  {Common resident}
  {Small compact hawk. Male blue-grey above, rufous barred white below. Female browner above. Deep red iris, yellow legs distinctive.}
  {Wooded areas, gardens and groves across campus}
  {Bold hunter, makes swift dashes at prey from concealed perches. Feeds mainly on lizards, large insects, small birds and rodents.}
  {Breeds March to June. Builds neat stick nest high in trees. Clutch of 3-4 bluish-white eggs. Female incubates for 28-30 days.}
  {shikra.jpg}

\birdentry{White-breasted Waterhen}{\textit{Amaurornis phoenicurus}}
  {Bulbul+}
  {Common resident}
  {Dark slaty-grey above, white face, neck and breast. Red base to bill and undertail. Legs greenish-yellow.}
  {Around ponds, marshes and wet ditches with good cover}
  {Skulking but not particularly shy. Walks deliberately, flicking tail. Feeds on insects, molluscs, seeds and vegetable matter.}
  {Breeds June to September. Builds pad of rushes in dense vegetation near water. Clutch of 6-7 buff eggs with reddish spots.}
  {white-breasted-waterhen.jpg}

\birdentry{Indian Pond Heron}{\textit{Ardeola grayii}}
  {Myna+}
  {Very common resident}
  {Earthy brown when at rest but startlingly white in flight. Breeding birds develop golden-buff plumes on back. Yellow bill with dark tip.}
  {Wherever there is water - ponds, marshes, paddy fields}
  {Stands motionless or walks slowly when hunting. Suddenly stabs at prey. Feeds mainly on frogs, fish, crabs and aquatic insects.}
  {Colonial nester. Breeds June to September in mixed heronries. Platform nest of twigs. Clutch of 3-5 pale blue eggs.}
  {indian-pond-heron.jpg}

\birdentry{Black-crowned Night Heron}{\textit{Nycticorax nycticorax}}
  {Crow+}
  {Common resident}
  {Stocky heron with black crown and back, grey wings, white forehead and underparts. Red eyes distinctive. Juveniles brown and streaked.}
  {Wetlands and water bodies with good tree cover nearby}
  {Most active at dusk and dawn. Roosts by day in dense trees. Stands hunched at water's edge. Feeds mainly on fish, frogs, and aquatic insects.}
  {Colonial nester. Breeds April to September. Platform nest of sticks in trees near water. Clutch of 3-4 pale blue eggs. Both parents share duties.}
  {black-crowned-night-heron.jpg}

\birdentry{Cattle Egret}{\textit{Bubulcus ibis}}
  {Myna++}
  {Very common resident}
  {Stocky white egret with short thick neck. Breeding birds develop rich orange-buff plumes on head, neck and back. Yellow bill becomes orange-red in breeding.}
  {Grasslands and agricultural areas, often following grazing cattle}
  {Constantly follows cattle and other livestock, catching disturbed insects. More terrestrial than other egrets. Often in flocks.}
  {Colonial nester. Breeds June to September. Platform nest in mixed heronries. Clutch of 3-4 pale blue eggs. Both parents incubate and feed young.}
  {cattle-egret.jpg}

\birdentry{Little Egret}{\textit{Egretta garzetta}}
  {Myna++}
  {Common resident}
  {Slender white egret with long neck, black bill and legs. Distinctive yellow feet. Breeding birds develop long lacy plumes on head, neck and back.}
  {All types of wetlands and water bodies}
  {Active hunter, running and darting after prey in shallow water. Often uses feet to stir up prey. More aquatic than Cattle Egret.}
  {Colonial nester. Breeds June to September in mixed heronries. Platform nest of sticks. Clutch of 3-5 pale greenish-blue eggs.}
  {little-egret.jpg}

\birdentry{Blue Rock Pigeon}{\textit{Columba livia}}
  {Myna+}
  {Very common resident}
  {Bluish-grey with iridescent green and purple sheen on neck. Two prominent black wingbars and dark terminal tail band. Many color variants in urban birds.}
  {Urban areas, especially around buildings and monuments}
  {Gregarious, feeds in flocks on ground. Swift direct flight. Highly adapted to human settlements.}
  {Breeds throughout year. Flimsy platform nest on ledges of buildings. Clutch of 2 white eggs. Both parents incubate and feed squabs with 'crop milk'.}
  {blue-rock-pigeon.jpg}

\birdentry{Spotted Dove}{\textit{Streptopelia chinensis}}
  {Myna}
  {Common resident}
  {Vinaceous-grey dove with black-and-white checkered patch on hindneck. Long graduated tail with white corners. Pale buffy-pink underparts.}
  {Gardens, groves, and light woodland across campus}
  {Usually in pairs. Walks sedately on ground searching for seeds. Characteristic mellow cooing call 'ku-kroo-ku'.}
  {Breeds mainly February to September. Flimsy platform nest of twigs in bushes or small trees. Clutch of 2 white eggs. Both parents share duties.}
  {spotted-dove.jpg}

\birdentry{Little Brown Dove}{\textit{Streptopelia senegalensis}}
  {Myna-}
  {Common resident}
  {Dainty dove, smaller and slimmer than Spotted Dove. Pinkish-brown above, vinous below. Blue-grey wings with rufous primaries. Distinctive speckled collar patch.}
  {Open areas with scattered trees and scrub}
  {More terrestrial than other doves. Frequently seen in pairs on ground or wires. Musical call 'proo-proo-proo'.}
  {Breeds year-round. Simple platform nest low in thorny trees. Clutch of 2 white eggs. Shared parental care.}
  {little-brown-dove.jpg}

\birdentry{Red-wattled Lapwing}{\textit{Vanellus indicus}}
  {Myna++}
  {Common resident}
  {Large plover with distinctive red fleshy wattles in front of eyes. Black-and-white head and neck, brown back, white belly. Red bill with black tip.}
  {Open areas, playgrounds, and margins of water bodies}
  {Bold and noisy. Runs in short bursts, stopping abruptly. Famous for persistent 'did-he-do-it' calls day and night.}
  {Breeds mainly March to August. Nest is shallow scrape on ground, often on gravel. Clutch of 4 stone-colored eggs with black blotches. Both parents very aggressive in nest defense.}
  {red-wattled-lapwing.jpg}

\birdentry{Asian Koel}{\textit{Eudynamys scolopaceus}}
  {Crow-}
  {Common resident}
  {Striking sexual dimorphism - male glossy blue-black with crimson eye; female brown with white spots and bars. Both with heavy pale green bill.}
  {Wooded areas and gardens with fruiting trees}
  {More often heard than seen. Male's loud 'ku-ooo' calls increase in summer. Mainly frugivorous but also eats insects and eggs.}
  {Brood parasite primarily on House Crows. Breeding season March to August coinciding with hosts. Female lays one egg per crow nest.}
  {asian-koel.jpg}

\birdentry{Grey-bellied Cuckoo}{\textit{Cacomantis passerinus}}
  {Bulbul-}
  {Common summer visitor}
  {Grey bird with dark grey breast and pale belly. Yellow eye-ring distinctive. Female browner. Juvenile heavily barred below.}
  {Found in wooded areas across campus.}
  {Rather secretive. Known for distinctive ascending whistle 'pee-pee-pee-pee'. Feeds mainly on insects, especially hairy caterpillars.}
  {Brood parasite, laying eggs mainly in nests of Prinias and Cisticolas. Breeding season coincides with hosts, March to September.}
  {grey-bellied-cuckoo.jpg}

\birdentry{Common Hawk-Cuckoo}{\textit{Hierococcyx varius}}
  {Myna++}
  {Common resident}
  {Remarkable resemblance to Shikra hawk in size, shape and plumage - an example of aggressive mimicry. Finely barred underparts.}
  {Found throughout wooded areas of campus.}
  {Known for crescendo call "brain-fever, brain-fever, BRAIN-FEVER", increasing in pitch and intensity. Feeds mainly on insects, especially caterpillars.}
  {Brood parasite, primarily targeting babblers. Peak breeding season March to July coinciding with host species.}
  {common-hawk-cuckoo.jpg}

\birdentry{Greater Coucal}{\textit{Centropus sinensis}}
  {Crow}
  {Common resident}
  {Large, black bird with chestnut wings and long tail. Red eyes and curved black bill distinctive. Female slightly larger.}
  {Found in dense vegetation across the campus.}
  {Skulking habit, often seen walking on ground. Deep resonant 'whoop-whoop-whoop' calls. Feeds on snakes, lizards, large insects, and bird eggs.}
  {Builds a domed nest in dense vegetation. Lays 3-5 white eggs. Both parents share incubation (18-19 days) and chick care.}
  {greater-coucal.jpg}

\birdentry{Rose-ringed Parakeet}{\textit{Psittacula krameri}}
  {Myna++}
  {Very common resident}
  {Bright green parakeet with long tail. Males have pink and black neck ring, females lack this. Red bill distinctive.}
  {Abundant throughout the campus, especially in wooded areas.}
  {Noisy, gregarious birds. Often in large communal roosts. Feeds mainly on fruits, seeds, buds, and grain crops.}
  {Breeds December to May. Nests in tree hollows. Lays 4-6 white eggs. Female incubates for 22-24 days while male feeds her.}
  {rose-ringed-parakeet.jpg}

\birdentry{Alexandrine Parakeet}{\textit{Psittacula eupatria}}
  {Crow}
  {Uncommon resident}
  {Much larger than Rose-ringed with massive cherry-red beak. Males have broad rose-pink and black collar. Distinctive maroon shoulder patch.}
  {Found in mature woodlands with large trees, less common than other parakeets.}
  {Similar habits to Rose-ringed but more wary. Feeds primarily on fruits, seeds, and grain. Harsh rolling calls in flight.}
  {Nests in large cavities of mature trees. Lays 2-4 white eggs. Female incubates for 23-26 days while male feeds her.}
  {alexandrine-parakeet.jpg}

\birdentry{Plum-headed Parakeet}{\textit{Psittacula cyanocephala}}
  {Myna}
  {Common resident}
  {Male has plum-red head with black collar and blue nape; female bluish-grey headed. Both have green body, yellow-tipped tail and red shoulder patch.}
  {Regular visitor to wooded areas of the campus.}
  {Usually in pairs or small flocks. More arboreal than other parakeets. Feeds mainly on wild figs, fruits, and ripening grain.}
  {Nests in tree hollows, February to May. Lays 4-6 white eggs. Female incubates for 21-23 days while male feeds her.}
  {plum-headed-parakeet.jpg}

\birdentry{House Swift}{\textit{Apus affinis}}
  {Myna-}
  {Very common resident}
  {Small swift with white rump and shallow-forked tail. Overall dark brown-black plumage. Wings long and scythe-shaped.}
  {Common throughout urban areas of the campus.}
  {Aerial feeder, constantly on wing. Characteristic screaming parties near nesting sites. Often seen in large groups making low sweeping flights.}
  {Colonial nester, builds retort-shaped mud nests under building eaves and bridges. Several pairs may share common entrance tunnel. Lays 2-3 white eggs.}
  {house-swift.jpg}

\birdentry{Asian Palm Swift}{\textit{Cypsiurus balasiensis}}
  {Sparrow-}
  {Common resident}
  {Smaller than House Swift, slender with deeply forked tail. Overall sooty brown. Distinctive stiff, fluttering wing beats.}
  {Found around palm trees, especially Borassus and Coconut palms.}
  {Very fast flier with quick directional changes, rarely glides. Often seen around single palms which they circle repeatedly.}
  {Breeds year-round. Attaches tiny cup nest to underside of palm frond using saliva. Lays 2-3 elongated white eggs.}
  {asian-palm-swift.jpg}

\birdentry{White-breasted Kingfisher}{\textit{Halcyon smyrnensis}}
  {Myna+}
  {Common resident}
  {Large kingfisher with bright blue back, chocolate brown head, and white breast. Large red bill distinctive. Broad blue wing-band visible in flight.}
  {Found near water bodies and in gardens across campus.}
  {Bold and noisy. Maintains feeding territory which it advertises with loud screaming calls. Feeds on fish, frogs, lizards, insects, and even small birds and mice.}
  {Breeds March to June. Nests in horizontal tunnel (1m deep) in earth banks. Lays 4-7 round white eggs. Both parents incubate and feed young.}
  {white-breasted-kingfisher.jpg}

\birdentry{Common Kingfisher}{\textit{Alcedo atthis}}
  {Bulbul-}
  {Common resident}
  {Small kingfisher with brilliant blue upperparts and orange underparts. Long bill and short tail. Female has reddish lower mandible.}
  {Found near water bodies and ponds on campus. Each bird maintains linear territory along water.}
  {Perches low over water, dives vertically for fish with splash. Often hovers before plunging. Sharp 'chee-chee' flight call.}
  {Breeds February to April. Nests in 60-90cm tunnel in earth bank. Lays 5-7 round white eggs. Both parents incubate and feed young.}
  {common-kingfisher.jpg}

\birdentry{Asian Green Bee-eater}{\textit{Merops orientalis}}
  {Sparrow+}
  {Common resident}
  {Small slender green bird (18-20 cm) with distinctive elongated central tail feathers. Rich golden-green plumage with turquoise-blue throat patch bordered by black gorget. Black eye-stripe running through red-brown eyes. Slender curved bill. Non-breeding birds duller with shorter tail streamers.}
  {Open areas across campus with scattered trees, particularly around the administrative buildings and sports grounds.}
  {Highly aerial, characteristically hawking insects from prominent perches on bare branches or wires. Makes graceful swooping flights to catch flying insects, especially bees and wasps, which it subdues by repeatedly striking against perch. Often seen in loose groups of 3-10 birds.}
  {Colonial nester, breeding season principally March-June. Nests in self-excavated tunnels in earth banks, 1-2 m deep ending in oval chamber. Lays 4-7 spherical glossy white eggs. Both parents participate in excavation, incubation and feeding young.}
  {asian-green-bee-eater.jpg}

\birdentry{Coppersmith Barbet}{\textit{Megalaima haemacephala}}
  {Bulbul+}
  {Common resident}
  {Small stocky bird (16.5-17 cm) with distinctive pattern: grass-green overall with crimson forehead, crown and throat patch. Black eye-stripe and throat-band. Yellow patches around eyes and at base of lower mandible. Heavy, chisel-like bill. Sexes alike.}
  {Wooded areas throughout campus, particularly partial to fig trees and other fruiting species.}
  {Sedentary and arboreal. Known for monotonous metallic tuk..tuk..tuk call repeated for long periods, resembling a coppersmith at work. Rather lethargic between feeding bouts. Predominantly frugivorous, especially fond of fig fruits, but also takes insects.}
  {Breeds February to April, occasionally year-round. Excavates nest hole in dead tree branch, usually on underside. Typical clutch 3-4 white eggs. Both parents share nest excavation, incubation (14 days) and care of young.}
  {coppersmith-barbet.jpg}

\birdentry{Dusky Crag Martin}{\textit{Hirundo concolor}}
  {Sparrow-}
  {Common resident}
  {Small martin (13-14 cm) with sooty brown plumage overall. Slightly paler below with indistinct streaking on throat. Short, slightly forked tail. Broad-based wings. Lacks the contrasting rump patch of other swallows. Sexes alike.}
  {Around campus buildings and rocky outcrops, particularly abundant near the main buildings and library complex.}
  {Aerial feeder, catching insects in flight with agile twists and turns. Often seen gliding close to building walls. More sedentary than other swallows, maintaining small territories around nesting sites. Flight more fluttering and less direct than other swallows.}
  {Breeds mainly February to April, but can breed year-round. Builds characteristic retort-shaped mud nest under overhangs on buildings or rock faces. Clutch typically 2-3 white eggs spotted with brown. Both parents share all nesting duties.}
  {dusky-crag-martin.jpg}

\birdentry{Barn Swallow}{\textit{Hirundo rustica}}
  {Sparrow}
  {Common winter visitor}
  {Graceful swallow (17-19 cm including long forked tail streamers). Steel-blue black above, creamy white below with rusty forehead and throat. Broad dark blue breast band. Female slightly duller with shorter tail streamers. Juveniles browner with shorter tails.}
  {Open areas across campus, particularly abundant over lawns and playing fields.}
  {Supremely graceful aerialist, coursing low over ground with fluid movements. Often perches on wires in groups. Most numerous early morning and evening when insects are abundant. Occasionally seen "drinking" on the wing from water bodies.}
  {Does not breed on campus; winters September to April. Palearctic migrant.}
  {barn-swallow.jpg}

  \birdentry{Red-rumped Swallow}{\textit{Hirundo daurica}}
  {Sparrow}
  {Common resident}
  {Similar to Barn Swallow but with rufous rump and shorter tail streamers.}
  {Found throughout campus, especially near buildings.}
  {Aerial feeder, often in mixed flocks with other swallows.}
  {Builds bottle-shaped mud nests under structures. Lays 3-4 white eggs. Both parents incubate for about 14-16 days and care for the chicks.}
  {red-rumped-swallow.jpg}


\birdentry{Wire-tailed Swallow}{\textit{Hirundo smithii}}
  {Sparrow}
  {Common resident}
  {Distinctive swallow (13-14 cm plus 5-7 cm wire-like tail streamers in adult male). Glossy steel-blue above with pure white underparts. Chestnut cap contrasts with blue back. Female similar but with shorter tail "wires". Young birds lack the wire-like tail extensions.}
  {Around water bodies and open areas on campus, particularly near the artificial ponds and wetlands.}
  {Expert flyer, typically forages lower than other swallows with characteristic gliding interspersed with rapid wing beats. Often perches on thin twigs or wires, their long tail "wires" distinctive. Usually seen in pairs or small parties.}
  {Breeds mainly March to October. Builds neat cup-shaped mud nest, usually under bridges or culverts over water. Clutch of 2-4 white eggs with brown speckles. Both parents share incubation and feeding duties.}
  {wire-tailed-swallow.jpg}

\birdentry{Long-tailed Shrike}{\textit{Lanius schach}}
  {Bulbul+}
  {Common resident}
  {Medium-sized shrike (23-25 cm) with long graduated tail. Head and upper mantle grey, back and wings rufous-brown. Black mask through eye extending to ear-coverts. Whitish underparts often tinged buff on flanks. Female slightly duller. Juvenile browner with faint barring below.}
  {Open areas with scattered trees, particularly around the sports fields and administrative blocks.}
  {Bold hunter, typically perches conspicuously on exposed branches, posts, or wires. Pounces on prey from perch, often impaling victims on thorns for later consumption. Has a varied diet including large insects, lizards, small birds, and rodents.}
  {Breeds March to September, peak in April-June. Builds substantial cup nest of twigs, grass, and roots in thorny bushes or small trees. Clutch of 3-6 eggs, pale greenish-white heavily spotted with brown. Incubation 14-16 days, mainly by female.}
  {long-tailed-shrike.jpg}

\birdentry{Golden Oriole}{\textit{Oriolus oriolus}}
  {Myna+}
  {Common resident}
  {Striking bird (24-25 cm) with marked sexual dimorphism. Male brilliant yellow with contrasting black wings, central tail feathers, and eye-stripe. Female and young olive-green above, whitish below with dark streaking. All forms show distinctive white wing patch.}
  {Canopy of well-wooded areas across campus, particularly in areas with fruiting trees.}
  {Shy and elusive despite bright coloration. More often heard than seen. Rich, fluty calls including characteristic 'pee-lo' whistle. Moves deliberately through canopy searching for fruits and insects.}
  {Breeds April to July. Constructs distinctive hammock-like nest suspended in horizontal tree fork, usually high up. Lays 2-4 glossy white eggs with sparse black spots. Both parents incubate for 14-15 days and feed young.}
  {golden-oriole.jpg}

\birdentry{Black Drongo}{\textit{Dicrurus macrocercus}}
  {Myna+}
  {Common resident}
  {Conspicuous glossy black bird (28-31 cm, including tail) with distinctly forked tail. Adult has bright crimson iris. Strong, slightly hooked black bill. Sexes alike. Juveniles duller with brownish underparts and white spots in axillaries.}
  {Ubiquitous across campus in open areas with scattered trees, particularly common around grazing cattle.}
  {Exceptionally bold and aggressive, regularly mobbing much larger birds. Highly aerial, makes acrobatic sallies to catch flying insects. Often uses larger animals as beaters to flush prey. Varied repertoire includes accomplished mimicry.}
  {Breeds March to August, peak in April-June. Builds neat cup nest in outer branches of trees. Clutch typically 3-5 eggs, salmon-pink to reddish-white with darker spots. Both parents incubate for about 14-15 days.}
  {black-drongo.jpg}

\birdentry{Ashy Drongo}{\textit{Dicrurus leucophaeus}}
  {Myna+}
  {Winter visitor}
  {Similar in size to Black Drongo (28-30 cm) but overall smoky grey instead of glossy black. Less deeply forked tail. Dark red-brown iris. Juvenile browner with white speckles on underparts.}
  {Prefers more densely wooded areas than Black Drongo, commonly seen in campus gardens and groves.}
  {Generally less aggressive than Black Drongo but equally accomplished flycatcher. Often joins mixed-species hunting parties in winter. Flight more buoyant with slower wing beats. Call softer and less harsh.}
  {Does not breed on campus. Migrant visitor from Himalayas, present September to April.}
  {ashy-drongo.jpg}

\birdentry{Common Myna}{\textit{Acridotheres tristis}}
  {Myna}
  {Very common resident}
  {Sturdy bird (23-25 cm) with distinctive yellow bill, legs, and bare skin patches behind eye. Rich brown body plumage with black head, throat and upper breast. Conspicuous white wing patches and outer tail corners. Sexes alike. Juvenile duller with darker bill.}
  {Abundant throughout campus, particularly around buildings and open areas.}
  {Highly adaptable and opportunistic. Walks with characteristic head-bobbing gait. Usually in pairs or small groups. Omnivorous diet includes insects, fruits, nectar, and human food scraps. Roosts communally, often in large numbers.}
  {Breeds primarily March to September. Nests in cavities in trees, buildings, and other structures. Clutch of 4-6 glossy blue eggs. Both parents incubate for 13-14 days and care for young.}
  {common-myna.jpg}

\birdentry{Jungle Myna}{\textit{Acridotheres fuscus}}
  {Myna}
  {Common resident}
  {Slimmer than Common Myna (22-24 cm). Overall ashy-grey plumage becoming darker on wings and tail. Distinctive tuft of erect feathers at base of bill. Orange-yellow bill and legs. Small patch of bare yellow skin behind eye. Sexes alike.}
  {Prefers more wooded areas than Common Myna, frequent in campus gardens and groves.}
  {More arboreal than Common Myna. Usually in pairs or small parties. Flight direct with rapid wing beats. Varied diet including fruits, nectar, insects, and small reptiles. Voice less harsh than Common Myna.}
  {Breeds March to August. Nests in tree hollows, sometimes in old woodpecker holes. Clutch of 3-5 pale blue eggs. Both parents participate in incubation (13-14 days) and feeding young.}
  {jungle-myna.jpg}

\birdentry{Brahminy Starling}{\textit{Sturnia pagodarum}}
  {Myna}
  {Common resident}
  {Distinctive starling (19-21 cm) with long crest. Overall vinaceous-buff plumage with grey wings and tail. Black cap extends to pointed crest. White cheeks and underparts. Bright yellow bill with bluish base. Sexes similar but female slightly duller.}
  {Found in well-wooded areas and gardens across campus, particularly where flowering trees occur.}
  {Usually seen in pairs or small groups. More arboreal than mynas. Agile flier, adept at hawking insects in air. Feeds extensively on tree nectar and berries. Has melodious calls including sweet whistles.}
  {Breeds April to September. Nests in tree holes and building cavities. Clutch of 3-5 pale blue eggs. Both parents share incubation (14-15 days) and feeding duties.}
  {brahminy-starling.jpg}

\birdentry{Oriental Magpie-Robin}{\textit{Copsychus saularis}}
  {Bulbul}
  {Common resident}
  {Striking black and white bird (19-20 cm) with long frequently cocked tail. Male glossy blue-black above with white wingbars and outer tail feathers. Female similar but grey-brown where male is black. Both sexes have gleaming white breast and abdomen.}
  {Common in wooded areas and gardens with good undergrowth.}
  {Bold and confiding. Frequently seen on ground with tail cocked. One of the finest songsters, repertoire includes rich melodious whistles and warbles. Most vocal at dawn. Territory fiercely defended by both sexes.}
  {Breeds mainly March to July. Nests in tree hollows, walls, or buildings. Clutch of 4-5 eggs, pale green-blue with reddish-brown spots. Female incubates for 12-13 days, both parents feed young.}
  {oriental-magpie-robin.jpg}

\birdentry{Spot-breasted Fantail}{\textit{Rhipidura albogularis}}
  {Sparrow-}
  {Common resident}
  {Small active flycatcher (17-18 cm including long fan-shaped tail). Dark brown above with spotted white breast and prominent white supercilium. Broad white tips to tail feathers conspicuous in display. Sexes alike but juveniles duller with less distinct spotting.}
  {Inhabits wooded areas and gardens with good tree cover across campus.}
  {Highly active and acrobatic, constantly on move with tail spread fan-wise. Makes short aerial sallies to catch insects, often returning to same perch. Frequently joins mixed hunting parties. Voice includes sweet trilling notes.}
  {Breeds mainly March to September. Builds beautiful tiny cup nest of fine grass bound with cobwebs on horizontal branch. Typically lays 2-3 cream-colored eggs with reddish spots. Both parents share nesting duties.}
  {spot-breasted-fantail.jpg}

\birdentry{Pied Bushchat}{\textit{Saxicola caprata}}
  {Bulbul-}
  {Common resident}
  {Small chat (14-15 cm) with marked sexual dimorphism. Male glossy black with conspicuous white wing patch and rump. Female warm brown above, paler below, with rufous rump. Both sexes have relatively short tail and thin black bill.}
  {Frequents open areas with scattered bushes, particularly around sports fields and building margins.}
  {Characteristic pose on prominent low perches from which makes short flights to catch insects. Males particularly conspicuous, often performing song-flights in breeding season. Frequently bobs tail when agitated.}
  {Breeds March to August, peak April-June. Nest well-concealed in grass tussock or base of bush. Clutch 3-5 pale blue-green eggs with rusty spots. Female incubates for 12-13 days, both parents feed young.}
  {pied-bushchat.jpg}

\birdentry{Red-vented Bulbul}{\textit{Pycnonotus cafer}}
  {Bulbul}
  {Very common resident}
  {Medium-sized bulbul (20-22 cm) with distinctive pointed black crest. Dark brown-black above with scale-like pattern on back. Brownish-white below with distinctive crimson-red vent. Narrow white tips to tail feathers. Sexes alike but juveniles duller with brownish crest.}
  {Ubiquitous across campus in all habitats, from dense gardens to open areas with scattered trees.}
  {Active and sprightly with loud cheerful calls. Usually in pairs or family parties. Highly adaptable, equally at home in trees and bushes. Diet includes fruits, flower nectar, and insects. Aggressive defender of feeding territories.}
  {Breeds practically year-round with peak February to May. Neat cup nest in bushes or small trees, 1-3 m high. Clutch typically 2-3 pinkish-white eggs with red-brown spots. Both parents share 12-14 day incubation and feeding duties.}
  {red-vented-bulbul.jpg}

\birdentry{Red-whiskered Bulbul}{\textit{Pycnonotus jocosus}}
  {Bulbul}
  {Common resident}
  {Distinctive bulbul (20-22 cm) with long pointed black crest and crimson-red ear patches ("whiskers"). Brown above, white below with prominent black gorget. Crimson patch under tail. Narrow white tips to tail feathers. Sexes alike, juveniles duller with shorter crest.}
  {Prefers well-wooded areas and gardens with dense shrubbery across campus.}
  {More arboreal than Red-vented Bulbul. Actively hops among branches with crest raised. Clear melodious calls including distinctive 'kick-pettigrew'. Frequently seen in pairs, territorial throughout year.}
  {Breeds mainly February to September. Builds neat cup nest in dense bushes or creepers. Clutch of 2-4 pale pink eggs with red-brown spots. Incubation 11-12 days by both parents, who also jointly feed young.}
  {red-whiskered-bulbul.jpg}

\birdentry{Large Grey Babbler}{\textit{Turdoides malcolmi}}
  {Myna+}
  {Common resident}
  {Large gregarious babbler (25-27 cm) with distinctive long, graduated tail. Overall grey-brown above with pale shaft-streaks giving streaked appearance. Dull white below. Strong curved bill. Pale yellow eyes. Sexes alike.}
  {Found in open scrub and gardens across campus, particularly areas with thorny vegetation.}
  {Highly social, moving in noisy parties of 6-10 birds. Feeds mainly on ground, methodically turning over leaves and debris. Members of group maintain constant contact with harsh chattering calls. Communal preening common.}
  {Cooperative breeder, March to September. Several group members help build untidy cup nest in thorny bushes. Clutch 3-4 deep turquoise-blue eggs. Group members share incubation (15-16 days) and feeding of young.}
  {large-grey-babbler.jpg}

\birdentry{Tickell's Blue Flycatcher}{\textit{Cyornis tickelliae}}
  {Bulbul-}
  {Common resident}
  {Small flycatcher (13-14 cm) with marked sexual dimorphism. Male bright blue above, orangish-rufous on throat and breast, white on belly. Female duller blue-grey above, pale rufous below. Both sexes have black bill and legs.}
  {Inhabits shaded areas in well-wooded parts of campus with good undergrowth.}
  {Rather shy and retiring. Makes short sallies from perch to catch insects, often returning to same spot. Most active at dawn and dusk. Males have sweet whistling song delivered from concealed perch.}
  {Breeds April to August. Nests in tree hollow or similar cavity, sometimes quite low. Clutch of 3-4 pale pink eggs with reddish-brown spots. Female does most incubation, both parents feed young.}
  {tickells-blue-flycatcher.jpg}

\birdentry{Ashy Prinia}{\textit{Prinia socialis}}
  {Sparrow-}
  {Common resident}
  {Small warbler (13-14 cm) with distinctive seasonal variation. Breeding birds ash-grey above, whitish below with rusty flanks; non-breeding birds warmer brown above. Long graduated tail frequently held cocked. Dark bill, pale legs. Sexes alike.}
  {Found throughout campus in gardens, scrub, and grassy areas with scattered bushes.}
  {Active and restless, constantly moving through vegetation. Makes harsh 'chee-chee' calls when alarmed. Notable for its "tailoring" skill in nest construction. Territory aggressively defended by both sexes.}
  {Breeds mainly June to September during monsoon. Builds deep purse-like nest by stitching together leaves. Clutch 3-5 bright brick-red eggs. Both parents share 12-13 day incubation and care of young.}
  {ashy-prinia.jpg}

\birdentry{Jungle Prinia}{\textit{Prinia sylvatica}}
  {Sparrow-}
  {Common resident}
  {Slightly larger than Ashy Prinia (14-15 cm). Distinctive seasonal dimorphism: breeding birds dark grey-brown above, whitish below with prominent dark streaks on breast; non-breeding birds paler with less distinct streaking. Long graduated tail. Pale legs. Sexes alike.}
  {Prefers more open habitats than Ashy Prinia, found in grassland and scrub areas across campus.}
  {More terrestrial than other prinias, often descending to ground. Flight peculiar - brief bursts with deeply undulating path. Distinctive rasping 'trree-trree' calls. Males sing from prominent perches in breeding season.}
  {Breeds June to September, coinciding with monsoon. Builds globular nest with side entrance in grass clumps close to ground. Clutch 3-5 greenish-white eggs with reddish speckles. Both parents share all nesting duties.}
  {jungle-prinia.jpg}

\birdentry{Common Tailorbird}{\textit{Orthotomus sutorius}}
  {Sparrow--}
  {Common resident}
  {Tiny warbler (13 cm) with relatively long, graduated tail usually held cocked. Bright olive-green above, whitish below. Rufous crown and cheeks distinctive. Male breeding has elongated central tail feathers. Dark bill, pale flesh-colored legs.}
  {Common throughout campus wherever there are bushes and small trees, even in built-up areas.}
  {Active and restless, constantly moving through vegetation searching for insects. Loud 'to-wit to-wit' calls repeated persistently. Notable for unique nest-building technique, stitching living leaves together with plant fiber or spider silk.}
  {Breeds mainly May to September. Creates remarkable nest by sewing together large leaves to form cavity, lined with soft materials. Clutch 3-5 reddish-white eggs with brown spots. Female incubates 12 days, both parents feed young.}
  {common-tailorbird.jpg}

\birdentry{Lesser Whitethroat}{\textit{Sylvia curruca}}
  {Sparrow-}
  {Common winter visitor}
  {Small warbler (13-14 cm) with relatively short tail. Grey-brown above, white below with faint buff wash on flanks. Distinctive dark mask through eye contrasting with white throat. First-winter birds browner. Bill short, dark with pale base.}
  {Found in wooded areas and scrub across campus during winter months.}
  {Active but somewhat skulking. Constantly moves through bushes gleaning insects from foliage. Characteristic rattling call notes. Often joins other small insectivorous birds in mixed feeding parties.}
  {Does not breed on campus; winters September to April. Breeds in temperate Eurasia.}
  {lesser-whitethroat.jpg}

\birdentry{House Sparrow}{\textit{Passer domesticus}}
  {Sparrow}
  {Very common resident}
  {Small, stocky bird (14-15 cm) with marked sexual dimorphism. Male has grey crown, chestnut nape, black bib, and grey underparts. Female dull brown above with broad buff supercilium, pale brown below. Both sexes have brown wings with single white bar.}
  {Common around buildings and human habitation across campus.}
  {Highly gregarious and adapted to human presence. Feeds mainly on ground, taking grains, seeds, and scraps. Noisy, with characteristic chirping calls. Males display by puffing feathers and hopping near females.}
  {Breeds throughout year with peak March to June. Nests in holes in buildings or dense creepers. Clutch 3-6 white eggs heavily marked with brown. Both parents incubate for 13-14 days and feed young.}
  {house-sparrow.jpg}

\birdentry{Baya Weaver}{\textit{Ploceus philippinus}}
  {Sparrow}
  {Common resident}
  {Sparrow-sized (15 cm) with seasonal sexual dimorphism. Breeding male bright yellow with brown crown, black face-mask, and brown wings; non-breeding male and female sparrow-like, warm brown above with pale supercilium and underparts.}
  {Found in grassland areas and reed beds on campus, especially near water bodies.}
  {Highly colonial during breeding season. Males construct elaborate pendant nests, often several in succession. Expert weavers, using long strips of grass and palm fronds. Feed mainly on grass seeds and paddy, occasionally taking insects.}
  {Breeds during monsoon, June to September. Males build multiple retort-shaped nests; females select and line chosen ones. Clutch 2-4 pure white eggs. Female alone incubates for 14-15 days and cares for young.}
  {baya-weaver.jpg}

\birdentry{Indian Robin}{\textit{Copsychus fulicatus}}
  {Sparrow}
  {Common resident}
  {Small thrush (16-17 cm) with marked sexual dimorphism. Male glossy jet-black with prominent white shoulder patches visible in flight. Female rich brown above, darker brown below. Both sexes have characteristic orange-rufous vent and under tail-coverts. Long tail frequently held cocked.}
  {Common in open areas with scattered bushes, rocky ground, and around buildings across campus.}
  {Terrestrial habits, running and hopping on ground with tail cocked at sharp angle. Bold and confiding. Males territorial year-round, frequently singing from exposed perches. Diet mainly insects and small invertebrates.}
  {Breeds mainly March to September, peak in April-May. Nest in holes in walls, banks, or among rocks. Clutch 2-4 pale blue-green eggs with reddish-brown spots. Female incubates 12-13 days, both parents feed young.}
  {indian-robin.jpg}

\birdentry{Oriental White-eye}{\textit{Zosterops palpebrosus}}
  {Sparrow-}
  {Common resident}
  {Tiny bird (10-11 cm) with distinctive white ring around eye. Upper parts olive-green, throat and under tail-coverts bright yellow, rest of underparts whitish. Small pointed bill slightly curved. Sexes alike.}
  {Found throughout wooded areas of campus, particularly in flowering trees and bushes.}
  {Very active and acrobatic, often hanging upside down to probe flowers and buds. Usually in small parties, maintaining contact with soft 'chee-chee' calls. Flight swift and direct. Diet includes insects, nectar, and small berries.}
  {Breeds March to September, peak April-June. Builds beautiful tiny cup nest suspended in tree fork. Clutch 2-4 pale blue eggs. Both parents share nest-building, 10-12 day incubation, and feeding young.}
  {oriental-white-eye.jpg}

\birdentry{Scaly-breasted Munia}{\textit{Lonchura punctulata}}
  {Sparrow-}
  {Common resident}
  {Small finch (11-12 cm) with distinctive scaly pattern on breast and flanks. Upper parts rich brown, underparts white with dark brown scaling. Stout, conical dark bill. Sexes alike but juveniles plain brown lacking scale pattern.}
  {Found in grassy areas, scrub, and around cultivation across campus.}
  {Highly social, usually in small flocks. Feeds mainly on grass seeds, often seen clinging to grass heads. Sweet twittering contact calls. Flight swift and direct, flock moving in tight formation.}
  {Breeds July to October, coinciding with grass seeding. Builds large, rough ball nest with side entrance in thorny bushes or tall grass. Clutch 4-6 pure white eggs. Both parents share all nesting duties including 13-14 day incubation.}
  {scaly-breasted-munia.jpg}

\birdentry{Rufous Treepie}{\textit{Dendrocitta vagabunda}}
  {Myna++}
  {Common resident}
  {Large, long-tailed crow (47-50 cm, half being tail). Head, neck and breast sooty-black, upper parts rich rufous-brown. Wings black with conspicuous white patch. Grey-tipped, graduated black tail. Sexes alike.}
  {Found in wooded areas across campus, particularly in areas with large trees.}
  {Bold and inquisitive. Moves through trees with series of glides and hops, long tail used for balance. Omnivorous diet includes fruits, insects, lizards, bird eggs and nestlings. Has range of loud, metallic calls.}
  {Breeds March to July. Builds neat cup nest of twigs high in tree fork. Clutch 3-5 pale greenish-white eggs with brown spots. Both parents build nest and feed young, female incubates alone for 16-18 days.}
  {rufous-treepie.jpg}

\birdentry{House Crow}{\textit{Corvus splendens}}
  {Crow}
  {Very common resident}
  {Medium-sized crow (42-44 cm) with distinctive grey neck collar. Body glossy black with slight purple-blue sheen. Grey extends to upper breast and nape. Heavy black bill. Sexes alike but juveniles duller with brownish cast to grey areas.}
  {Abundant throughout campus, especially near human habitation and dining areas.}
  {Highly intelligent and adaptable. Bold scavenger, taking wide variety of food including human refuse. Usually in pairs or small groups. Roosts communally. Wide repertoire of calls including familiar harsh 'caw'.}
  {Breeds mainly March to July. Builds untidy stick nest in trees. Clutch 4-5 pale blue-green eggs with brown spots. Female incubates for 16-17 days, fed on nest by male. Both parents feed young.}
  {house-crow.jpg}

\birdentry{Large-billed Crow}{\textit{Corvus macrorhynchos}}
  {Crow+}
  {Common resident}
  {Larger than House Crow (52-54 cm) with heavier, more arched bill. Entirely glossy black with purple-blue sheen. No grey collar. Wings relatively longer and broader. Massive bill distinctive. Sexes alike but juveniles duller.}
  {Found in wooded areas of campus and surrounding regions, less urban than House Crow.}
  {More solitary and wary than House Crow. Flight stronger with deeper wing beats. Primarily carnivorous, feeding on carrion, small vertebrates, large insects. Deep, resonant 'quaaaark' call distinctive.}
  {Breeds March to June. Builds large platform nest of sticks high in tall trees. Clutch 3-5 pale greenish-blue eggs heavily spotted with brown. Female incubates for 18-19 days while male feeds her.}
  {large-billed-crow.jpg}

\birdentry{Spotted Owlet}{\textit{Athene brama}}
  {Myna+}
  {Common resident}
  {Small, chunky owl (21 cm) with rounded head. Upper parts brown heavily spotted with white. Facial disc whitish with dark border. Underparts whitish with brown spots. Large yellow eyes and prominent white eyebrows. Sexes alike.}
  {Found in wooded areas and gardens across campus, particularly around old buildings and large trees.}
  {Crepuscular and partially diurnal. Often seen during day in tree hollows or on bare branches. Characteristic head-bobbing when alert. Hunts insects, small reptiles, mice from low perches. Various calls including harsh churring and whistles.}
  {Breeds February to April. Nests in tree hollows, old buildings. Clutch 3-5 round white eggs. Female incubates for 28-30 days while male provides food. Both parents feed young.}
  {spotted-owlet.jpg}

\birdentry{Mottled Wood Owl}{\textit{Strix ocellata}}
  {Crow+}
  {Uncommon resident}
  {Large owl (45-48 cm) with distinctive vermiculated plumage. No ear-tufts. Overall mottled grey-brown with white spots and bars. Round face with dark border. Dark eyes and yellow-green bill. Female slightly larger.}
  {Found in densely wooded areas of campus with mature tree cover.}
  {Strictly nocturnal. Roosts during day in thick foliage or tree hollows. Powerful silent flight. Hunts from perch, taking rats, birds, large insects. Deep resonant double hoot 'wu-woo' carries far at night.}
  {Breeds December to March. Uses natural tree hollows or old raptor nests. Clutch 2-3 white eggs. Female incubates about 30 days while male provides food.}
  {mottled-wood-owl.jpg}

\birdentry{Indian Grey Hornbill}{\textit{Ocyceros birostris}}
  {Crow}
  {Common resident}
  {Large grey bird (61-66 cm) with long tail and distinctive casque on bill. Overall grey with paler underparts. Long curved bill with prominent casque. Female smaller with smaller casque. Wings show white spots in flight.}
  {Found in wooded areas across campus with large old trees, especially fig trees.}
  {Primarily arboreal, moving methodically through canopy. Usually in pairs. Flight strong with heavy wing beats alternating with glides. Feeds mainly on figs and other fruits, occasionally taking lizards and insects.}
  {Breeds March to June. Female seals herself in tree cavity using own droppings and mud, leaving narrow slit through which male feeds her and chicks. Clutch 2-4 dull white eggs. Female emerges when chicks half-grown.}
  {indian-grey-hornbill.jpg}

\birdentry{Indian Roller}{\textit{Coracias benghalensis}}
  {Myna++}
  {Common resident}
  {Crow-sized (30-34 cm) with striking blue plumage. Brown head and back contrast with bright azure-blue wings and tail. Throat violet-blue with white streaks. Deep ultramarine wing patches conspicuous in flight. Sexes alike.}
  {Found in open areas with scattered trees across campus.}
  {Conspicuous percher on exposed branches and wires. Makes spectacular aerial displays with rolling flight and harsh calls. Hunts from perch, taking large insects, small lizards, frogs, and occasionally mice.}
  {Breeds March to July. Nests in tree hollows or holes in old buildings. Clutch 3-5 glossy white eggs. Both parents share incubation (17-19 days) and feeding duties.}
  {indian-roller.jpg}

\birdentry{Black-headed Cuckoo-shrike}{\textit{Coracina melanoptera}}
  {Myna+}
  {Common resident}
  {Medium-sized bird (23-24 cm). Male pale grey with glossy black head, throat and upper breast. Female similar but head grey with faint barring below. Both sexes have broad black wings with narrow white edge. Long, graduated tail.}
  {Found in wooded areas and gardens with good canopy cover.}
  {Methodically searches foliage for insects, moving deliberately through canopy. Often joins mixed-species hunting parties. Flight strong and direct. Call a harsh 'kree-kree' and various sharp notes.}
  {Breeds April to July. Builds small, shallow cup nest of fine twigs on horizontal branch. Clutch 2-3 pale grey-green eggs with dark brown spots. Both parents share nesting duties.}
  {black-headed-cuckoo-shrike.jpg}

\birdentry{Small Minivet}{\textit{Pericrocotus cinnamomeus}}
  {Sparrow}
  {Common resident}
  {Small slim bird (16-17 cm) with long tail. Striking sexual dimorphism: male black above with bright orange-red patches, orange-red below; female grey above with yellow patches, yellow below. Both show distinctive wing pattern in flight.}
  {Found in wooded areas throughout campus, particularly in areas with tall trees.}
  {Active and restless, moving through canopy in small parties. Constantly utters sweet whistling notes to maintain contact. Expert at catching insects on wing. Frequently joins mixed-species foraging flocks.}
  {Breeds March to June. Builds tiny, neat cup nest camouflaged with cobwebs high in tree fork. Clutch 2-4 pale green eggs with brown spots. Both parents share all nesting duties.}
  {small-minivet.jpg}

\birdentry{Tickell's Flowerpecker}{\textit{Dicaeum erythrorhynchos}}
  {Sparrow--}
  {Common resident}
  {Tiny bird (9-10 cm) with short tail. Grey-brown above, dull white below with faint streaking on breast. Distinctive pale median streak on crown. Short, curved bill. Sexes alike but male slightly brighter.}
  {Found wherever flowering and fruiting trees occur, especially parasitic Loranthus.}
  {Extremely active, constantly flitting between flowers and fruit. Sharp 'chick-chick' calls. Specialized in feeding on Loranthus berries, helping disperse this parasitic plant. Also takes nectar and small insects.}
  {Breeds mainly March to May, but can breed year-round. Builds beautiful pear-shaped hanging nest with side entrance. Clutch 2-3 white eggs. Both parents share nest-building, incubation and chick care.}
  {tickells-flowerpecker.jpg}

\birdentry{Grey Wagtail}{\textit{Motacilla cinerea}}
  {Sparrow}
  {Common winter visitor}
  {Slender bird (18-19 cm) including long tail. Grey upper parts, bright yellow underparts. Distinctive white supercilium and double wing-bars. Breeding male has black throat. Long tail constantly wagged up and down.}
  {Found near water bodies and damp areas on campus, including artificial ponds and drains.}
  {Active forager along water edges, running and walking while wagging tail. Catches insects both on ground and in flight. More solitary than other wagtails. Flight undulating with sharp 'tsi-tsi' calls.}
  {Does not breed on campus; winters September to April. Breeds in temperate Asia and Europe.}
  {grey-wagtail.jpg}

\birdentry{White Wagtail}{\textit{Motacilla alba}}
  {Sparrow}
  {Common winter visitor}
  {Medium-sized wagtail (17-19 cm) with long tail. Variable plumage: winter birds grey above, white below with black breast band. Face pattern varies between races. Distinctive walking gait with tail-wagging.}
  {Found in open areas, lawns, and near water bodies across campus.}
  {Walks gracefully on ground with characteristic tail-wagging. Often in small loose groups. Feeds mainly on ground-dwelling insects and small invertebrates. Flight strongly undulating with 'chiswick' call.}
  {Does not breed on campus; winters September to April. Several races visit in winter, differing in face patterns.}
  {white-wagtail.jpg}

\birdentry{White-browed Wagtail}{\textit{Motacilla maderaspatensis}}
  {Sparrow+}
  {Common resident}
  {Largest Indian wagtail (19-21 cm). Striking black and white plumage: black above with prominent white supercilium and wing patches. White below with bold black breast-band. Long black tail with white edges. Sexes alike.}
  {Found near water bodies and open grassy areas across campus, particularly around artificial ponds.}
  {Active forager, running and walking gracefully while wagging tail. Often in pairs, territorial throughout year. More attached to specific sites than other wagtails. Clear, melodious whistling calls.}
  {Breeds March to September, peak in April-May. Builds substantial nest in holes in walls or banks near water. Clutch 3-5 greyish-white eggs with brown speckles. Both parents share incubation and feeding duties.}
  {white-browed-wagtail.jpg}

\birdentry{Common Iora}{\textit{Aegithina tiphia}}
  {Sparrow-}
  {Common resident}
  {Small bird (13-14 cm) with seasonal sexual dimorphism. Breeding male black above with two white wing-bars, bright yellow below; non-breeding male and female green above, yellow below. Bill slender, slightly curved.}
  {Found in wooded areas and gardens with good tree cover, particularly where flowering trees occur.}
  {Active and acrobatic, constantly moving through foliage searching for insects. Sweet melodious calls including whistles and chattering notes. Often hangs upside down while foraging. Usually seen in pairs.}
  {Breeds April to September. Builds neat, tiny cup nest in tree fork, well-camouflaged with cobwebs. Clutch 2-3 pinkish-white eggs with dark spots. Both parents share nesting duties.}
  {common-iora.jpg}

\birdentry{Asian Tit}{\textit{Parus cinereus}}
  {Sparrow-}
  {Common resident}
  {Small, sprightly bird (14 cm) with distinctive pattern. Grey above, white below with black cap extending to nape. Bold black stripe down center of white breast and belly. White cheeks and nape patch conspicuous. Sexes alike.}
  {Found in wooded areas and gardens across campus, especially areas with mature trees.}
  {Active and acrobatic, often hanging upside down to probe bark and leaves. Usually in pairs or family parties. Various calls including sharp 'titee-titee' and whistles. Diet includes insects, seeds, and fruits.}
  {Breeds March to July. Nests in tree holes, sometimes using old barbet holes. Clutch 4-6 white eggs with red-brown spots. Female incubates, both parents feed young.}
  {asian-tit.jpg}

\birdentry{Purple Sunbird}{\textit{Cinnyris asiaticus}}
  {Sparrow--}
  {Common resident}
  {Tiny bird (10 cm) with long curved bill. Striking sexual dimorphism: breeding male metallic blue-black with purple sheen and pectoral tufts; non-breeding male and female olive-brown above, yellow below. Both forms show dark breast band.}
  {Found throughout campus wherever flowering plants occur.}
  {Very active, constantly moving between flowers. Hovers briefly while probing flowers with long bill. Males perform hovering display flights with excited twittering. Takes insects in addition to nectar.}
  {Breeds mainly February to May. Builds beautiful pendant nest with porch-like entrance. Clutch 2-3 greenish-white eggs with brown spots. Female builds and incubates, both parents feed young.}
  {purple-sunbird.jpg}

\birdentry{Purple-rumped Sunbird}{\textit{Leptocoma zeylonica}}
  {Sparrow--}
  {Common resident}
  {Tiny sunbird (10 cm) with curved bill. Male has metallic blue-green crown, maroon breast-band, purple rump; rest of underparts bright yellow. Female olive above, yellow below. Both sexes show pale throat, dark eye-stripe.}
  {Found in gardens and wooded areas with plenty of flowering plants.}
  {Highly active, moving rapidly between flowers. Less given to hovering than Purple Sunbird. Sweet whistling calls. Often in pairs, males aggressively defend flowering territories.}
  {Breeds year-round with peak February to May. Builds elongated pendant nest with porched entrance. Clutch 2 greenish-white eggs with brown markings. Female incubates, both parents feed young.}
  {purple-rumped-sunbird.jpg}

\chapter*{Index}
\addcontentsline{toc}{chapter}{Index}
\printindex

\end{document}
