\documentclass[aspectratio=169]{beamer}

% Basic packages
\usepackage{fontspec}
\usepackage{polyglossia}
\usepackage{xcolor}
\usepackage{url}

% Set up languages
\setdefaultlanguage[numerals=devanagari]{marathi}
\setotherlanguage{english}

% Font configuration
% Main Devanagari font
\newfontfamily\devanagarifont[
    Script=Devanagari,
    Language=Marathi,
    UprightFont=*-Regular,
    BoldFont=*-Bold,
    ItalicFont=*-Regular,
    BoldItalicFont=*-Bold,
    Scale=1.2
]{Noto Serif Devanagari}

% Devanagari sans serif font - required by polyglossia
\newfontfamily\devanagarifontsf[
    Script=Devanagari,
    Language=Marathi,
    UprightFont=*-Regular,
    BoldFont=*-Bold,
    Scale=1.2
]{Noto Sans Devanagari}

% Devanagari monospace font
\newfontfamily\devanagarifonttt[
    Script=Devanagari,
    Language=Marathi,
    UprightFont=*-Regular,
    BoldFont=*-Bold,
    Scale=1.0
]{Noto Sans Devanagari}

% Latin text font
\newfontfamily{\latintext}[
    Script=Latin,
    Ligatures=TeX,
    UprightFont=*-Regular,
    BoldFont=*-Bold,
    Scale=1.0
]{Noto Serif}

% Set main font
\setmainfont{Noto Serif Devanagari}[
    Script=Devanagari,
    Language=Marathi,
    UprightFont=*-Regular,
    BoldFont=*-Bold,
    Scale=1.2
]

% Set sans-serif font
\setsansfont{Noto Sans Devanagari}[
    Script=Devanagari,
    Language=Marathi,
    UprightFont=*-Regular,
    BoldFont=*-Bold,
    Scale=1.2
]

% Theme and color customization
\usetheme{Madrid}
\usecolortheme{owl}
\setbeamercolor{normal text}{fg=white}
\setbeamercolor{background canvas}{bg=black}
\setbeamercolor{structure}{fg=white}
\setbeamercolor{frametitle}{fg=white}
\setbeamercolor{title}{fg=white}
\setbeamercolor{section in toc}{fg=white}
\setbeamertemplate{navigation symbols}{}

% Required packages
\usepackage{graphicx}
\graphicspath{{../images/}}

% Fix image credit command to use Latin font for credits
\newcommand{\imagecredit}[1]{%
  \def\stripext##1.jpg{##1}%
  \edef\creditfile{\stripext#1_credit.txt}%
  {\scriptsize\em \latintext Credit: {\latintext \input{../images/\creditfile}}}%
}

% Configure hyperref after other packages
\PassOptionsToPackage{unicode}{hyperref}
\usepackage{hyperref}
\urlstyle{same}

% Fix URL font
\def\UrlFont{\latintext}

% Fix all places where Latin text appears
\newcommand{\latinphrase}[1]{{\latintext #1}}

\title{सावित्रीबाई फुले पुणे विद्यापीठ परिसरातले पक्षी}
\author{योगेश वाडदेकर}
\institute{सावित्रीबाई फुले पुणे विद्यापीठ}
\date{२०२५}

\begin{document}

\begin{frame}
    \maketitle
\end{frame}

\begin{frame}{परिसराचा आढावा}
    \begin{columns}[T]
        \column{0.5\textwidth}
        सावित्रीबाई फुले पुणे विद्यापीठ ({\latinphrase SPPU}) परिसर, शहराच्या मध्यभागी असलेले ४११ एकर जैवविविधता हॉटस्पॉट आहे.
        \begin{itemize}
            \item {\latinphrase 18.5529 N} {\latinphrase N} {\latinphrase 73.8352 E} {\latinphrase E} अक्षांश-रेखांश
            \item समुद्रसपाटीपासून ५६० मीटर उंची
            \item विविध अधिवास समृद्ध पक्षिजीवन
            \item वृक्षराजी, गवताळ प्रदेश, जलाशय
        \end{itemize}
        \column{0.5\textwidth}
        परिसरात १५० हून अधिक प्रकारची झाडे:
        \begin{itemize}
            \item वड ({\latinphrase\textit{Ficus benghalensis}})
            \item पिंपळ ({\latinphrase\textit{Ficus religiosa}})
            \item कडुलिंब ({\latinphrase\textit{Azadirachta indica}})
            \item काटेसावर ({\latinphrase\textit{Bombax ceiba}})
            \item अनेक शोभेची झाडे
        \end{itemize}
    \end{columns}
\end{frame}

\begin{frame}{कधी पाहावे}
    \begin{columns}[T]
        \column{0.5\textwidth}
        \textbf{दिवसाची वेळ}
        \begin{itemize}
            \item पहाटे (सकाळी ६:००-९:००)
            \item दुपारनंतर (४:००-६:३०)
            \item मध्यान्ह - शिकारी पक्षी
            \item संध्याकाळ - निशाचर प्रजाती
        \end{itemize}
        \column{0.5\textwidth}
        \textbf{हंगामानुसार}
        \begin{itemize}
            \item हिवाळा (नोव्हेंबर-फेब्रुवारी)
            \item पावसाळा (जून-सप्टेंबर)
            \item उन्हाळा (मार्च-मे)
            \item पावसाळ्यानंतर (ऑक्टोबर)
        \end{itemize}
    \end{columns}
\end{frame}

\begin{frame}{कसे पाहावे}
    \begin{columns}[T]
        \column{0.5\textwidth}
        \textbf{आवश्यक उपकरणे}
        \begin{itemize}
            \item दुर्बीण ({\latinphrase 8x42} किंवा {\latinphrase 10x42})
            \item क्षेत्र मार्गदर्शिका
            \item नोंदवही
        \end{itemize}
        \column{0.5\textwidth}
        \textbf{उपयुक्त ॲप्स}
        \begin{itemize}
            \item {\latinphrase eBird}
            \item {\latinphrase Merlin Bird ID}
            \item {\latinphrase BirdNet}
        \end{itemize}
    \end{columns}
\end{frame}

% Template for bird entries
\newcommand{\birdslide}[9]{%
\begin{frame}{#1 ({\latinphrase \textit{#2}})}
    \begin{columns}[T]
        \column{0.5\textwidth}
        \includegraphics[width=\textwidth,height=0.7\textheight,keepaspectratio]{#9}
        \imagecredit{#9}
        \column{0.5\textwidth}
        \textbf{आकार:} #3 \\
        \textbf{स्थिति:} #4 \\[0.5em]
        \textbf{वैशिष्ट्ये:} #5 \\[0.5em]
        \textbf{कुठे आढळतो:} #6 \\[0.5em]
        \textbf{सवयी:} #7 \\[0.5em]
        \textbf{घरटे:} #8
    \end{columns}
\end{frame}
}

% Bird entries (first 10 species)
\section{शिकारी पक्षी}
\birdslide{घार}{\latinphrase \textit{Milvus migrans}}
{घारीएवढा}
{अतिनेहमी आढळणारा स्थानिक}
{मोठा शिकारी पक्षी. काटेरी शेपटी आणि लांब पंख हे वैशिष्ट्य. पंख काळसर तपकिरी, तळाशी किंचित फिकट.}
{परिसरातील सर्वत्र}
{उंच आकाशात गरगर फिरताना दिसते. लहान सस्तन प्राणी, पक्षी आणि मृत प्राणी खाते.}
{डिसेंबर ते एप्रिल दरम्यान प्रजनन. उंच झाडांवर काड्या-कचऱ्यापासून घरटे.}
{black-kite.jpg}

\birdslide{काळा शराटी}{\latinphrase \textit{Pseudibis papillosa}}
{घारीएवढा}
{नेहमी आढळणारा स्थानिक}
{मोठा काळा पक्षी, मानेवर लाल पट्टा. लांब वाकडी चोच आणि डोक्यावरील उघडा लाल भाग ठळक.}
{परिसरातील मोकळ्या जागा, गवताळ प्रदेश आणि पाणवठे}
{हळूहळू चालत शिकार करतो. लहान गटात आढळतो. कीटक आणि लहान प्राणी खातो.}
{पावसाळ्यात प्रजनन. मोठ्या झाडांवर मंचासारखे घरटे.}
{red-naped-ibis.jpg}

\birdslide{बहिरी ससाणा}{\latinphrase \textit{Falco peregrinus}}
{घारीएवढा}
{दुर्मिळ हिवाळी पाहुणा}
{मोठा, शक्तिशाली ससाणा. वरचा भाग गडद करडा, खालून पट्टेदार. काळा 'मिशांसारखा' पट्टा विशिष्ट.}
{कधीकधी उंच इमारती आणि मोकळ्या जागांमध्ये}
{वेगवान आणि चपळ शिकारी. हवेत भरारी मारून पक्षी पकडतो.}
{परिसरात प्रजनन करत नाही; फक्त हिवाळी पाहुणा.}
{peregrine-falcon.jpg}

\birdslide{शिक्रा}{\latinphrase \textit{Accipiter badius}}
{मैनेपेक्षा मोठा}
{नेहमी आढळणारा स्थानिक}
{छोटा ससाणा, लाल डोळे आणि पट्टेदार छाती विशिष्ट. वरून करडा, खालून पांढरा, तांबूस पट्ट्यांसह.}
{परिसरातील वृक्षराजी आणि बागा}
{लहान पक्षी, सस्तन प्राणी आणि कीटक खातो. खूप चपळ शिकारी.}
{मार्च ते जुलै प्रजनन काळ. झाडांवर काड्या-पानांचे घरटे.}
{shikra.jpg}

\section{जलचर पक्षी}
\birdslide{पांढऱ्या छातीची पाणकोंबडी}{\latinphrase \textit{Amaurornis phoenicurus}}
{बुलबुलापेक्षा मोठी}
{नेहमी आढळणारा स्थानिक}
{मध्यम आकाराचा पक्षी. पांढरा चेहरा, गळा, छाती; गडद तपकिरी शरीर. चोचीजवळ लाल पट्टा.}
{परिसरातील पाणवठे आणि दलदलीचे भाग}
{पाण्याजवळ राहते. कीटक, छोटे मासे आणि वनस्पती खाते.}
{दाट वनस्पतीत घरटे.}
{white-breasted-waterhen.jpg}

\birdslide{ढोकरी}{\latinphrase \textit{Ardeola grayii}}
{मैनेपेक्षा मोठा}
{अतिनेहमी आढळणारा स्थानिक}
{मध्यम आकाराचा बगळा. तपकिरी शरीर, पांढरे पंख. पिवळी चोच आणि पाय.}
{परिसरातील पाणवठे आणि दलदलीचे भाग}
{मासे, बेडूक आणि कीटक खातो. पाण्याजवळ बसून शिकार करतो.}
{जून ते सप्टेंबर प्रजनन काळ. पाण्याजवळील झाडांवर घरटे.}
{indian-pond-heron.jpg}

\birdslide{रात्र ढोकरी}{\latinphrase \textit{Nycticorax nycticorax}}
{कावळ्यापेक्षा मोठा}
{दुर्मिळ स्थानिक}
{मध्यम आकाराचा बगळा. काळी टोपी, पाठ; करडे पंख; पांढरा तळभाग. लाल डोळे.}
{परिसरातील पाणवठे आणि दलदलीचे भाग}
{रात्री सक्रिय. मासे, बेडूक आणि कीटक खातो.}
{जून ते सप्टेंबर प्रजनन काळ. पाण्याजवळील झाडांवर घरटे.}
{black-crowned-night-heron.jpg}

\birdslide{गवती बगळा}{\latinphrase \textit{Bubulcus ibis}}
{मैनेपेक्षा मोठा}
{अतिनेहमी आढळणारा स्थानिक}
{मध्यम आकाराचा पांढरा बगळा. प्रजनन काळात डोके, छाती, पाठीवर केशरी पिसारे.}
{परिसरातील गवताळ भाग आणि मोकळ्या जागा}
{गुरांच्या जवळपास राहतो. कीटक आणि लहान प्राणी खातो.}
{जून ते सप्टेंबर प्रजनन काळ. वसाहतीत राहतो.}
{cattle-egret.jpg}

\birdslide{छोटा बगळा}{\latinphrase \textit{Egretta garzetta}}
{मैनेपेक्षा मोठा}
{नेहमी आढळणारा स्थानिक}
{मध्यम आकाराचा शुभ्र पांढरा बगळा. काळी चोच, काळे पाय, पिवळे पंजे.}
{परिसरातील पाणवठे आणि दलदलीचे भाग}
{मासे, बेडूक आणि कीटक खातो. पाण्यात उभा राहून शिकार करतो.}
{जून ते सप्टेंबर प्रजनन काळ. वसाहतीत राहतो.}
{little-egret.jpg}

\section{समारोप}
\begin{frame}{योगदान कसे करावे}
    \begin{itemize}
        \item {\latinphrase eBird} वर पक्षी नोंदी करा
        \item छायाचित्रे शेअर करा
        \item नवीन प्रजातींची माहिती द्या
        \item पक्षीमैत्री अधिवास जतन करा
        \item नैतिक पक्षीनिरीक्षण पद्धती वापरा
    \end{itemize}
    \vspace{1em}
    \centering
    {\latinphrase SPPU eBird Hotspot}: {\latinphrase \url{https://ebird.org/hotspot/L1838309}}
\end{frame}

\end{document}