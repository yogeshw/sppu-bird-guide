\documentclass[aspectratio=169]{beamer}

% Basic packages
\usepackage{fontspec}
\usepackage{polyglossia}
\usepackage{xcolor}
\usepackage{url}

% Set up languages
\setdefaultlanguage[numerals=devanagari]{marathi}
\setotherlanguage{english}

% Font configuration
% Main Devanagari font
\newfontfamily\devanagarifont[
    Script=Devanagari,
    Language=Marathi,
    UprightFont=*-Regular,
    BoldFont=*-Bold,
    ItalicFont=*-Regular,
    BoldItalicFont=*-Bold,
    Scale=1.2
]{Noto Serif Devanagari}

% Devanagari sans serif font - required by polyglossia
\newfontfamily\devanagarifontsf[
    Script=Devanagari,
    Language=Marathi,
    UprightFont=*-Regular,
    BoldFont=*-Bold,
    Scale=1.2
]{Noto Sans Devanagari}

% Devanagari monospace font
\newfontfamily\devanagarifonttt[
    Script=Devanagari,
    Language=Marathi,
    UprightFont=*-Regular,
    BoldFont=*-Bold,
    Scale=1.0
]{Noto Sans Devanagari}

% Latin text font
\newfontfamily\latintext{Noto Serif}[
    Script=Latin,
    Ligatures=TeX
]

% Set main font
\setmainfont{Noto Serif Devanagari}[
    Script=Devanagari,
    Language=Marathi,
    UprightFont=*-Regular,
    BoldFont=*-Bold,
    Scale=1.2
]

% Set sans-serif font
\setsansfont{Noto Sans Devanagari}[
    Script=Devanagari,
    Language=Marathi,
    UprightFont=*-Regular,
    BoldFont=*-Bold,
    Scale=1.2
]

% Theme and color customization
\usetheme{Madrid}
\usecolortheme{owl}
\setbeamercolor{normal text}{fg=white}
\setbeamercolor{background canvas}{bg=black}
\setbeamercolor{structure}{fg=white}
\setbeamercolor{frametitle}{fg=white}
\setbeamercolor{title}{fg=white}
\setbeamercolor{section in toc}{fg=white}
\setbeamertemplate{navigation symbols}{}

% Required packages
\usepackage{graphicx}
\graphicspath{{../images/}}

% Fix the imagecredit command to handle Latin text properly
\newcommand{\imagecredit}[1]{%
  \def\stripext##1.jpg{##1}%
  \edef\creditfile{\stripext#1_credit.txt}%
  {\scriptsize\latintext Credit: \input{../images/\creditfile}}%
}

% Configure hyperref after other packages
\PassOptionsToPackage{unicode}{hyperref}
\usepackage{hyperref}
\urlstyle{same}

% Fix URL font
\def\UrlFont{\latintext}

% Fix all places where Latin text appears
\newcommand{\latinphrase}[1]{{\latintext #1}}

% Fix scientific name and Latin text handling
\newcommand{\sciname}[1]{\mbox{\latintext\textit{#1}}}

\title{सावित्रीबाई फुले पुणे विद्यापीठ परिसरातले पक्षी}
\author{योगेश वाडदेकर}
\institute{सावित्रीबाई फुले पुणे विद्यापीठ}
\date{२०२५}

\begin{document}

\begin{frame}
    \maketitle
\end{frame}

\begin{frame}{परिसराचा आढावा}
    \begin{columns}[T]
        \column{0.5\textwidth}
        सावित्रीबाई फुले पुणे विद्यापीठ ({\latinphrase SPPU}) परिसर, शहराच्या मध्यभागी असलेले ४११ एकर जैवविविधता हॉटस्पॉट आहे.
        \begin{itemize}
            \item १८.५५२९ ऊ. ७३.८३५२ पू. अक्षांश-रेखांश
            %{\latinphrase 18.5529 N} {\latinphrase 73.8352 E} {\latinphrase E} अक्षांश-रेखांश
            \item समुद्रसपाटीपासून ५६० मीटर उंची
            \item विविध अधिवास समृद्ध पक्षिजीवन
            \item वृक्षराजी, गवताळ प्रदेश, जलाशय
        \end{itemize}
        \column{0.5\textwidth}
        परिसरात १५० हून अधिक प्रकारची झाडे:
        \begin{itemize}
            \item वड (\sciname{Ficus benghalensis})
            \item पिंपळ (\sciname{Ficus religiosa})
            \item कडुलिंब (\sciname{Azadirachta indica})
            \item काटेसावर (\sciname{Bombax ceiba})
            \item अनेक शोभेची झाडे
        \end{itemize}
    \end{columns}
\end{frame}

\begin{frame}{कधी पाहावे}
    \begin{columns}[T]
        \column{0.5\textwidth}
        \textbf{दिवसाची वेळ}
        \begin{itemize}
            \item पहाटे (सकाळी ६:००-९:००)
            \item दुपारनंतर (४:००-६:३०)
            \item मध्यान्ह - शिकारी पक्षी
            \item संध्याकाळ - निशाचर प्रजाती
        \end{itemize}
        \column{0.5\textwidth}
        \textbf{हंगामानुसार}
        \begin{itemize}
            \item हिवाळा (नोव्हेंबर-फेब्रुवारी)
            \item पावसाळा (जून-सप्टेंबर)
            \item उन्हाळा (मार्च-मे)
            \item पावसाळ्यानंतर (ऑक्टोबर)
        \end{itemize}
    \end{columns}
\end{frame}

\begin{frame}{कसे पाहावे}
    \begin{columns}[T]
        \column{0.5\textwidth}
        \textbf{आवश्यक उपकरणे}
        \begin{itemize}
            \item दुर्बीण ({\latintext 8x42} किंवा {\latintext 10x42})
            \item क्षेत्र मार्गदर्शिका
            \item नोंदवही
        \end{itemize}
        \column{0.5\textwidth}
        \textbf{उपयुक्त ॲप्स}
        \begin{itemize}
            \item {\latintext eBird}
            \item {\latintext Merlin Bird ID}
            \item {\latintext BirdNet}
        \end{itemize}
    \end{columns}
\end{frame}

% Template for bird entries
\newcommand{\birdslide}[9]{%
\begin{frame}{#1 (\sciname{#2})}
    \begin{columns}[T]
        \column{0.5\textwidth}
        \includegraphics[width=\textwidth,height=0.7\textheight,keepaspectratio]{#9}
        \imagecredit{#9}
        \column{0.5\textwidth}
        \textbf{आकार:} #3 \\
        \textbf{स्थिति:} #4 \\[0.5em]
        \textbf{वैशिष्ट्ये:} #5 \\[0.5em]
        \textbf{कुठे आढळतो:} #6 \\[0.5em]
        \textbf{सवयी:} #7 \\[0.5em]
        \textbf{घरटे:} #8
    \end{columns}
\end{frame}
}

% Bird entries (first 10 species with corrections)
\section{शिकारी पक्षी आणि बगळे}
\birdslide{घार}{Milvus migrans}
{घारीएवढा}
{अतिनेहमी आढळणारा स्थानिक}
{मोठा शिकारी पक्षी. काटेरी शेपटी. पंख काळसर तपकिरी. उंच आकाशात घिरट्या घालताना पंख वर वळलेले दिसतात.}
{परिसरातील सर्वत्र, विशेषतः मानवी वस्तीजवळ}
{कचऱ्यावर जगणारा. पिल्लांच्या काळात कीटक व लहान प्राणी खातो. कावळ्यांना हुसकावून त्यांचे अन्न हिसकावून घेतो.}
{डिसेंबर ते एप्रिल दरम्यान प्रजनन. उंच झाडांवर मोठे काड्यांचे घरटे. २-३ पांढरी अंडी घालतो.}
{black-kite.jpg}

\birdslide{काळा शराटी}{Pseudibis papillosa}
{घारीएवढा}
{नेहमी आढळणारा स्थानिक}
{मोठा काळा पक्षी, मानेवर लाल पट्टा व डोक्यावर उघडा लाल भाग. लांब वाकडी चोच. मानेला व पंखांना धातूसारकी झळाळी.}
{मोकळ्या जागा, गवताळ प्रदेश आणि ओलसर जमीन}
{छोट्या कळपात आढळतो. दलदलीत चोच खुपसून कीटक, गोगलगाय व लहान बेडूक खातो.}
{मार्च ते ऑगस्ट दरम्यान प्रजनन. उंच झाडांवर मंचासारखे घरटे. २-४ फिकट निळी अंडी.}
{red-naped-ibis.jpg}

\birdslide{बहिरी ससाणा}{Falco peregrinus peregrinator}
{घारीएवढा}
{दुर्मिळ स्थानिक}
{शक्तिशाली ससाणा. वरचा भाग काळसर स्लेटी, छाती व पोट पांढरे-तपकिरी पट्टेदार. डोक्यावरून काळा पट्टा जातो.}
{उंच इमारती व मोकळ्या जागांमध्ये}
{हवेत उंच भरारी मारून कबुतरे व पाणपक्षी यांच्यावर झडप घालतो. वेगवान व चपळ शिकारी.}
{फेब्रुवारी ते एप्रिल प्रजनन. खडकाळ कड्यावर किंवा उंच इमारतींवर घरटे. ३-४ तपकिरी ठिपक्यांची अंडी.}
{peregrine-falcon.jpg}

\birdslide{शिक्रा}{Accipiter badius}
{मैनेपेक्षा मोठा}
{नेहमी आढळणारा स्थानिक}
{छोटा ससाणा, लाल डोळे आणि पट्टेदार छाती विशिष्ट. वरून करडा, खालून पांढरा, तांबूस पट्ट्यांसह.}
{परिसरातील वृक्षराजी आणि बागा}
{लहान पक्षी, सस्तन प्राणी आणि कीटक खातो. खूप चपळ शिकारी.}
{मार्च ते जुलै प्रजनन काळ. झाडांवर काड्या-पानांचे घरटे.}
{shikra.jpg}

\birdslide{पांढऱ्या छातीची पाणकोंबडी}{Amaurornis phoenicurus}
{बुलबुलापेक्षा मोठी}
{नेहमी आढळणारा स्थानिक}
{मध्यम आकाराचा पक्षी. पांढरा चेहरा, गळा, छाती; गडद तपकिरी शरीर. चोचीजवळ लाल पट्टा.}
{परिसरातील पाणवठे आणि दलदलीचे भाग}
{पाण्याजवळ राहते. कीटक, छोटे मासे आणि वनस्पती खाते.}
{दाट वनस्पतीत घरटे.}
{white-breasted-waterhen.jpg}

\birdslide{ढोकरी}{Ardeola grayii}
{मैनेपेक्षा मोठा}
{अतिनेहमी आढळणारा स्थानिक}
{मध्यम आकाराचा बगळा. तपकिरी शरीर, पांढरे पंख. पिवळी चोच आणि पाय.}
{परिसरातील पाणवठे आणि दलदलीचे भाग}
{मासे, बेडूक आणि कीटक खातो. पाण्याजवळ बसून शिकार करतो.}
{जून ते सप्टेंबर प्रजनन काळ. पाण्याजवळील झाडांवर घरटे.}
{indian-pond-heron.jpg}

\birdslide{रात्र ढोकरी}{Nycticorax nycticorax}
{कावळ्यापेक्षा मोठा}
{दुर्मिळ स्थानिक}
{मध्यम आकाराचा बगळा. काळी टोपी, पाठ; करडे पंख; पांढरा तळभाग. लाल डोळे.}
{परिसरातील पाणवठे आणि दलदलीचे भाग}
{रात्री सक्रिय. मासे, बेडूक आणि कीटक खातो.}
{जून ते सप्टेंबर प्रजनन काळ. पाण्याजवळील झाडांवर घरटे.}
{black-crowned-night-heron.jpg}

\birdslide{गवती बगळा}{Bubulcus ibis}
{मैनेपेक्षा मोठा}
{अतिनेहमी आढळणारा स्थानिक}
{मध्यम आकाराचा पांढरा बगळा. प्रजनन काळात डोके, छाती, पाठीवर केशरी पिसारे.}
{परिसरातील गवताळ भाग आणि मोकळ्या जागा}
{गुरांच्या जवळपास राहतो. कीटक आणि लहान प्राणी खातो.}
{जून ते सप्टेंबर प्रजनन काळ. वसाहतीत राहतो.}
{cattle-egret.jpg}

\birdslide{छोटा बगळा}{Egretta garzetta}
{मैनेपेक्षा मोठा}
{नेहमी आढळणारा स्थानिक}
{मध्यम आकाराचा शुभ्र पांढरा बगळा. काळी चोच, काळे पाय, पिवळे पंजे.}
{परिसरातील पाणवठे आणि दलदलीचे भाग}
{मासे, बेडूक आणि कीटक खातो. पाण्यात उभा राहून शिकार करतो.}
{जून ते सप्टेंबर प्रजनन काळ. वसाहतीत राहतो.}
{little-egret.jpg}

\section{पारवे आणि होले}
\birdslide{पारवा}{Columba livia}
{मैनेपेक्षा मोठा}
{अतिनेहमी आढळणारा स्थानिक}
{जाडजूड करडा पक्षी. मानेवर इंद्रधनुषी पिसारे. पंखांवर दोन काळे पट्टे, शेपटीवर काळा पट्टा.}
{परिसरातील इमारती आणि मोकळ्या जागा}
{कळपाने राहतो. जमिनीवरून धान्य वेचतो. शहरी जीवनाशी जुळवून घेतो.}
{वर्षभर प्रजनन. इमारतींवर काड्यांचे घरटे.}
{blue-rock-pigeon.jpg}

\birdslide{ठिपकेवाला होला}{Streptopelia chinensis}
{मैनेएवढा}
{नेहमी आढळणारा स्थानिक}
{बारीक, लांब शेपटीचा पारवा. गुलाबी-करडा रंग. मानेवर विशिष्ट काळे-पांढरे ठिपके.}
{परिसरातील वृक्षराजी आणि बागांमध्ये}
{नेहमी जोडीने किंवा छोट्या कळपात. जमिनीवर धान्य आणि बिया खातो.}
{झाडे आणि झुडपांमध्ये सैल घरटे.}
{spotted-dove.jpg}

\birdslide{छोटा तपकिरी होला}{Streptopelia senegalensis}
{मैनेपेक्षा लहान}
{नेहमी आढळणारा स्थानिक}
{लहान आणि नाजूक पारवा. फिकट तपकिरी रंग, मानेवर जांभळी छटा. मानेच्या बाजूंवर काळे-लाल चौकटी नमुना.}
{परिसरातील सर्वत्र, विशेषतः मोकळ्या भागात}
{नेहमी जोडीने दिसतो. जमिनीवर बिया वेचतो. गोड संगीतमय आवाज करतो.}
{झुडपे आणि लहान झाडांवर साधे घरटे.}
{little-brown-dove.jpg}

\section{टिटवी आणि कोकिळे}
\birdslide{टिटवी}{Vanellus indicus}
{मैनेपेक्षा बरीच मोठी}
{नेहमी आढळणारा स्थानिक}
{मोठी टिटवी. डोळ्यांसमोर लाल मांसल वाढ विशिष्ट. तपकिरी पंख, काळे डोके व छाती, पांढरा चेहरा व तळभाग.}
{परिसरातील मोकळ्या भागात, विशेषतः पाणवठ्यांजवळ}
{"डिड-ही-डू-इट" या धोक्याच्या आवाजासाठी प्रसिद्ध. दिवसा-रात्री सक्रिय.}
{जमिनीवर उथळ खड्ड्यात घरटे. खडकाळ भागात.}
{red-wattled-lapwing.jpg}

\birdslide{कोकीळ}{Eudynamys scolopacea}
{कावळ्यापेक्षा लहान}
{नेहमी आढळणारा स्थानिक}
{नर चमकदार काळा, लाल डोळे. मादी तपकिरी, पांढरे ठिपके, खालून पट्टेदार. लिंगभेद स्पष्ट.}
{परिसरातील सर्वत्र, विशेषतः वृक्षराजीत}
{विशिष्ट "कू-ऊ" आवाजासाठी प्रसिद्ध. कावळ्यांच्या घरट्यात परजीवी. फळे व बेरी खातो.}
{कावळे व इतर पक्ष्यांच्या घरट्यात परजीवी म्हणून अंडी घालतो.}
{asian-koel.jpg}

\birdslide{कारुण्य कोकीळ}{Cacomantis passerinus}
{मैनेपेक्षा लहान}
{उन्हाळी पाहुणा}
{करडा पक्षी, छातीवर गडद करडा रंग, तळाशी फिकट. पिवळा डोळ्याभोवतीचा कडा विशिष्ट.}
{परिसरातील वृक्षराजीत}
{लपून-छपून राहतो. हळुवार शिट्टीसारखा आवाज करतो. मुख्यतः कीटक खातो.}
{फुलपाखरे व शिंपी पक्ष्यांच्या घरट्यात परजीवी म्हणून अंडी घालतो.}
{grey-bellied-cuckoo.jpg}

\birdslide{पावश्या}{Hierococcyx varius}
{मैनेपेक्षा बराच मोठा}
{नेहमी आढळणारा स्थानिक}
{शिक्र्यासारखा दिसणारा. खालून पट्टेदार. पिवळा डोळ्याभोवतीचा कडा. उडताना शिक्र्यासारखा दिसतो.}
{परिसरातील वृक्षराजीत सर्वत्र}
{विशिष्ट "ब्रेन-फीवर" आवाजासाठी प्रसिद्ध. कीटक व अळ्या खातो.}
{सातभाई व इतर पक्ष्यांच्या घरट्यात परजीवी.}
{common-hawk-cuckoo.jpg}

\birdslide{भारद्वाज}{Centropus sinensis}
{कावळ्याएवढा}
{नेहमी आढळणारा स्थानिक}
{मोठा काळा पक्षी. केशरी पंख व लांब शेपटी. लाल डोळे व वाकडी काळी चोच विशिष्ट.}
{परिसरातील दाट वनस्पतीत}
{लपून-छपून वावरतो. जमिनीवर चालताना दिसतो. खोल गुरगुरण्याचा आवाज.}
{दाट वनस्पतीत जमिनीजवळ छतासारखे घरटे.}
{greater-coucal.jpg}

\section{पोपट}
\birdslide{राजपोपट}{Psittacula krameri}
{मैनेपेक्षा बराच मोठा}
{अतिनेहमी आढळणारा स्थानिक}
{हिरवा पोपट, लांब शेपटी. नराच्या मानेभोवती गुलाबी व काळा पट्टा, मादीला नसतो.}
{परिसरातील सर्वत्र, विशेषतः वृक्षराजीत}
{आवाजी, कळपवासी. मोठ्या कळपात दिसतो. फळे, बिया व धान्य खातो.}
{डिसेंबर ते मे प्रजनन. झाडांच्या पोकळीत घरटे.}
{rose-ringed-parakeet.jpg}

\birdslide{करण पोपट}{Psittacula eupatria}
{कावळ्याएवढा}
{दुर्मिळ स्थानिक}
{राजपोपटापेक्षा मोठा. जाड लाल चोच. नराला गुलाबी व काळा गळपट्टा.}
{परिसरातील वृक्षराजीत, इतर पोपटांपेक्षा कमी}
{राजपोपटासारखीच सवय पण जास्त सावध. फळे व बिया खातो.}
{मोठ्या जुन्या झाडांच्या पोकळीत घरटे.}
{alexandrine-parakeet.jpg}

\birdslide{टोई पोपट}{Psittacula cyanocephala}
{मैनेएवढा}
{नेहमी आढळणारा स्थानिक}
{नराचे डोके जांभळे-लाल, मादीचे निळसर-करडे. दोघांचेही शरीर हिरवे व पिवळट शेपटीचे टोक.}
{परिसरातील वृक्षराजीत नियमित}
{नेहमी जोडीने किंवा लहान कळपात. इतर पोपटांपेक्षा जास्त झाडावर राहतो. फळे व बिया खातो.}
{झाडांच्या पोकळीत घरटे.}
{plum-headed-parakeet.jpg}

\section{पाकोळी आणि खंड्या}
\birdslide{घर पाकोळी}{Apus affinis}
{मैनेपेक्षा लहान}
{अतिनेहमी आढळणारा स्थानिक}
{लहान घारांडी, पांढरी कंबर व काटेरी शेपटी. गडद तपकिरी-काळा रंग.}
{परिसरातील इमारतींच्या परिसरात}
{हवेत तरंगत राहणारा. मोठ्या कळपात इमारतींभोवती घिरट्या घालताना दिसतो.}
{वसाहतीत राहणारा. इमारतींच्या छपराखाली मातीचे घरटे.}
{house-swift.jpg}

\birdslide{ताड पाकोळी}{Cypsiurus balasiensis}
{चिमणीपेक्षा लहान}
{नेहमी आढळणारा स्थानिक}
{घरच्या घारांडीपेक्षा लहान, बारीक शरीर व खोलगट काटेरी शेपटी. एकूण धुरकट तपकिरी.}
{परिसरातील ताडाच्या झाडांभोवती}
{अतिशय वेगवान उड्डाणक्षम, क्वचितच बसतो. ताडाभोवती दिसतो. हवेतील कीटक खातो.}
{वर्षभर प्रजनन. ताडाच्या पानाखाली लाळेने घरटे चिकटवतो.}
{asian-palm-swift.jpg}

\birdslide{पांढऱ्या छातीचा धीवर}{Halcyon smyrnensis}
{मैनेपेक्षा मोठा}
{नेहमी आढळणारा स्थानिक}
{मोठा खंड्या, चमकदार निळी पाठ, चॉकलेटी डोके व पांढरी छाती. मोठी लाल चोच ठळक.}
{परिसरातील पाणवठे व बागा}
{ठळकपणे बसून राहतो, शिकारीसाठी झडप घालतो. मासे, बेडूक, सरडे व मोठे कीटक खातो.}
{मार्च ते जून प्रजनन. जमिनीच्या कडेत भुयार खणून घरटे.}
{white-breasted-kingfisher.jpg}

\birdslide{सामान्य धीवर}{Alcedo atthis}
{बुलबुलापेक्षा लहान}
{नेहमी आढळणारा स्थानिक}
{लहान खंड्या, चमकदार निळी वरची बाजू व केशरी खालची बाजू. लांब चोच व छोटी शेपटी.}
{परिसरातील पाणवठे व तलावांजवळ}
{पाण्यावर खाली बसून मासे पकडतो. शिकारीपूर्वी मान हलवतो.}
{फेब्रुवारी ते एप्रिल प्रजनन. जमिनीच्या कडेत भुयार खणून घरटे.}
{common-kingfisher.jpg}

\birdslide{वेडा राघू}{Merops orientalis}
{चिमणीपेक्षा मोठा}
{नेहमी आढळणारा स्थानिक}
{बारीक हिरवा पक्षी, निळा गळा, काळी डोळ्यांची पट्टी. लांब मधली शेपटीची पिसे ठळक.}
{परिसरातील विरळ झाडांच्या मोकळ्या भागात}
{सुंदर उड्डाणपटू, हवेत कीटक पकडतो. तारा व उघड्या फांद्यांवर बसतो.}
{मार्च ते जून प्रजनन. जमिनीत भुयार खणून घरटे.}
{asian-green-bee-eater.jpg}

\section{तांबट आणि बर्बेट}
\birdslide{तांबट}{Megalaima haemacephala}
{बुलबुलापेक्षा मोठा}
{नेहमी आढळणारा स्थानिक}
{लहान, हिरवा पक्षी, किरमिजी कपाळ व गळा. पिवळा डोळ्याभोवतीचा भाग ठळक.}
{परिसरातील वृक्षराजीत सर्वत्र}
{धातूसारख्या 'टुक-टुक-टुक' आवाजासाठी प्रसिद्ध. मुख्यतः फळे, विशेषतः उंबर खातो.}
{मेलेल्या झाडांच्या खोडात किंवा फांदीत घरटे खणतो.}
{coppersmith-barbet.jpg}

\section{भिंगरी आणि पंकोळी}
\birdslide{धूसर कडा पंकोळी}{Hirundo concolor}
{चिमणीपेक्षा लहान}
{नेहमी आढळणारा स्थानिक}
{लहान, गडद तपकिरी घारूळी, किंचित काटेरी शेपटी. एकसमान धुरकट रंग.}
{परिसरातील इमारती व खडकाळ भागात}
{हवेत कीटक पकडणारी. इमारतींजवळ उडताना दिसते.}
{इमारतींवर व खडकांवर मातीचे वाटीसारखे घरटे.}
{dusky-crag-martin.jpg}

\birdslide{माळ भिंगरी}{Hirundo rustica}
{चिमणीएवढी}
{सामान्य हिवाळी पाहुणा}
{लांब, खोल काटेरी शेपटी. वर निळसर-काळी, कपाळ व गळा तांबूस, खालून फिकी.}
{हिवाळ्यात परिसरातील मोकळ्या भागात}
{सुंदर उड्डाणपटू, हवेत कीटक पकडते. तारांवर बसते.}
{परिसरात प्रजनन करत नाही; फक्त हिवाळी पाहुणी.}
{barn-swallow.jpg}

\birdslide{लाल पुठ्ठ्याची भिंगरी}{Hirundo daurica}
{चिमणीएवढी}
{नेहमी आढळणारा स्थानिक}
{माळ भिंगरीसारखीच पण लहान. तांबूस पुठ्ठा व मान विशिष्ट.}
{परिसरातील मोकळ्या भागात}
{माळ भिंगरीसारखीच कीटक खाणारी. मोठ्या कळपात उडते.}
{पूलांखाली व इमारतींवर मातीचे घरटे.}
{red-rumped-swallow.jpg}

\birdslide{तारबाली भिंगरी}{Hirundo smithii}
{चिमणीएवढी}
{नेहमी आढळणारा स्थानिक}
{निळी-काळी, पांढरी छाती. लांब तारासारखी शेपटी विशिष्ट.}
{परिसरातील मोकळ्या भागात, विशेषतः पाणवठ्यांजवळ}
{पाण्याजवळ उडत कीटक पकडते. तारांवर बसते.}
{इमारती व पुलांवर मातीचे घरटे.}
{wire-tailed-swallow.jpg}

\section{खाटिक व इतर पक्षी}
\birdslide{लांब शेपटीचा खाटीक}{Lanius schach}
{बुलबुलापेक्षा मोठा}
{नेहमी आढळणारा स्थानिक}
{करडा-तपकिरी पक्षी, काळा चेहरा व लांब शेपटी. कुंपणावर बसतो.}
{विरळ झाडी व मोकळ्या जागांमध्ये}
{कीटक व लहान प्राणी खातो. शिकार काट्यावर टोचून ठेवतो.}
{काटेरी झुडपात नीटनेटके घरटे बांधतो.}
{long-tailed-shrike.jpg}

\birdslide{हळद्या}{Oriolus oriolus}
{मैनेपेक्षा मोठा}
{नेहमी आढळणारा स्थानिक}
{नर सोनेरी पिवळा, काळी पंखे व शेपटी. मादी हिरवट-पिवळी.}
{परिसरातील वृक्षराजीत}
{लाजाळू स्वभाव, वृक्षमुकुटात राहतो. गोड बासरीसारखा आवाज. फळे व कीटक खातो.}
{एप्रिल ते जुलै प्रजनन. झाडावर झोळीसारखे घरटे.}
{golden-oriole.jpg}

\birdslide{कोतवाल}{Dicrurus macrocercus}
{मैनेपेक्षा मोठा}
{नेहमी आढळणारा स्थानिक}
{चमकदार काळा रंग, काटेरी शेपटी. जाड चोच व मजबूत पाय.}
{परिसरातील सर्वत्र, विशेषतः मोकळ्या जागांमध्ये}
{कीटक व लहान प्राणी खातो. इतर पक्ष्यांच्या आवाजाची नक्कल करतो.}
{मार्च ते जुलै प्रजनन. बाह्य फांदीवर वाटीसारखे घरटे.}
{black-drongo.jpg}

\birdslide{राखी कोतवाल}{Dicrurus leucophaeus}
{मैनेपेक्षा मोठा}
{हिवाळी पाहुणा}
{एकसमान राखाडी रंग, काटेरी शेपटी. जाड चोच.}
{परिसरातील वृक्षराजी व मोकळे भाग}
{कीटक व लहान प्राणी खातो. हवेतून शिकार करतो.}
{परिसरात प्रजनन करत नाही; फक्त हिवाळी पाहुणा.}
{ashy-drongo.jpg}

\section{साळुंकी व मैना}
\birdslide{साळुंकी}{Acridotheres tristis}
{मैना}
{नेहमी आढळणारा स्थानिक}
{तपकिरी शरीर, काळे डोके, पिवळी चोच व पाय. उडताना पंखांवर पांढरे पट्टे दिसतात.}
{परिसरातील सर्वत्र विपुल}
{कळपात राहते. कीटक, फळे व मनुष्याच्या अन्नावर जगते.}
{एप्रिल ते जुलै प्रजनन. इमारतींच्या पोकळीत घरटे.}
{common-myna.jpg}

\birdslide{जंगली मैना}{Acridotheres fuscus}
{मैना}
{नेहमी आढळणारा स्थानिक}
{साळुंकीसारखी पण अधिक गडद. डोळ्यामागे पांढरा पट्टा नाही.}
{परिसरातील वृक्षराजी व मोकळे भाग}
{साळुंकीसारखीच सवय पण कमी मानवी वस्तीत दिसते.}
{झाडांच्या पोकळीत घरटे.}
{jungle-myna.jpg}

\birdslide{ब्राह्मणी मैना}{Sturnia pagodarum}
{मैनेपेक्षा लहान}
{नेहमी आढळणारा स्थानिक}
{फिकट करडी, डोक्यावरचे लांब पिसे व काळी टोपी विशिष्ट. तांबूस शरीर, करडी पंखे.}
{परिसरातील वृक्षराजी व मोकळ्या जागा}
{कळपवासी, कीटक व फळे खाते. शिट्टीसारखा मधुर आवाज.}
{एप्रिल ते जुलै प्रजनन. झाडांच्या व इमारतींच्या पोकळीत घरटे.}
{brahminy-starling.jpg}

\birdslide{दयाळ}{Copsychus saularis}
{बुलबुल}
{नेहमी आढळणारा स्थानिक}
{नर काळा-पांढरा ठळक रंगसंगती, मादी अधिक करडी. लांब शेपटी वर धरतो.}
{परिसरातील वृक्षराजी व बागा}
{सुंदर गायक. जमिनीवर उड्या मारत कीटक शोधतो.}
{मार्च ते जुलै प्रजनन. झाडाच्या पोकळीत किंवा इमारतींवर घरटे.}
{oriental-magpie-robin.jpg}

\birdslide{पांढऱ्या कंठाची नाचण}{Rhipidura albogularis}
{चिमणीपेक्षा लहान}
{नेहमी आढळणारा स्थानिक}
{लहान काळा-पांढरा पक्षी, ठिपकेदार छाती, पांढरी भुवई व पंख्यासारखी शेपटी पसरून ठेवतो.}
{परिसरातील दाट झाडी असलेले भाग}
{नृत्य करत शिकार करतो. हवेत उडणारे कीटक पकडतो.}
{मार्च ते जुलै प्रजनन. फांदीवर कपासारखे घरटे.}
{spot-breasted-fantail.jpg}

\section{माळरानातील पक्षी}
\birdslide{कवड्या गप्पीदास}{Saxicola caprata}
{बुलबुलापेक्षा लहान}
{नेहमी आढळणारा स्थानिक}
{नर काळा, पांढरी कंबर व पंखांवर पट्टे; मादी तपकिरी, खालून फिकी.}
{परिसरातील विरळ झुडपे असलेले मोकळे भाग}
{झुडपे व कुंपणांवर ठळकपणे बसतो. छोटी उड्डाणे करून कीटक पकडतो.}
{मार्च ते ऑगस्ट प्रजनन. गवतात व झुडपात वाटीसारखे घरटे. ३-४ फिकट निळसर-हिरवी, तांबूस ठिपक्यांची अंडी.}
{pied-bushchat.jpg}

\birdslide{लाल बुड्या बुलबुल}{Pycnonotus cafer}
{मैना}
{नेहमी आढळणारा स्थानिक}
{तपकिरी शरीर, काळे डोके, पिवळी चोच व पाय. उडताना पंखांवर पांढरे पट्टे दिसतात.}
{परिसरातील सर्वत्र विपुल}
{कळपात राहते. कीटक, फळे व मनुष्याच्या अन्नावर जगते.}
{एप्रिल ते जुलै प्रजनन. इमारतींच्या पोकळीत घरटे.}
{common-myna.jpg}

\birdslide{शिपाई बुलबुल}{Pycnonotus jocosus}
{मैना}
{नेहमी आढळणारा स्थानिक}
{तपकिरी शरीर, काळे डोके, पिवळी चोच व पाय. उडताना पंखांवर पांढरे पट्टे दिसतात.}
{परिसरातील सर्वत्र विपुल}
{कळपात राहते. कीटक, फळे व मनुष्याच्या अन्नावर जगते.}
{एप्रिल ते जुलै प्रजनन. इमारतींच्या पोकळीत घरटे.}
{common-myna.jpg}

\birdslide{राखी सातभाई}{Turdoides malcolmi}
{मैना}
{नेहमी आढळणारा स्थानिक}
{तपकिरी शरीर, काळे डोके, पिवळी चोच व पाय. उडताना पंखांवर पांढरे पट्टे दिसतात.}
{परिसरातील सर्वत्र विपुल}
{कळपात राहते. कीटक, फळे व मनुष्याच्या अन्नावर जगते.}
{एप्रिल ते जुलै प्रजनन. इमारतींच्या पोकळीत घरटे.}
{common-myna.jpg}

\birdslide{टिकेलची निळी माशीमार}{Muscicapa tickelliae}
{मैना}
{नेहमी आढळणारा स्थानिक}
{तपकिरी शरीर, काळे डोके, पिवळी चोच व पाय. उडताना पंखांवर पांढरे पट्टे दिसतात.}
{परिसरातील सर्वत्र विपुल}
{कळपात राहते. कीटक, फळे व मनुष्याच्या अन्नावर जगते.}
{एप्रिल ते जुलै प्रजनन. इमारतींच्या पोकळीत घरटे.}
{common-myna.jpg}

\section{वटवट्या व शिंपी}
\birdslide{राखी वटवट्या}{Prinia socialis}
{चिमणीपेक्षा लहान}
{नेहमी आढळणारा स्थानिक}
{बारीक करडा पक्षी, लांब टप्प्याटप्प्यांची शेपटी. तळाशी फिकट.}
{परिसरातील गवताळ भाग व बागा}
{सतत हालचाल करणारा, टिक्-टिक् आवाज काढतो. कीटक खातो.}
{मे ते सप्टेंबर प्रजनन. गवतात व झुडपात पिशवीसारखे घरटे.}
{ashy-prinia.jpg}

\birdslide{रान वटवट्या}{Prinia sylvatica}
{चिमणीपेक्षा लहान}
{नेहमी आढळणारा स्थानिक}
{राखी वटवट्यापेक्षा मोठा, डोके व पाठ गडद करडी, भुवई ठळक.}
{परिसरातील गवताळ भाग व झुडपे}
{सतत हालचाल करत राहतो. कीटक व लहान बिया खातो.}
{जून ते सप्टेंबर प्रजनन. गवताच्या झुंबरात चेंडूसारखे घरटे.}
{jungle-prinia.jpg}

\birdslide{शिंपी}{Orthotomus sutorius}
{चिमणीपेक्षा बराच लहान}
{नेहमी आढळणारा स्थानिक}
{छोटा हिरवा पक्षी, तांबूस डोके, लांब अणकुचीदार चोच. नराला लांब शेपटी.}
{परिसरातील बागा व झुडपांमध्ये}
{चपळ व कसरती. पानांमध्ये फिरत कीटक खातो. टिक्-टिक् आवाज.}
{मुख्यतः मे ते सप्टेंबर प्रजनन. जिवंत पानांना शिवून अद्भुत घरटे बनवतो.}
{common-tailorbird.jpg}

\birdslide{छोटा शुभ्रकंठी वटवट्या}{Sylvia curruca}
{चिमणीपेक्षा लहान}
{सामान्य हिवाळी पाहुणा}
{करडा पक्षी, काळी टोपी व गाल, पांढरा घसा. शेपटी छोटी.}
{परिसरातील झुडपे व दाट वनस्पती}
{लपून-छपून राहतो. कीटक व लहान फळे खातो.}
{परिसरात प्रजनन करत नाही; फक्त हिवाळी पाहुणा.}
{lesser-whitethroat.jpg}

\section{मनोली व चिमणी}
\birdslide{चिमणी}{Passer domesticus}
{चिमणी}
{नेहमी आढळणारा स्थानिक}
{नर करडा-तपकिरी, काळा गळपट्टा व तांबूस मान; मादी एकसमान तपकिरी.}
{परिसरातील इमारती व मानवी वस्तीजवळ}
{कळपवासी, मुख्यतः धान्य व बिया खाते.}
{जवळपास वर्षभर प्रजनन. इमारतींच्या पोकळीत घरटे.}
{house-sparrow.jpg}

\birdslide{सुगरण}{Ploceus philippinus}
{चिमणी}
{नेहमी आढळणारा स्थानिक}
{प्रजननकाळात नर चमकदार पिवळा, तपकिरी पाठ; मादी व हंगामाबाहेरील नर फिके.}
{परिसरातील ताडाची झाडे व उंच गवत}
{कळपवासी. मुख्यतः धान्य व गवताची बी खातो.}
{पावसाळ्यात प्रजनन. नर अनेक टोपीसारखी घरटी बांधतो.}
{baya-weaver.jpg}

\birdslide{ठिपकेवाली मनोली}{Lonchura punctulata}
{चिमणीपेक्षा लहान}
{नेहमी आढळणारा स्थानिक}
{तपकिरी पक्षी, छातीवर ठिपक्यांचे नमुने, पोट पांढरे.}
{परिसरातील गवताळ भाग व बागांमध्ये}
{कळपाने राहते. बिया व धान्य खाते.}
{पावसाळ्यात प्रजनन. चेंडूसारखे घरटे बनवते.}
{scaly-breasted-munia.jpg}

\birdslide{चरक}{Copsychus fulicatus}
{चिमणी}
{नेहमी आढळणारा स्थानिक}
{नर काळा, लाल शेपटीखालचा भाग; मादी तपकिरी, लाल शेपटीखालचा भाग.}
{मोकळ्या जागा व कमी झाडी असलेला भाग}
{जमिनीवर उड्या मारत कीटक खातो. लांब शेपटी वर करतो.}
{मार्च ते जुलै प्रजनन. पोकळीत किंवा भेगांमध्ये घरटे.}
{indian-robin.jpg}

\birdslide{चष्मेवाला}{Zosterops palpebrosus}
{चिमणीपेक्षा लहान}
{नेहमी आढळणारा स्थानिक}
{छोटा जैतुनी-हिरवा पक्षी, डोळ्याभोवती पांढरा कडा विशिष्ट. पिवळा गळा व शेपटीखालचा भाग.}
{परिसरातील वृक्षराजी व बागांमध्ये}
{चपळ व कसरती. छोट्या कळपात पानांमध्ये फिरतो. कीटक व मधुरस खातो.}
{मार्च ते सप्टेंबर प्रजनन. झाडाच्या फांदीवर नीटनेटके वाटीसारखे घरटे.}
{oriental-white-eye.jpg}

\section{काकाटु व घुबडे}
\birdslide{टकाचोर}{Dendrocitta vagabunda}
{मैनेपेक्षा बराच मोठा}
{नेहमी आढळणारा स्थानिक}
{मोठा तांबूस व काळा पक्षी, लांब टप्प्याटप्प्यांची शेपटी विशिष्ट.}
{परिसरातील वृक्षराजी व मोकळ्या जागा}
{कीटक, फळे व अंडी खातो. बुद्धिमान व सामाजिक.}
{मार्च ते जुलै प्रजनन. झाडाच्या फांदीवर नीटनेटके वाटीसारखे घरटे.}
{rufous-treepie.jpg}

\birdslide{कावळा}{Corvus splendens}
{कावळा}
{नेहमी आढळणारा स्थानिक}
{डोक्यावर व मानेवर करडा भाग, इतरत्र काळा. जाड चोच.}
{परिसरातील सर्वत्र}
{बुद्धिमान व सामाजिक. विविध प्रकारचे अन्न खातो. मनुष्याच्या वस्तीजवळ राहतो.}
{मार्च ते जुलै प्रजनन. झाडांवर मंचासारखे घरटे.}
{house-crow.jpg}

\birdslide{डोमकावळा}{Corvus macrorhynchos}
{कावळ्यापेक्षा मोठा}
{नेहमी आढळणारा स्थानिक}
{शहरी कावळ्यापेक्षा मोठा, संपूर्ण काळा, जाड चोच.}
{परिसरातील वृक्षराजीत}
{मांसाहारी, शिकारी. कावळ्यापेक्षा कमी सामाजिक.}
{मार्च ते जून प्रजनन. उंच झाडांवर मोठे मंचासारखे घरटे.}
{large-billed-crow.jpg}

\birdslide{पिंगळा}{Athene brama}
{मैनेपेक्षा मोठा}
{नेहमी आढळणारा स्थानिक}
{छोटे घुबड, करडा रंग, पांढरे ठिपके. मोठे पिवळे डोळे.}
{परिसरातील वृक्षराजी व इमारतींच्या पोकळीत}
{रात्री सक्रिय. उंदीर व मोठे कीटक खातो.}
{डिसेंबर ते मार्च प्रजनन. झाडांच्या व इमारतींच्या पोकळीत घरटे.}
{spotted-owlet.jpg}

\birdslide{ठिपकेवाले वन घुबड}{Strix ocellata}
{कावळ्यापेक्षा मोठा}
{नेहमी आढळणारा स्थानिक}
{मोठे घुबड, तपकिरी रंग, सर्वांगावर पांढरे ठिपके. मोठे भुवयांचे कडे.}
{परिसरातील जुन्या वृक्षराजीत}
{रात्री सक्रिय. लहान सस्तन प्राणी व पक्षी खातो.}
{डिसेंबर ते मार्च प्रजनन. मोठ्या झाडांच्या पोकळीत घरटे.}
{mottled-wood-owl.jpg}

\section{धनेश व नीलकंठ}
\birdslide{भारतीय राखी धनेश}{Ocyceros birostris}
{कावळा}
{नेहमी आढळणारा स्थानिक}
{करडा पक्षी, काळी चोचीवर वाकडी वाढ. लांब पंख व टप्प्याटप्प्यांची शेपटी.}
{परिसरातील वृक्षराजीत}
{फळे व लहान प्राणी खातो. जोडीने राहतो. मोठा व घोगरा आवाज.}
{एप्रिल ते जून प्रजनन. मोठ्या झाडांच्या पोकळीत घरटे.}
{indian-grey-hornbill.jpg}

\birdslide{भारतीय नीलपंख}{Coracias benghalensis}
{मैनेपेक्षा बराच मोठा}
{नेहमी आढळणारा स्थानिक}
{निळा पक्षी, गुलाबी-तपकिरी छाती. उडताना पंखांवर निळे पट्टे ठळक दिसतात.}
{परिसरातील मोकळ्या जागा व विरळ झाडी}
{तारेवर बसून कीटक व लहान प्राणी टपतो. आकाशात कलाबाज्या करतो.}
{मार्च ते जुलै प्रजनन. झाडांच्या पोकळीत घरटे.}
{indian-roller.jpg}

\section{कोकीळ-खाटिक व गोमेट}
\birdslide{काळ्या डोक्याचा कोकीळ-खाटिक}{Coracina melanoptera}
{मैनेपेक्षा मोठा}
{नेहमी आढळणारा स्थानिक}
{करडा पक्षी, काळे डोके व पंख. मादी अधिक फिकी, पट्टेदार छाती.}
{परिसरातील वृक्षराजीत}
{लपून-छपून राहतो. फळे व कीटक खातो.}
{एप्रिल ते जुलै प्रजनन. फांदीवर छोटे वाटीसारखे घरटे.}
{black-headed-cuckoo-shrike.jpg}

\birdslide{छोटा गोमेट}{Pericrocotus cinnamomeus}
{चिमणी}
{नेहमी आढळणारा स्थानिक}
{नर काळा व नारिंगी; मादी करडी व पिवळी. दोघांनाही लांब शेपटी.}
{परिसरातील वृक्षराजी व बागांमध्ये}
{कळपाने राहतो. कीटक खातो. सक्रिय व चंचल.}
{मार्च ते जून प्रजनन. फांदीवर छोटे वाटीसारखे घरटे.}
{small-minivet.jpg}

\section{फुलटोचा व धोबी}
\birdslide{टिकेलचा फुलटोचा}{Dicaeum erythrorhynchos}
{चिमणीपेक्षा बराच लहान}
{नेहमी आढळणारा स्थानिक}
{छोटा करडा पक्षी, खालून पांढरा, वर गडद. लाल चोच विशिष्ट.}
{परिसरातील वृक्षराजीत}
{चपळ व कसरती. फळे व फुलांमधून मधुरस खातो.}
{वर्षभर प्रजनन. नाशपातीसारखे लटकते घरटे.}
{tickells-flowerpecker.jpg}

\birdslide{करडा धोबी}{Motacilla cinerea}
{चिमणी}
{सामान्य हिवाळी पाहुणा}
{बारीक पक्षी, लांब शेपटी, करडा वरचा भाग व पिवळा खालचा भाग. सतत शेपटी हलवत राहतो.}
{परिसरातील पाणवठे व ओलसर भागात}
{सक्रिय शिकारी, धावत-चालत शेपटी हलवत राहतो. जमिनीवरून व हवेतून कीटक पकडतो.}
{परिसरात प्रजनन करत नाही; फक्त हिवाळी पाहुणा.}
{grey-wagtail.jpg}

\birdslide{पांढरा धोबी}{Motacilla alba}
{चिमणी}
{सामान्य हिवाळी पाहुणा}
{काळा-पांढरा पक्षी, करडी पाठ, पांढरा चेहरा व खालचा भाग. सतत शेपटी हलवतो.}
{परिसरातील पाणवठे व मोकळ्या जागांमध्ये}
{जमिनीवरून कीटक व किडे खातो. हिवाळ्यात मोठ्या संख्येने येतो.}
{परिसरात प्रजनन करत नाही; फक्त हिवाळी पाहुणा.}
{white-wagtail.jpg}

\birdslide{पांढऱ्या भुवईचा धोबी}{Motacilla maderaspatensis}
{चिमणीपेक्षा मोठा}
{नेहमी आढळणारा स्थानिक}
{काळा-पांढरा पक्षी, ठळक पांढरी भुवई. सतत शेपटी हलवत राहतो.}
{परिसरातील पाणवठे व ओलसर भागात}
{पाण्याजवळ राहतो. कीटक व किडे खातो. जोडीने दिसतो.}
{मार्च ते सप्टेंबर प्रजनन. भिंतींच्या व कड्यांच्या भेगांमध्ये घरटे.}
{white-browed-wagtail.jpg}

\section{सुभग व रामगंगा}
\birdslide{सुभग}{Aegithina tiphia}
{चिमणीपेक्षा लहान}
{नेहमी आढळणारा स्थानिक}
{नर काळा व पिवळा (प्रजननकाळात), जैतुनी-तपकिरी वर, पिवळी खाली. मादी हिरवट.}
{परिसरातील फुलझाडे असलेल्या भागात}
{अत्यंत सक्रिय, फुलांसमोर तरंगतो. लांब वाकडी चोच मधुरसासाठी.}
{एप्रिल ते सप्टेंबर प्रजनन. फांदीत नीटनेटके वाटीसारखे घरटे.}
{common-iora.jpg}

\birdslide{कवडी रामगंगा}{Parus cinereus}
{चिमणीपेक्षा लहान}
{नेहमी आढळणारा स्थानिक}
{छोटा करडा पक्षी, काळे डोके व गळपट्टा. पांढरे गाल विशिष्ट.}
{परिसरातील वृक्षराजीत सर्वत्र}
{सक्रिय, छोट्या कळपात वृक्षमुकुटात फिरतो. गोड शिट्टीसारखा आवाज.}
{मार्च ते जुलै प्रजनन. झाडांच्या पोकळीत घरटे.}
{asian-tit.jpg}

\section{शिंजीर}
\birdslide{जांभळा शिंजीर}{Cinnyris asiaticus}
{चिमणीपेक्षा बराच लहान}
{नेहमी आढळणारा स्थानिक}
{नर चमकदार जांभळा-काळा, मादी फिकट करडी. दोघांनाही लांब वाकडी चोच.}
{परिसरातील फुलझाडे असलेल्या भागात}
{फुलांमधून मधुरस पितो. हवेत तरंगत राहतो.}
{वर्षभर प्रजनन. नाशपातीसारखे लटकते घरटे.}
{purple-sunbird.jpg}

\birdslide{जांभळ्या पुठ्ठ्याचा शिंजीर}{Leptocoma zeylonica}
{चिमणीपेक्षा बराच लहान}
{नेहमी आढळणारा स्थानिक}
{नर जांभळा-हिरवा चमकदार, पुठ्ठा जांभळा; मादी फिकट पिवळी.}
{परिसरातील फुलझाडे असलेल्या भागात}
{फुलांमधून मधुरस पितो. हवेत तरंगत राहतो.}
{मुख्यतः फेब्रुवारी ते मे प्रजनन. झोळीसारखे लटकते घरटे.}
{purple-rumped-sunbird.jpg}

% Add the final frame again
\begin{frame}{योगदान कसे करावे}
    \begin{itemize}
        \item {\latintext eBird} वर पक्षी नोंदी करा
        \item छायाचित्रे शेअर करा
        \item नवीन प्रजातींची माहिती द्या
        \item पक्षीमैत्री अधिवास जतन करा
        \item नैतिक पक्षीनिरीक्षण पद्धती वापरा
    \end{itemize}
    \vspace{1em}
    \centering
    {\latintext SPPU eBird Hotspot:} {\latintext\url{https://ebird.org/hotspot/L1838309}}
\end{frame}

\end{document}